\documentclass[a4paper]{article}

\usepackage[dvips]{changebar}

%% an environment for bits of the texts which have been changed
% \newenvironment{changed}{\chgbarbegin}{\chgbarend}

\setlength{\parindent}{0mm}
\setlength{\parskip}{.5ex}


\begin{document}

{\Large{The Future of the UniForM workbench}}

\rule{\textwidth}{.2pt} 

%% \\[1ex]

\today 
\hfill
\begin{tabular}[t]{l}
George Russell \\
Christoph L\"uth \\
Till Mossakowski
\end{tabular}\\[3ex]


This paper explores the future development of the UniForM workbench.
We do this in two stages: first from the prospective user's point of
view, describing how the workbench is to be used, and secondly from
the prospective tool integrator's point of view, describing how the
workbench would be \emph{instantiated} to a particular setting.

Since the UniForM workbench is a generic tool integration framework,
in order to apply it to a particular setting (e.g. developing Haskell
programs or formal program development in CASL), one has to make
changes and adaptions at compile time; this gives a particular
instance of the workbench.


\section{The User's Perspective}

This section describes how a user will interact with the UniForM
workbench. We describe the interaction in two stages: first, those
interactions which are the same for all instantiations of the
workbench, followed by more specific instances of the workbench.


\subsection{Generic Usage}

At the moment, there are two main interaction modes: the version
graph, and the development graph. The version graph shows all
available versions, and let the user pick some to work on, serving as
a session management as well. The development graph is where the
actual development takes place, i.e. objects are changed and new
objects are created. 

\subsubsection{Version Graph}

When the user starts the workbench, he is presented with the
\emph{version graph}. The version graph shows all versions present in
the workbench, as a dag (where nodes correspond to versions, and links 
to the is-revised-by relation). 

Note that in terms of the workbench, a \emph{version} is a
\emph{configuration} of versioned objects (containing files,
directories etc.) In CVS terms, a version corresponds to a particular
tag. \emph{We may want to find a better name for this?}


Some versions will be \emph{checked out}, i.e. copies are placed in
the user's workspace\footnote{Probably but not necessarily his local
  file space.}; we call them \emph{working versions}. Those will be
highlighted depending on their status:
\begin{itemize}
\item They have been checked out but there no changes yet (wrt. to the
  repository);
\item There are local changes wrt. to the repository.
\item There are changes in the repository but not locally.
\item There are both local changes and changes in the repository.
\end{itemize}

The user can now:
\begin{itemize}
\item Select a working version. The workbench will close the version
  graph and open the development graph
  (Sect.~\ref{sec:development-graph}).
\item Select one version to be checked out into his local workspace.
\item Select a working version to be merged with updates the
  repository. This may cause \emph{merge conflicts}, if the merge
  operation fails. 
\item Select a working version to check it back into the
  repository. It will fail if there are changes in the repository wrt
  to the working version, requiring to user to merge the working
  version with the repository first, or if there are outstanding,
  unresolved merge conflicts.
\item Select two (or more) versions to merge. This will produce a new
  working version.
\end{itemize}

The last working version the user has been working with (if any) will
be highlighted; a button will let the user resume his work there
directly. 

\emph{Future expansions} here might include the ability to lock
versions and outdate versions.

\subsubsection{Development Graph}
\label{sec:development-graph}

The \emph{development graph} shows the development objects (source
files, specifications, \&c). Here, the actual development takes
place, and so the development graph is fairly instance-specific. 

The development graph has 
\begin{itemize}
\item as nodes, the \emph{development objects}. These include folders
  (directories) and files. 

  Objects (\emph{or only files?}) have \emph{attributes}; some of
  these are instance-specific, but one attribute which they are
  required to have is the \emph{type}. The particular types are
  defined by the instance.

\item as edges, \emph{links} between development objects. Links
  correspond to dependencies between objects.
  
  One kind of link which is provided for all instances is the
  \emph{contains}-relationship between folders and other folders or
  files. Other links are instance-specific.
\end{itemize}

The user can now do the following:
\begin{itemize}
\item Select an operation on one particular object (from a pop-up menu
  on the object): 
  \begin{itemize}
  \item Delete the object.
  \item Rename the object, or move it (to somewhere else in the folder
    hierarchy.) Moving is probably achied by drag\&drop rather than
    the pop-up menu, though.
  \item Change those attributes which can be changed.
  \item Open or close a folder, causing the contained objects to
    appear or disappear.
  \item Edit the object. The default editing operation is to put the
    object into an XEmacs buffer, but particular types of objects may
    not be editable at all, or have a more particular edit operation.
    Editing may result in a changed object, which may result in links
    being changed. \emph{For the time being, for implementation
      reasons, we restrict ourselves to outgoing links changing here.}
    
    To resolve change conflicts (see above), we require the user to
    edit all objects for which a merge conflict occured (i.e. the
    merge operation failed).
  \item Select instance-specific operations.
  \end{itemize}
\item Select operations which work on the whole graph:
  \begin{itemize}
  \item Return to the version graph.
  \item Update this working version (\emph{This may be a future
      expansion, we don't really need this here at this point.}).
  \item Create new objects. Not all types of objects can be created
    from scratch.
  \end{itemize}
\end{itemize}

\emph{Future expansions:} For particular instances of the workbench,
where the development graph is more a development tree, displaying the
development graph in a tree-like fashion (like Windows' Explorer or
the Finder on the Mac) may be more appropriate. 

In a more general vein, we may want to have different \emph{views} of
the same development graph which only shows particular object or link
types, displayed in a different fashion. (E.g. the tree-like display
just proposed would display the view of the graph given by restricting
links to the \emph{contains}-link.) Thus, a \emph{graph view} contains
a set of object types and link types visible in this view, plus a
display method.

From one view, the user can always change to another view by an
appropriate operation, or even use more than one view concurrently.

\subsection{Specific Instances}

\subsubsection{The Hugs workbench}

This is as sketched Einar's thesis. It would be nice to extend this to
a general Haskell development environment, which also handles the
GHC's many intermediate files.

The \emph{object types} are Haskell source files. They can be edited. 

As \emph{links}, we have import links denoting one module imports
another. This 

\emph{Remark:} Since there is only type of link between objects, which
moreover does not suitably organize the display of the
objects,\footnote{For example, the corresponding graph for HTk would
  not be very helpful.} the actual Hugs development graph might
more appropriately be viewed as the objects in the usual folder
hierarchie and no import links; and the view with all import links is
called the \emph{import} or \emph{dependency graph}.

\subsubsection{The Isabelle workbench}

Proving with Isabelle, or one if its many instances. This subsumes the
HOL-Z workbench, the HOL-CSP workbench and the transformation system
TAS in its present incarnation, but not CASL.

The \emph{object types} and the operations on them are:
\begin{description}
\item[Theories:] Isabelle's theories, filename ending in
  \texttt{.thy}\footnote{HOL-Z's filenames end in \texttt{.zthy}, but
    HOL-Z is no longer actively supported.},
  plus ML code for that theory, filename ending in \texttt{.ML}. 
  
  Theories can be edited.
  
  Theories are read into Isabelle from within Isabelle, i.e. there is
  an Isabelle command to load a particular theory.

\item[Proof scripts:] self-contained scripts of Isabelle commands which 
  prove one or more theorems. 
  
  Proof scripts can't be edited (since this might make them
  inconsistent). They are created from within Isabelle, i.e. Isabelle
  offers the user the possibility to export a proof as a proof-script.
  They can be read into other sessions, again from within Isabelle
  rather than the devleopment graph.

  Proof scripts can be added to theories. In that case, the script is
  always read when the theory is loaded into Isabelle.

\item[Session scripts:] scripts of Isabelle commands which lead from an
  initial Isabelle configuration to a particular Isabelle state.  
  
  Sessions are edited by starting Isabelle with them. When Isabelle
  terminates orderly, it writes a new session into the session file;
  if it aborts, it may leave the session file untouched.

  A session can be created from scratch. Initially, it does not
  contain anything at all. 
  
  The user will be able to load theories and proof scripts into the
  session, and create new proofs and theorems.  All of this will be
  initiated from within Isabelle.
\end{description}


Between objects, we have links as follows:
\begin{description}
\item[Extension links] from a theory to another theory denote
  conservative extensions --- the basic way in which Isabelle builds
  theories.
\item[Loads-theory links] from a session or proof script to a theory
  indicate that the  session or proof script loads the theory. 
\item[Lives-in] links from a proof script to a theory indicate the
  theory which the theorem(s) proved in the proof script inhabit. (All 
  Isabelle theorems live in a particular theory.) 
\item[Loads-script links] from a session or proof script to a theory
  indicate that the session or proof script reads the proof
  script. (So proof scripts can depend on other proof scripts.)
\end{description}

All of these links are dependency links (i.e. change propagates along
them.)

\subsubsection{The CASL workbench}

Development of CASL specifications, and theorem proving
of proof obligations and goals using different theorem
provers. A future extension should also cover
heterogeneous specifications. i.e. specifications 
consisting of parts written in different  logics
like higher-order CASL, HasCASL, CSP-CASL
(MULTIPLE project),
and Haskell program development (HasCASL project). 

The {\em object types} and the operations on them are:

\begin{description}
\item[CASL library file]
Text file, ending .casl. They can be edited with a text editor
and are read in with CATS.

\item[CASL library folder]
contains all files associated to a CASL library

\item[CASL specification text] 
Ending .spec.
Text file, derived from a CASL library file
(just by splitting it up into pieces). Can be
displayed, and edited only as part of the library
(i.e. any changes have to be propagated to the library file).

\item[CASL view text] 
Ending .view.
Text file, derived from a CASL library file. Can be
displayed, and edited only as part of the library
(i.e. any changes have to be propagated to the library file).

\item[CASL specification code]
Ending .spec.xml.
Derived from CASL specification text. Cannoted be edited.
Read in by CATS and the development graph manager.

\item[CASL view code]
Ending .view.xml.
Derived from CASL view text. Cannoted be edited.
Read in by CATS and the development graph manager.


\item[Development graph state]
The development graph manager constructs a development graph
(not to be confused with the workbench development graph,
since it contains as nodes the CASL specifications
and as links the CASL views, and some more internal
nodes and links, but not e.g. CASL libraries)
out of the specificattions and views contained in a CASL
library. Note that it does not suffice to supply just
the library to the development graph manager, since
it has to display textual information for individual
nodes of the development graph.

A development graph contains proof obligations
and proof goals that can be solved using different
theorem provers, e.g. Isabelle or Inka.

\item[Isabelle]
The first prover to be connected with the development
graph manager is Isabelle. Should therefore the
Isabelle instantiation of the workbench included here?
A difference is of course that Isabelle is called
from the (and with input supplied by the)
development graph manager.
\end{description}

MBase stores objects at a more fine-grained level,
e.g. theorems, proofs etc. Should we support this level
as well?

Between objects, we have the following links:

\begin{description}
\item[Import links]
between CASL libraries.

\item[Contains links] between a CASL library
and the specifications and views (the text files,
or code files, or both?)
contained in it.

\item[Uses links] between a CASL specification
and the CASL specifications and views used by it
(the text files, or code files, or both?).

\item[Source links] from a CASL view to its
source specification (the text files,
or code files, or both?).

\item[Target links] from a CASL view to its
target specification (the text files,
or code files, or both?).

\item[``is code of'' links] from a CASL
specification (or view) code file to the
corresponding text file.

\item[loads links] from a development
graph state to CASL library, specifications
and views used by it.

\item[calls links] between a development
graph state and the Isabelle session(s) called
by it.

\end{description}


\section{The Implementor's Point of View}

\emph{Presently, this section describes internals of the UniForM
  workbench as well, which probably should not be visible to a tool
  integrator.}

\subsection{Versions} 

The user is provided with a (single) Versions window.  This shows all the 
(global) versions, as a DAG.  The following operations are 
permitted with a version
\begin{enumerate}
\item  retrieve a version producing a Working Version.
    (A Working Version corresponds to an area in the users own file space;
    it should not be accessible by everyone else.  However access to the
    Working Version will be filtered through UniForM, so we can keep track
    of the changes, and don't have to download everything each time.
%    \begin{changebar}
      We store the changes the user makes to the version somewhere in
      the local file space, and write them back into the repository as 
      a whole. 
%    \end{changebar}
  \item (this will be done later) merge two versions to produce a
    Working Version.  (This will be done as follows.  Each type of
    object will have a merge operation.  For folders this is a
    straightforward recurse operation.  For text files, we produce a
    ``merged file'', something like \texttt{diff -Dfoo} or cvs, and
    put it into a text editor.
\end{enumerate}

%Once the user has a working version the files in the working version are
%displayed as a DAG.  (Or some of them anyway; we allow the user to
%expand/collapse nodes.)  The user can, for each file
%\begin{enumerate}
%\item edit it (we bung it into an editor);
%\item edit attributes (using HTk at first, but we should write this
%    so we can easily change to something else.) 
%\item rename it;
%\item delete it.
%\end{enumerate}
%Also of course the user can create new objects.

%The user can then check in a complete new working version.

\subsection{Objects}

Each object has a type. For non-folder objects, there is a class
\texttt{ObjType} which contains all operations available for objects
of that type:

\begin{enumerate}
\item is this a source file type (in the first stage it always will
  be. \emph{What would be examples for useful non-source files?})
\item the editing method: none, default, or a specific tool;
% \item for non-source files, an optional tool for editing source files.
\item required attributes for this type. 
  
  Note that attributes can be inherited.  For example (once we've
  added tools), a (target) library (.a) object will probably require
  as attributes (of the .a object or a containing folder) a list of
  the component objects.
\item a tool for creating it.
\item display information:
  \begin{itemize}
  \item the icon
  \item the name of the type
  \item a method for displaying an object of that type
  \item others (perhaps a printing method, or a method which renders
    the object in HTML so we can display the development graph in a
    net browser)
  \end{itemize}
\end{enumerate}

Folder objects have a folder type, which wraps up its contents into an
existential type, where all components are of class \texttt{ObjType}.

Adding types can be done 
\begin{itemize}
\item at runtime, by specifying simple things (label, tool as a
  command line, icon file);
\item at compile/link time. Adding types at compile time creates a new
  instance of the workbench. Different instances of the workbench are
  incompatible in the sense that e.g. the Hugs workbench can't
  exchanges objects with (or even talk to) the Isabelle workbench.
\end{itemize}
%Hopefully the latter will rarely, if ever, be necessary, especially as
%it introduces the possibility of communicating UniForM's disagreeing
%about what types are around..

%Editors: there should be a single XEmacs session
%around, and "bunging something into the editor" will mean popping up a
%new frame containing it. 


% \begin{changebar}
% \subsection{Objects}

\subsubsection{Attributes}

Links are implemented as particular attributes of type
\texttt{Location}, which uniquely identifies an object regardless of
its particular location (i.e. file path). I think since links are user
often there should be a library of functions which implement functions
on links, such as:
\begin{itemize}
\item get the source object 
\item get the target object
\item for an object, get all incoming links
\item for an object, get all outgoing links
\end{itemize}

Also, links can in turn have attributes, such as a name, or the colour
in which we want it displayed. These attributes could be implemented
by having a \texttt{ObjLink} class which specifies these (so these are
not file attributes in the technical sense above).

Implementing the two operations which find all incoming or outgoing
links may be complicated; we can probably get away for now with only
implementing the second. (We need this for the change notifications
below.)

\subsubsection{Change notifications}

\emph{Needs to be rewritten.}

Change notifiers (of type \texttt{IA a}) allow tools to listen for
changes of objects and links, as follows:
\begin{itemize}
\item a new version of an object has been checked in;
\item change propagated along dependency links, i.e. if a new version
  is checked in for an object $o$, for all objects $p$ such that there
  is a dependency link $l:o\rightarrow p$ a change notifier is
  generated;
\item creation or deletion of a link;
\item creation or deletion of an object. 
\end{itemize}

The standard UniForM workbench interface will deal with most of these
notifiers (by e.g. highlighting objects which are currently in the
user's private workspace, and for which a change notification has been
received), but if we encapsulate a tool, we may want to access
notification interactions directly. (For example, Isabelle wants to
know when its theory files changes.)

% \end{changebar}

\subsection{Second Stage: Tools}

There are two sort of tools: oneshot and systematic.  

\subsubsection{Oneshot Tools}

A oneshot tool has to be specified explicitly by the user as a means of 
creating a new object.  An example would be trying to apply a tactic.
UniForM may need to remember the tool applied.  Specifically for oneshot
tools mapping to non-source objects, we remember the tool.  For oneshot
tools mapping to source objects, we remember the tool, unless the user
goes on to edit the source object, in which case we remember the source
object.  ?? oneshot tools mapping to multiple objects ??  We need to make
sure that on retrieving a version, the oneshot tools are executed in the
correct order.

% \begin{changebar}
%  Can't we describe oneshot tools as a specific \emph{command line
%    tool}, which essentially is the encapsulation of the shell (or
%  other scripting tools like Tcl, Perl, Python\dots)? 
  
%  Applying a single tactic is probably not application of a tool ---
%  the Isabelle encapsulation should abstract from Isabelle's datatypes
%  like tactics, rewriting sets and so on.
% \end{changebar}

\subsubsection{Systematic Tools}

Systematic tools map to objects to objects on the basis of rules based on
the types and names of the objects.  Here are two sorts:

\begin{verbatim}
(name,type1) -> (name,type2)
\end{verbatim}

This kind of systematic tool maps one type to another type (always
a non-source type).  This would
correspond to compiling a source file to an object file, for example.

\begin{verbatim}
(name,folder type) -> (something in this folder,type2)
\end{verbatim}

This would correspond to compiling, for example, a bunch of source objects
(in the folder) to a library in that folder.

In fact we can (and will) generalise this, allowing multiple target objects.
\begin{verbatim}
(name,type) -> (name,type1),(name,type2),...
\end{verbatim}

In addition, systematic tools should come with an additional action
(probably yet another tool) which computes dependencies for
\texttt{(name,type1)}.  These dependencies will be ensured up-to-date
before the object is computed.  The dependencies will be assumed a
pure function of the file contents+attributes, so will be cached (as
one of the file attributes) and hopefully not recomputed too often,
though we will doubtless need a way of recomputing them when things go
wrong.

I HOPE that most tools (oneshot and systematic) can also be defined at 
run-time.  Thus there will also be a (global) tools graph, and 
UniForM will complain when the rules given above lead to multiple ways of
getting the same file (or at least expect some disambiguation).

% \begin{changebar}
  Tools also create links. This is done just as with objects --- thus,
  we add constructors \texttt{CreateLink}, \texttt{DeleteLink} or
  \texttt{ChangeLink} (new source or new target) to the
  \texttt{Change} datatype. 

  But it can be tedious to explicitly set or delete links everytime we
  create a new version. (For Isabelle, it is fairly easy to write
  something which reads out the set of all ancestor theories for a 
  theory, but it would be tedious to keep track of new ancestors and
  even more cumbersome to find out which ancestors have been removed.)
 
  Thus, some links are generated automatically by a tool which is
  called when a new version of an object is created (i.e. after
  editing an object). This can be achieved by a change
  notification interactor, which listens for changes and generates new 
  links if the version of an object changes.
% \end{changebar}

\end{document}


%%% Local Variables: 
%%% mode: latex
%%% TeX-master: t
%%% End: 
