\documentclass[12pt,a4paper]{article}
\usepackage{t1enc}
\usepackage[latin1]{inputenc}
\usepackage[german]{babel}
\usepackage{epsfig}
\usepackage{wrapfig}
\usepackage{graphicx}
\markboth{}{}

\setlength{\oddsidemargin}{-0.7cm}
\setlength{\evensidemargin}{-0.7cm}
\setlength{\textwidth}{17cm}
\setlength{\topmargin}{-1cm}
\setlength{\textheight}{23.5cm}

\title{Verkapselung von Tk-Widgets f�r HTk}
\author{ludi@tzi.de}
\date {}

\begin{document}

\maketitle


\section*{Grundlegendes}

...

Zentral bei der Verkapselung eines Tk-Widgets in HTk ist der (abstrakt)
exportierte Datentyp. Dieser dient nach au�en hin als Repr�sentation
des jeweiligen Widgets. Darauf k�nnen z.B. sp�ter verschiedene
Konfigurationen angewendet werden.

\begin{verbatim}
newtype Label a = Label GUIOBJECT deriving Eq
\end{verbatim}

Dieser Typ kann weitere Widget-spezifische Felder haben, dies
ist eine Frage der individuellen Implementierung. Er enth�lt jedoch immer
ein Feld des Typs \verb|GUIOBJECT|


\section*{Der Typ GUIOBJECT und die Klasse GUIObject}

Dieser Type \verb|GUIOBJECT| ist die interne Repr�sentation der
verschiedenen Oberfl�chen-Objekte (im wesentlichen Widgets,
Canvas-Objekte, Men�elemente, aber auch spezielleres wie z.B. Text-Tags
oder Canvas-Tags).
Diese interne Repr�sentation durch \emph{einen} Typen erm�glicht z.B.
Listen verschiedster Objekte im internen Status zu halten oder
als Funktions-Argumente zu �bergeben.

Der \verb|GUIOBJECT|-Wert mu� jedoch i.d.R. nicht selbst konstruiert
werden, hierzu gibt es spezielle Funktionen (z.B. \verb|createGUIObject|
im Modul \verb|Core|, \verb|createCanvasItem| im Modul
\verb|CanvasItemAux|). Aussehen tut er wie folgt:

\begin{verbatim}
data GUIOBJECT = GUIOBJECT ObjectID (Ref OST) | ROOT
\end{verbatim}

Den \verb|ROOT|-Konstruktor m�chte ich hier nicht weiter erl�utern. Er hat
eine spezielle Bedeutung und steht f�r die Wish-Instanz bzw. das
Hauptfenster, dies mu� man aber jetzt nicht verstehen. F�r "normale"
GUI-Objekte wird ein Wert entsprechend des ersten Konstruktors erzeugt.
Dieser enth�lt die \verb|ObjectID|, die dieses spezielle Widget eindeutig 
identifiziert und eine Referenz auf den zu diesem Objekt geh�rende Status.

(GUIObjectKind, GUIObjectName...)

Der (exportierte, s.o.) Widget-Typ mu� immer eine Instanz der Klasse
\verb|GUIObject| sein:

\begin{verbatim}
instance GUIObject (Label a) where 
  toGUIObject (Label w) = w
  cname _ = "Label"
\end{verbatim}

So kann durch die Klassenmethode \verb|toGUIObject| der externe Typ
in die interne Repr�sentation gewandelt werden.

...



\section*{Widget-Methoden}

...


\section*{Erzeugung}

...


\end{document}