\documentclass[a4paper,12pt]{article}

\usepackage{isolatin1}
\usepackage{verbatim}
\usepackage{palatino}
\usepackage{color}
\usepackage{xspace}
\usepackage{hyperref}


%%% -- Configuration ----------------------------------------------------

\hyperbaseurl{file:///home/cxl/src/uni/htk/doc/hdoc/}

%
% Colour setup (to come?)
%

\definecolor{linkcol}{rgb}{0.1,0.1,0.4} % dark blue

\ifx\hypersetup\undefined\relax\else
 \hypersetup{%
    %breaklinks=true,
    colorlinks=true,
    %hyperindex=true,
    pdfpagemode=None,
    linkcolor=linkcol,
    citecolor=linkcol,
    %plainpages=false,
    hypertexnames=false
 }
\fi


%%% -- Useful macros ----------------------------------------------------

%% The code environment typesets its contents verbatim.
\def\code{\verbatim}
\def\endcode{\endverbatim}
%% Same typesetting as code, but different name; this is
%% for code you do not want to show up in literal scripts. 
%% (i.e. the code with syntax errors in it :-)
\def\xcode{\verbatim}
\def\endxcode{\endverbatim}
%% Code snippets in the text:
% \def\codetxt{\textcolor{codecol}\verb} %% hmm...
% \let\MMTextTT=\texttt{}
% \renewcommand{\texttt}[1]{\textcolor{codecol}{\MMTextTT{#1}}}


\title{An Introduction to \HTk \\
  Graphical User Interfaces for Haskell Programs}

\author{Christoph L�th \\ FB 3 -- Mathematik und Informatik,
  Universit�t Bremen}

\newcommand{\HTk}{\textsc{HTk}\xspace}


\begin{document}

\maketitle{}

\section{Getting Started}

This article is an introduction to the basics of \HTk, a toolkit to
build graphical user interfaces (GUIs) in Haskell. \HTk{} is based on
an encapsulation of Tcl/Tk \cite{Ousterhout,Welsh}, but we will not
assume any previous knowledge of Tcl/Tk. The article is meant as a
rough guide and introduction to the structure of HTk; it is not meant
as a complete reference manual. Rather, it should give readers enough
information and background to get them started on their first HTk
programs, to know which parts of HTk might be potentially useful in
the applications they have in mind, what is feasible to build with HTk
and what not, and finally to enable them to find further information
quickly in the \href{index.html}{online reference material}.

\subsection{Basics}

When we design and implement a graphical user interface, we have to
take two aspects into account: the \emph{static} aspect, 
which is to specify its appearance (which buttons to place where, what
menues to display, etc.), and the \emph{dynamic} aspect, which
specifies its behaviour in reaction to the user's actions. 

In \HTk, these two aspects are modelled by \emph{monads}. The dynamic
aspect is modelled in the \texttt{IO} monad, where all of Haskell's
external interactions takes place. The dynamic aspect is modelled by
\emph{events}. For a more complete description of events, we refer to
\cite{ger:Events}. For the time being, events are an abstract datatype
with two main operations.

The central operation is \texttt{sync :: Event a-> IO a} which holds
the current thread until an event of type \texttt{Event a} occurs.
Further, \texttt{(>>>=) :: Event a-> (a-> IO b)-> Event b} takes an
event and an IO action, and returns an event, which when we sync on it
performs the IO action after successful synchronisation. As with the
monad composition, \texttt{(>>>)::Event a-> IO b-> Event b} is the
derived version where the second function throws away its argument.

Moreover, events form a monad, which allows us to build complex
behaviour from basic behaviours in a compositional way by the monad's
composition.

Events are always embedded in the \texttt{IO} monad with the
\texttt{sync} operation. That the dynamic behaviour is not modelled
with \texttt{IO} actions directly reflects the fact that user
interaction in a graphical user interface is different from other
forms of IO, because it happens \emph{asynchronously}.

Further, events allow the user interface to be concurrent in a natural
and controlled way, which allows for a reasonable degree of
concurrency which is still tractable.

\subsection{A First Example}

To make this more concrete, consider a very simple example. We want to
open a window which contains just one button, which should be labelled
\textit{Press me!}. Whenever the user obligingly presses the button,
it should change its label to a different random string.

The static part of this program is fairly simple. There will be an
initialization function (which opens the window and such), and we want
to build a button with the inscription \texttt{Press me}. The
following code achieves this:

\begin{comment}
\begin{code}
module Main where

import HTk    
import Random
\end{code}
\end{comment}

\begin{code}  
main:: IO ()
main =
  do main <- initHTk []

     b <- newButton main [text "Press me!"] 
     pack b []
\end{code}
This introduces three important concepts in \HTk:
\begin{itemize}
\item firstly, the elements of the graphical user interface are
  organized hierarchically. When we create a new button, we have to
  pass it the GUI element which it is part of (here, the main window).
\item secondly, GUI elements are created with functions called
  \texttt{new}X, which take a \emph{configuration} list as
  argument. The configuration determines the visual appearance; here,
  the text which is displayed on the button
\item thirdly, creating a GUI element does not display it \textit{per
    se}. To display it, we have to explicitly place it on the screen;
  this is done with the \texttt{pack} command. This command also takes
  a list of configurations as arguments; more on that below.

\textbf{There's also the business with Button String. For the rest of
  this note, I'll assume we are going to drop this (i.e. just have Button).}
\end{itemize}

To specify the dynamic behaviour, we need two ingredients: firstly, we
need to connect the external event of the user clicking the button
with an element of the data type \texttt{Event}, and secondly, we need
to set up the program such that it reacts to the occurence of this
event by changing the button's label. 

Setting up external events to produce an \texttt{Event a} is called
\emph{binding}. When we bind an external event, we specify the
external action that we wish to bind (e.g. this button being clicked,
mouse movement over this window, right button being clicked with
control-key being pressed and user doing a handstand whilst whistling
"'Auld Lang Syne"`). The general case is the \texttt{bind} function
which we will see below, but for the simple case of a button being
clicked, we can use the function \texttt{clicked :: Button a-> IO
  (Event ())}. 

The composed event we want to synchronise on is the click of the
button, followed by changing the label. The following code achieves
the desired effect:
\begin{code}
     click <- clicked b
     spawnEvent 
      (forever 
        (click >>> do nu_label <- mapM randomRIO (replicate 5 ('a','z'))
                      b # text nu_label))
     finishHTk
\end{code}     
Here, \texttt{randomRIO (replicate 5 ('a','z'))} generates a list of
five actions of type \texttt{IO Char}, and \texttt{mapM} evaluates
them to a random string of length 5. The next line sets the label to
this random string; how exactly this works will be explained below.

Two more functions require an explanation here: \texttt{forever ::
  Event a-> Event a} takes an event, and returns this event composed
with itself. Thus, synchronising on this event will synchronise on it
once, then wait for this event occuring again. The effect here is that
the effect we want to achieve occurs recurrently. Had we left out the
\texttt{forever}, our program would just wait for one button press,
change the colour of the button once and go on its merry way (in this
case, terminate). With \texttt{forever}, we have it wait for the next
button press after the first one occurs. 

Finally, \texttt{spawnEvent} takes an event, and creates a concurrent
thread which synchronises on this event. This is not strictly
necessary here, since we do nothing else, but it is good practice to
leave handling of events to threads different from the main thread.
Exactly how many threads one creates --- one for each button, or just
one for the whole GUI --- is a matter of taste and judgement.

At the end of the program, the main thread has to wait for the GUI to
finish; if it just exited, the whole program would terminate. We do
this by calling \texttt{finishHTk}. This also handles the case that
the user closes the window by external means (e.g. the close button
provided by the window manager).

Note that our program is non-terminating. If the window manager does
not provide means to close a running application, we will have to use
\texttt{kill} or \texttt{xkill} to stop it. This is clearly
unsatisfactory, so we will now provide a second button to close the
window regularly. 

%%% Local Variables: 
%%% mode: latex
%%% TeX-master: "intro"
%%% End: 


\subsection{A Second Example}

\begin{comment}
\begin{code}
module Main where

import HTk    
import Random
\end{code}
\end{comment}

We have to augment our previous program in two aspects: statically, we
have to provide another button, and dynamically, we have to react to
this button being pressed by ending the program

For the first part, we create the second button just like the first
part. When we place it, we have to specify where it is going to be
placed.  We want it below the second button, and we want both buttons
to stretch out horizontally such that they are of the same length,
regardless of the size of the labels.  This is done by adding
\emph{packing options} to the \texttt{pack} command. Here,
\texttt{Side} says we want the first button at the top and the second
at the buttom, and \texttt{Fill X} specifies the stretching
behaviour mentioned above:

\begin{code}
main:: IO ()
main =
  do main <- initHTk []

     b <- newButton main [text "Press me!"]
     b2 <- newButton main [text "Close"]
     pack b [Side AtTop, Fill X]
     pack b2 [Side AtBottom, Fill X]
\end{code}

To change the dynamic behaviour, we first need the second button to
create an event with the \texttt{clicked} function. However, we need
to change the behaviour of the spawned event such that when this new
clicked event occurs, the program is finished. 

This combination of events as a case distinction --- "`when this event
occurs, do something, when the other event occurs, do something
different"' --- is achieved by the third important operation on
events, the \emph{choice} combinator \texttt{(+>) :: Event a-> Event
  a-> Event a}. Hence, we need to combine the previous dynamic
behaviour and the new behaviour by \texttt{+>}. The new behaviour,
finishing the program, is achieved by calling the \texttt{destroy}
action on \texttt{main}.  This closes the main window and lets the
program terminate gracefully:
\begin{code}
     click  <- clicked b
     click2 <- clicked b2
     spawnEvent 
      (forever 
        ((click >>> do nu_label <- mapM randomRIO (replicate 5 ('a','z'))
                       b # text nu_label
                       done)
        +> (click2 >>> destroy main)))
     finishHTk
\end{code}
Note that the choice occurs inside the \texttt{forever} (why?). We
could also have created two threads here, each listening to one
button. While in this simple situation, this would have been easier,
it is in general good practice to create only as many threads as
needed, since one otherwise tends to run into memory leaks by unused
threads lying around or even worse, nasty synchronisation problems.


%%% Local Variables: 
%%% mode: latex
%%% TeX-master: "intro"
%%% End: 


\subsection{Structure of this Paper}

The rest of this short paper is organized as follows: we will first
explain the organization of the datatypes modelling the static
behaviour of the graphical user interface. In
section~\ref{sec:events}, we will describe events and in particular
how to generate them from user input. After this, we will describe
every widget in detail.


\section{Elements of \HTk}

In general, \HTk has a couple of abstract datatypes used to model
elements of the graphical user interface, such as buttons, menues,
short text fields, longer text fields and so on. Let us examine the
buttons used in Section~\ref{ssec:ex1} above. There is an abstract
datatype \texttt{Button}, created with the following function
\begin{xcode}
newButton :: Container par=> par-> [Config Button]-> IO Button
\end{xcode}

Let us first examine the class \texttt{Container}. 

\subsection{The GUI element hierarchy and the \texttt{Container} class.}

The class \href{Packer.html#Container}{\texttt{Container}} designates
GUI elements into which other GUI elements may be packed.

Instances of \texttt{Container} include \texttt{Toplevel} (windows),
\texttt{HTk} (Tk's root window), and \texttt{Frame}; furthermore
\texttt{Canvas}, and and \texttt{Editor} (and a few Tix widgets).

The class \texttt{Container} is \emph{abstract} --- it has no class
functions, and only serves to structure the code. Abstract classes are
used frequently in HTk to impose a typing discipline onto Tk's untyped
GUI element structure, with the benefit that type checking can prevent
run time errors.


\subsection{Configurations and Resources}

Above, the text of the button was set with a \emph{configuration
  option}. Configuration options determine various attributes of a
widget. They can be given at the time of creation, or changed later
on. Not every widget supports all configurations, and this behaviour
is modelled in HTk by Haskell's type classes: configurations in
general are polymorphic over all widgets, but particular
configurations are restricted to certain classes of widgets.

For example, the text configuration is given by this class:
\begin{xcode}
class (GUIObject w, GUIValue v) => HasText w v where
  text :: HasText w v => v -> Config w
  getText :: HasText w v => w -> IO v
\end{xcode}
The class \texttt{GUIObject w} is one of HTk's most basic classes. Its
instances are widgets, and other interface elements we will encounter
later (canvas items, text tags). \texttt{GUIValue v} is another basic
class, the instances of which are all basic datatypes which can be
communicated to Tk: \texttt{Int}, \texttt{Double}, \texttt{Bool},
\texttt{String} and \texttt{[String]}. 

Widgets can be configured with a text are instances of the class
\texttt{HasText}, such as \texttt{Button}. 

The configuration classes can all be found in the module
\href{Configuration.html}{\texttt{Configuration}}.

The configuration type is just a type synonym\footnote{Type synonyms
  like that in class confusions confuse Hdoc, which is why they appear
  expanded at various places of HTk's source code--- just in case you
  happen to browse it, which you are more than welcome to.}
\begin{xcode}
type Config w = w -> IO w
\end{xcode}
As seen above, configurations can be given at the time of creation, or
later on. In the latter case, the helpful \texttt{(\#)} operator
provides useful syntactic sugar:
\begin{xcode}
( # ) :: a -> (a -> b) -> b
o # f = f o  
\end{xcode}

Note the difference between configuration options, which determine the
appearance, and behaviour of the widget, and packing options, which
determine the way it is packed. 

\subsubsection{Geometry}

The abstract data type \texttt{Distance}, implemented in the module
\href{Geometry.html}{\texttt{Geometry}}, represents distances in HTk.
Distances can be specified in \texttt{cm}, \texttt{mm}, \texttt{ic}
(inches) and \texttt{pp} (points), with functions  \texttt{cm:: Int->
  Distance}. Moreover, \texttt{Distance} is an instance of
\texttt{Num}, so we can specify the distance 3 (meaning 3 pixels)
directly. 

\subsubsection{Colours}

The abstract data type \texttt{Colour}, implemented in the module
\href{Geometry.html}{\texttt{Geometry}}, represents colours in
HTk. Just like distances, the type itself is abstract, but unlike
distances, there is a class \texttt{ColourDesignator}, the instances
of which give ways of describing colours, such as:
\begin{xcode}
instance ColourDesignator [Char]
instance ColourDesignator (Int, Int, Int)
instance ColourDesignator (Double, Double, Double)  
\end{xcode}
The strings are named colours (\texttt{red}, \texttt{white},
\texttt{black}, etc.), the tuples are RGB values. (The functions of
the type classes \texttt{Colour} and \texttt{ColourDesignator} are for
HTk's internal consumption only.)

\subsubsection{Fonts}

Fonts are implemented in the module
\href{Font.html}{\texttt{Font}}. They are specified in the usual way,
by giving a family, slant, spacing, width and weight. For example, the
family is given by 
\begin{xcode}
data FontFamily = Lucida | Times | Helvetica 
                | Courier | Symbol | Other String  
\end{xcode}
where the five enumerated types are available on most systems. With
\texttt{Other}, you can directly give a more exotic family such as
\texttt{clearlyu alternate glyphs}. 

Be warned that fonts are, in principle, not very portable under X,
since the available fonts are determined by the fonts of the X server
the programm is running on. It is best to stick to well-known font
families such as the above, and usual sizes. 

\subsection{Packing}

As mentioned above, after widgets have been created (with e.g.
\texttt{newButton}), they will not be displayed yet; this only happens
after they have been packed. One can use this effect by first creating
lots of widgets, and then packing them in one go, lessening the
unpleasant flicker effect occuring when the GUI is built one interface
at a time.\footnote{Unfortunately, this effect cannot be totally
  eliminated.}

Packing in particular determines the visual layout of the GUI by the
order in which the widgets are packed, and by packing options. Tk's
know different packing algorithms (or \emph{geometry managers}, in Tk
parlance); of these, HTk supports the standard packer, and the grid
packer.


\subsubsection{The Standard Packer}

The behaviour of the standard packer is easily explained, and hard to
understand. Widgets are packed with the function
\begin{xcode}
pack::Widget w => w -> [PackOption] -> IO ()  
\end{xcode}

The datatype
\href{PackOptions.html#PackOptions.PackOption}{PackOption}
is defined as 
\begin{xcode}
  data PackOption = Side SideSpec  | Fill FillSpec 
                  | Expand Toggle  | Anchor Anchor
                  | IPadX Distance | IPadY Distance
                  | PadX Distance  | PadY Distance
\end{xcode}
The first two constructors are most important here. The
\href{PackOptions.html#PackOptions.SideSpec}{SideSpec}
specifies where the widget is packed (top, bottom, left, right)e, and
\href{PackOptions.html#PackOptions.FillSpec}{FillSpec}
specifies in which direction it expands to fill the available space.
Bear in mind that widgets are packed as tight as possible (in
particular into windows), and that once packed, they are never
repacked. That is, if e.g. a widget is packed against the top, it will
sit in the middle (if no \texttt{Fill X} is specified), and will not
move if a widget is packed against the right-hand side.

\texttt{Expand} just means that the widget expands when the containing
element is expanded (i.e. the window is resized), and \texttt{Anchor}
specifies a gravity (a side to which the widgets stick). The rest
create a padding border around the widget in various directions.

It is quite normal that most of the times the packing will not look
like intended, and you will need to use frame widgets (see
\ref{ssec:frames}).

\subsubsection{The Grid Packer}

The grid packer divides the container widget into a grid, and allows
placement of widgets relative to that grid. To pack a single widget
use the 
\href{Packer.html#Packer.grid}
{following function:}
\begin{xcode}
grid :: Widget w => w -> [GridPackOption] -> IO ()
\end{xcode}

The datatype
\href{GridPackOptions.html#GridPackOptions.GridPackOption}
{\texttt{GridPackOptions}} specifies the packing options for the grid
packer.

Note that within the same container you cannot use different packing
algorithms. The first widget packed into a container defines the
packing for this container.

\section{Events}

In general, events are an abstract datatype for communication and
synchronisation, much in the spirit of process algebras such as CCS
\cite{Milner:CommunicationConcurrency}, CSP \cite{Hoare,Roscoe} or
the $\pi$ calculus \cite{Milner:PiCalculus}. Here, an 
\href{Events.html}{\texttt{Event}} is
an abstract datatype with operations such as \texttt{sync},
\texttt{+>} and \texttt{>>>=}, which additionally form a monad; we
refer to \cite{ger:Events} for more information. 

In \HTk, events are mostly generated from user interactions by means
of the \emph{bind} commands. By binding a user interaction (such as
clicking a button), we set it up to produce an event, on which we can
synchronise and thus produce a reaction to the user's action.

One important caveat here is that once you set up a binding, you
\emph{must} eventually synchronise on the resulting events, since
otherwise the unused events will pile up and result in a memory leak.

Bindings are generated by calling one of \texttt{clicked},
\texttt{bindSimple} and \texttt{bind}, where \texttt{bind} is most
flexible, which shows up in the type.

\subsection{Simple Clicks}
Simple interface elements such as buttons and menu entries which
  are instances of the class
  \href{HTk.html#HTk.HasCommand}{\texttt{HasCommand}}. For these, we
  have a function
\begin{xcode}    
clicked :: HasCommand w => w -> IO (Event ())
\end{xcode} 
  The event here occurs when the element is clicked. The event does
  not have any additional information. 

\subsection{Simple Binds}
For all GUI elements, the function
  \href{HTk.html#HTk.bindSimple}{\texttt{bindSimple}} 
\begin{xcode}    
bindSimple::GUIObject wid => wid -> WishEventType -> IO (Event (), IO ())
\end{xcode}
where \href{HTk.html#HTk.WishEventType}{\texttt{WishEventType}} is an
algebraic data type describing the kind of event we would like to bind
to, much along the lines of Tk's events which in turn are given by X
events. Each constructor corresponds to a different event, such as
\begin{itemize}
\item mouse buttons pressed and released (\texttt{ButtonPress (Maybe
    Int)}, \texttt{ButtonRelease (Maybe Int)}),
\item mouse movements (\texttt{Motion}),
\item the mouse entering or leaving a GUI element (\texttt{Enter},
  \texttt{Leave}),
\item keys pressed or release (\texttt{KeyPress (Maybe KeySym)},
  \texttt{KeyRelease (Maybe KeySym)}),
\item and various window events (\texttt{Map}, \texttt{Unmap},
  \texttt{Expose}, \ldots).
\end{itemize}
Note that not every GUI element can generate every event. Obviously,
only windows can generate window events, but more subtly, only entry
widgets, editor widgets and windows can generate \texttt{Key} events.

The return value of \texttt{bind} is an event, and an IO action which
unbinds the event. You should use this action if you are not
interested in the event anymore (i.e. it will not be synchronised on
anymore), otherwise there will be a memory leak.

\subsection{Full Bindings}

In fact, \texttt{bind} is only a simplified version of the
  function \href{HTk.html#HTk.bind}{\texttt{bind}}
\begin{xcode}
bind::GUIObject wid => wid -> [WishEvent] 
                       -> IO (Event EventInfo, IO ())
data WishEvent = WishEvent [WishEventModifier] 
                           WishEventType
\end{xcode}
where
\href{HTk.html#HTk.WishEventModifier}{\texttt{WishEventModifier}}
desribes possible event modifiers such as \texttt{Shift},
\texttt{Alt}, \texttt{Meta} (corresponding to the Shift, Alt or Meta
key being held), \texttt{Button1}, \ldots, \texttt{Button5}
(corresponding to mouse button 1 thru 5 being pressed) or \texttt{Double} and
\texttt{Triple}. Again, not all combinations of modifiers and events
make sense; for example, double and triple pertain to mouse button
presses, and modifying a mouse button press with a different button is
not really helpful. Note that the argument of bind is a list of
events, meaning of course that each of the events generates in event
(not only their combination).

Here, the first component of the return value is an event of
\texttt{EventInfo}, which is a labelled record as
follows:\footnote{Unfortunately, HDoc chokes on the relevant part of
  the source right now, so this is not in the reference manual.}
\begin{xcode}
data EventInfo = EventInfo { x :: Distance,
                             y :: Distance,
                             xRoot :: Distance,
                             yRoot :: Distance,
                             button :: Int  
\end{xcode}
The information here is the x and y component of the mouse position,
both relative to the window in which the event occurs, and the root
window (i.e. the screen), and the button being pressed. 


\section{Widgets}

\subsection{Basic Widgets}

\begin{itemize}
\item Frames
\label{ssec:frames}
\item Buttons
\item Labels
\item Message Boxes
\item Scrollbars
\end{itemize}

\subsection{Menues}

\subsection{The Textwidget}

\subsection{The Canvas Widget}

\section{The Toolkit}

\section{Larger Examples}

The Minesweeper game



\bibliography{intro}
\bibliographystyle{plain}


\end{document}

%%% Local Variables: 
%%% mode: latex
%%% TeX-master: t
%%% End: 
