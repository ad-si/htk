\documentclass[a4paper,12pt]{article}

\usepackage{isolatin1}
\usepackage{verbatim}
\usepackage{palatino}
\usepackage{color}
\usepackage{hyperref}

%%% -- Configuration ----------------------------------------------------

\hyperbaseurl{file:///home/cxl/src/uni/htk/doc/}

%
% Colour setup (to come?)
%

\definecolor{linkcol}{rgb}{0.1,0.1,0.4} % dark blue

\ifx\hypersetup\undefined\relax\else
 \hypersetup{%
    %breaklinks=true,
    colorlinks=true,
    %hyperindex=true,
    pdfpagemode=None,
    linkcolor=linkcol,
    citecolor=linkcol,
    %plainpages=false,
    hypertexnames=false
 }
\fi


%%% -- Useful macros ----------------------------------------------------

%% The code environment typesets its contents verbatim.
\def\code{\verbatim}
\def\endcode{\endverbatim}
%% Same typesetting as code, but different name; this is
%% for code you do not want to show up in literal scripts. 
%% (i.e. the code with syntax errors in it :-)
\def\xcode{\verbatim}
\def\endxcode{\endverbatim}
%% Code snippets in the text:
% \def\codetxt{\textcolor{codecol}\verb} %% hmm...
% \let\MMTextTT=\texttt{}
% \renewcommand{\texttt}[1]{\textcolor{codecol}{\MMTextTT{#1}}}


\title{An Introduction to \HTk{} \\
  Graphical User Interfaces for Functional Programs}

\author{Christoph L�th \\ FB 3 -- Mathematik und Informatik,
  Universit�t Bremen}


\newcommand{\HTk}{\textsc{HTk}}


\begin{document}

\maketitle{}

\section{Getting Started}

This article is an introduction to the basics of \HTk, a toolkit to
build graphical user interfaces (GUIs) in Haskell. \HTk{} is based on
an encapsulation of Tcl/Tk \cite{Ousterhout,Welsh}, but we will not
assume any previous knowledge of Tcl/Tk. The article is meant as a
rough guide and introduction to the structure of HTk; it is not meant
as a complete reference manual. Rather, it should give readers enough
information and background to get them started on their first HTk
programs, to know which parts of HTk might be potentially useful in
the applications they have in mind, what is feasible to build with HTk
and what not, and finally to enable them to find further information
quickly in the online reference material.

\subsection{Basics}

When we design and implement a graphical user interface, we have to
take two aspects into account: the \emph{static} aspect, 
which is to specify its appearance (which buttons to place where, what
menues to display, etc.), and the \emph{dynamic} aspect, which
specifies its behaviour in reaction to the user's actions. 

In \HTk, these two aspects are modelled by \emph{monads}. The dynamic
aspect is modelled in the \texttt{IO} monad, where all of Haskell's
external interactions takes place. The dynamic aspect is modelled by
\emph{events}. For a more complete description of events, we refer to
\cite{ger:Events}. For the time being, events are an abstract datatype
with two main operations.

The central operation is \texttt{sync :: Event a-> IO a} which holds
the current thread until an event of type \texttt{Event a} occurs.
Further, \texttt{(>>>=) :: Event a-> (a-> IO b)-> Event b} takes an
event and an IO action, and returns an event, which when we sync on it
performs the IO action after successful synchronisation. As with the
monad composition, \texttt{(>>>)::Event a-> IO b-> Event b} is the
derived version where the second function throws away its argument.

Moreover, events form a monad, which allows us to build complex
behaviour from basic behaviours in a compositional way by the monad's
composition.

Events are always embedded in the \texttt{IO} monad with the
\texttt{sync} operation. That the dynamic behaviour is not modelled
with \texttt{IO} actions directly reflects the fact that user
interaction in a graphical user interface is different from other
forms of IO, because it happens \emph{asynchronously}.

Further, events allow the user interface to be concurrent in a natural
and controlled way, which allows for a reasonable degree of
concurrency which is still tractable.

\input{simple1}

\input{simple2}

The rest of this short paper is organized as follows: we will first
explain the organization of the datatypes modelling the static
behaviour of the graphical user interface. In
section~\ref{sec:events}, we will describe events and in particular
how to generate them from user input. After this, we will describe
every widget in detail.

\section{Elements of \HTk{}}

In general, \HTk{} has a couple of abstract datatypes used to model
elements of the graphical user interface, such as buttons, menues,
short text fields, longer text fields and so on. Let us examine the
buttons used in Section~\ref{ssec:ex1} above. There is an abstract
datatype \texttt{Button}, created with the following function
\begin{xcode}
newButton :: Container par=> par-> [Config Button]-> IO Button
\end{xcode}

Let us first examine the class \texttt{Container}. 

\subsection{The GUI element hierarchy and the \texttt{Container} class.}

The class \href{Packer.html#Container}{\texttt{Container}} designates
GUI elements into which other GUI elements may be packed.

Instances of \texttt{Container} include \texttt{Toplevel} (windows),
\texttt{HTk} (Tk's root window), and \texttt{Frame}; furthermore
\texttt{Canvas}, and and \texttt{Editor} (and a few Tix widgets).

The class \texttt{Container} is \emph{abstract} --- it has no class
functions, and only serves to structure the code. Abstract classes are
used frequently in HTk to impose a typing discipline onto Tk's untyped
GUI element structure, with the benefit that type checking can prevent
run time errors.

\subsection{Packing}

As mentioned above, after widgets have been created (with e.g.
\texttt{newButton}), they will not be displayed yet; this only happens
after they have been packed. One can use this effect by first creating
lots of widgets, and then packing them in one go, lessening the
unpleasant flicker effect occuring when the GUI is built one interface
at a time.\footnote{Unfortunately, this effect cannot be totally
  eliminated.}

Packing in particular determines the visual layout of the GUI by the
order in which the widgets are packed, and by packing options. Tk's
know different packing algorithms (or \emph{geometry managers}, in Tk
parlance); of these, HTk supports the standard packer, and the grid
packer.

\subsubsection{The Standard Packer}

The behaviour of the standard packer is easily explained, and hard to
understand. Widgets are packed with the function
\begin{xcode}
pack::Widget w => w -> [PackOption] -> IO ()  
\end{xcode}

The datatype
\href{PackOptions.html#PackOptions.PackOption}{PackOption}
is defined as 
\begin{xcode}
  data PackOption = Side SideSpec  | Fill FillSpec 
                  | Expand Toggle  | Anchor Anchor
                  | IPadX Distance | IPadY Distance
                  | PadX Distance  | PadY Distance
\end{xcode}
The first two constructors are most important here. The
\href{PackOptions.html#PackOptions.SideSpec}{SideSpec}
specifies where the widget is packed (top, bottom, left, right)e, and
\href{PackOptions.html#PackOptions.FillSpec}{FillSpec}
specifies in which direction it expands to fill the available space.
Bear in mind that widgets are packed as tight as possible (in
particular into windows), and that once packed, they are never
repacked. That is, if e.g. a widget is packed against the top, it will
sit in the middle (if no \texttt{Fill X} is specified), and will not
move if a widget is packed against the right-hand side.

\texttt{Expand} just means that the widget expands when the containing
element is expanded (i.e. the window is resized), and \texttt{Anchor}
specifies a gravity (a side to which the widgets stick). The rest
create a padding border around the widget in various directions.

It is quite normal that most of the times the packing will not look
like intended, and you will need to use frame widgets (see
\ref{ssec:frames}).

\subsubsection{The Grid Packer}

The grid packer divides the container widget into a grid, and allows
placement of widgets relative to that grid. To pack a single widget
use the 
\href{Packer.html#Packer.grid}
{following function:}
\begin{xcode}
grid :: Widget w => w -> [GridPackOption] -> IO ()
\end{xcode}

The datatype
\href{GridPackOptions.html#GridPackOptions.GridPackOption}
{\texttt{GridPackOptions}} specifies the packing options for the grid
packer.

Note that within the same container you cannot use different packing
algorithms. The first widget packed into a container defines the
packing for this container.

\subsection{Configurations}

Explain configs here.

\subsection{Resources}

\begin{itemize}
\item Fonts 
\item Geometry
\item Dimensions
\end{itemize}

\section{Events}

\begin{itemize}
\item Binding
\item Tk-Events
\end{itemize}

\section{Widgets}

\subsection{Basic Widgets}


\begin{itemize}
\item Frames
\label{ssec:frames}
\item Buttons
\item Labels
\item Message Boxes
\item Scrollbars
\end{itemize}

\subsection{Menues}

\subsection{The Textwidget}

\subsection{The Canvas Widget}

\section{The Toolkit}

\section{Larger Examples}

The Minesweeper game



\bibliography{intro}
\bibliographystyle{plain}


\end{document}

%%% Local Variables: 
%%% mode: latex
%%% TeX-master: t
%%% End: 
