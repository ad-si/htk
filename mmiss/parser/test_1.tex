\documentclass[landscape, slides, light]{mmiss2}
\usepackage{latexsym}
\usepackage{casl}
\begin{document}
\begin{Package}[MMISS_Latex2]{Algebraic
Specification}{Date=6.09.2002,Version=4,PreviousVersion=3,Authors={Markus
Roggenbach, Michael Drouineaud},PriorAuthors={Markus
Roggenbach},Comment=Test,Language=de,LevelOfDetail=LectureNotes,InteractionLevel=Hyper}

% \maketitle  

\begin{Section}[MMISS_Latex2]{Chapter1}{Introduction}{Date=6.09.2002,Version=4,PreviousVersion=3,Authors={Markus
Roggenbach, Michael Drouineaud},PriorAuthors={Markus Roggenbach}}
\begin{Abstract}[]{a1}{Abstract}{}
  Dies ist der Abstract.
\end{Abstract}
\begin{Introduction}[]{intro1}{Introduction}{}
\begin{TextFragment}[Latex]{tf1}{}
    Dies ist die Einf\"uhrung.
\end{TextFragment}
\end{Introduction}

\begin{Section}{Section1.1}{Formal Methods in Software Design}{}
  Text, direkt am Beginn der Section1.1 
  ``Use of mathematics in software development''\\
\begin{TextFragment}[]{}{}
  Dies ist ein TextFragment hinter der Einf\"uhrung mit
  \Emphasis{hervorgehobenem} Text. Ausserdem steht hier noch
  ein \LaTeX{}-Befehl drin.
  \Link[]{para2}{}
  \ForwardLink[]{para2}{}
  \Reference{para2}{}
  \ForwardReference{para2}{}
\end{TextFragment}

\begin{Paragraph}[]{para1}{First Paragraph}{Authors=Achim Mahnke,Date=11.09.2002}
\begin{Definition}[]{def1}{First Definition}{}
  Text innerhalb der Definition (muss zum textFragment werden).

\begin{Program}[C]{}{First program}{}
\begin{ProgramFragment}[]{}{}{}
    Programtext in C.
\end{ProgramFragment}
  Dieser Text erscheint hinter dem ProgramFragment.
\end{Program}
\end{Definition}
\end{Paragraph}
\end{Section}
\begin{Paragraph}[]{para2}{Second Paragraph}{Authors=Achim Mahnke,Date=11.09.2002}
\begin{Table}[LaTeX]{Table1.1.1}{Waterfall Model}{}
{\small
\begin{center}
\begin{tabular}{lcl}
               & Requirement Elicitation & \\ 
               & and Analysis            &\\
& $\uparrow$ $\downarrow$ &\\
Nat.~Lang.   & Non Formal Specification &\\
& $\uparrow$ $\downarrow$ & Validation \\

\Emphasis{Spec.~Lang.}  & Formal Specification & \\
& $\uparrow$ $\downarrow$ & ``Inv.~\& Verify'' or\\

Progr.~Lang. & Implementation & Transformation \\
& $\uparrow$ $\downarrow$ &\\

& Test & \\

& $\uparrow$ $\downarrow$ &\\

& Maintenance& \\

\end{tabular}
\end{center}
}
\end{Table}
\begin{List}{List1.1.1}{itemize}{}
\ListItem{}
{\Emphasis{ writing} formal specifications}
\ListItem{}
{\Emphasis{ proving} properties about formal specifications}
\ListItem{}
{\Emphasis{ constructing} a program by mathematical \\
            manipulation of a formal specification}
\ListItem{}
{\Emphasis{ verifying} a program by mathematical argument}
\end{List}
\end{Paragraph}
\IncludeGroup{para1}{}
\end{Section}
\end{Package}
\end{document}

