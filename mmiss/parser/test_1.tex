\documentclass[landscape, slides, light]{mmiss2}

\usepackage{latexsym}
\usepackage{casl}

\newcommand{\gram}[1]{{\texttt{#1}}}
\newcommand{\power}{\hat{\ }}
\newcommand{\annoimplies}{\%{\bf implies}\ }
\newcommand{\annodef}{\%{\bf def}\ }
\newcommand{\annocons}{\%{\bf conservatively}\ }
\newcommand{\annoprec}{\%{\bf prec}\ }
\newcommand{\annorassoc}{\%{\bf right\ assoc}\ }
\newcommand{\annolassoc}{\%{\bf left\ assoc}\ }
\newcommand{\annobrackets}{{\bf brackets}\ }


%\title[Algebraic Specification]{Algebraic Specification --\\[1ex]A
%  Formalism to Specify\\[1ex] Sequential Programs}

%\author[M.Roggenbach]{Markus Roggenbach} 
%\date{April 2002}
\begin{document}
\begin{Package}[MMISS_Latex2]{pack1}{Algebraic
Specification}{Date=6.09.2002,Version=4,PreviousVersion=3,Authors={Markus
Roggenbach, Michael Drouineaud},PriorAuthors={Markus
Roggenbach},Comment=Test,Language=Schwitzer
D\"utsch,LevelOfDetail=LectureNotes,InteractionLevel=Hyper}

% \maketitle  

\begin{Section}[MMISS_Latex2]{Chapter1}{Introduction}{Date=6.09.2002,Version=4,PreviousVersion=3,Authors={Markus
Roggenbach, Michael Drouineaud},PriorAuthors={Markus Roggenbach},Comment=Test,Language=English,LevelOfDetail=LectureNotes,InteractionLevel=Hyper}

Text, direkt am Beginn der Section1

 
\begin{Section}{Section1.1}{Formal Methods in Software Design}{}
Text, direkt am Beginn der Section1.1 
``Use of mathematics in software development''\\

\begin{Abstract}[]{a1}{Abstract}{}
Dies ist der Abstract.
\end{Abstract}

Text zwischen Abstract und Introduction

\begin{Introduction}[]{intro1}{Introduction}{}
Dies ist die Einf\"uhrung.
\end{Introduction}
\begin{TextFragment}
Dies ist ein TextFragment hinter der Einf\"uhrung mit
\Emphasis{hervorgehobenem} Text. Ausserdem steht hier noch
ein \LaTeX{}-Befehl drin.
\end{TextFragment}
\begin{Paragraph}[]{para1}{First Paragraph}{author=Achim Mahnke,Date=11.09.2002}
\begin{Definition}[]{def1}{First Definition}{}
Text innerhalb der Definition (muss zum textFragment werden).

\begin{Program}[C]{}{First program}{}
\begin{ProgramFragment}[]{}{}{}
  Programtext in C.
\end{ProgramFragment}
Dieser Text erscheint hinter dem ProgramFragment.
\end{Program}
\end{Definition}
\end{Paragraph}
\end{Section}

\begin{Table}[LaTeX]{Table1.1.1}{Waterfall Model}{}
{\small
\begin{center}
\begin{tabular}{lcl}
               & Requirement Elicitation & \\ 
               & and Analysis            &\\
& $\uparrow$ $\downarrow$ &\\
Nat.~Lang.   & Non Formal Specification &\\
& $\uparrow$ $\downarrow$ & Validation \\

\Emphasis{Spec.~Lang.}  & Formal Specification & \\
& $\uparrow$ $\downarrow$ & ``Inv.~\& Verify'' or\\

Progr.~Lang. & Implementation & Transformation \\
& $\uparrow$ $\downarrow$ &\\

& Test & \\

& $\uparrow$ $\downarrow$ &\\

& Maintenance& \\

\end{tabular}
\end{center}
}
\end{Table}
\begin{List}{List1.1.1}{itemize}{}
\ListItem{}
{\Emphasis{ writing} formal specifications}
\ListItem{}
{\Emphasis{ proving} properties about formal specifications}
\ListItem{}
{\Emphasis{ constructing} a program by mathematical \\
            manipulation of a formal specification}
\ListItem{}
{\Emphasis{ verifying} a program by mathematical argument}
\end{List}
\end{Section}
\end{Package}
\end{document}

