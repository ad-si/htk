\documentclass[landscape, slides, light]{mmiss2}

\usepackage{latexsym}
\usepackage{casl}

\newcommand{\gram}[1]{{\texttt{#1}}}
\newcommand{\power}{\hat{\ }}
\newcommand{\annoimplies}{\%{\bf implies}\ }
\newcommand{\annodef}{\%{\bf def}\ }
\newcommand{\annocons}{\%{\bf conservatively}\ }
\newcommand{\annoprec}{\%{\bf prec}\ }
\newcommand{\annorassoc}{\%{\bf right\ assoc}\ }
\newcommand{\annolassoc}{\%{\bf left\ assoc}\ }
\newcommand{\annobrackets}{{\bf brackets}\ }


%\title[Algebraic Specification]{Algebraic Specification --\\[1ex]A
%  Formalism to Specify\\[1ex] Sequential Programs}

%\author[M.Roggenbach]{Markus Roggenbach} 
%\date{April 2002}
\begin{document}
\begin{Package}[MMISS_Latex2]{pack1}{Algebraic
Specification}{date=6.09.2002,versionId=4,previousVersion=3,authors={Markus
Roggenbach, Michael Drouineaud},priorAuthors={Markus
Roggenbach},versionComment=Test,xml:lang=de,levelOfDetailId=LectureNotes,interactionLevelId=Hyper}

% \maketitle  

\begin{Section}[MMISS_Latex2]{Chapter1}{Introduction}{date=6.09.2002,versionId=4,previousVersion=3,authors={Markus
Roggenbach, Michael Drouineaud},priorAuthors={Markus Roggenbach}}\begin{Abstract}[]{a1}{Abstract}{}
  Dies ist der Abstract.
  \end{Abstract}\begin{Introduction}[]{intro1}{Introduction}{}
    Dies ist die Einf\"uhrung.
  \end{Introduction}

  \begin{Section}{Section1.1}{Formal Methods in Software Design}{}
  Text, direkt am Beginn der Section1.1 
  ``Use of mathematics in software development''\\
  \begin{TextFragment}
  Dies ist ein TextFragment hinter der Einf\"uhrung mit
  \Emphasis{hervorgehobenem} Text. Ausserdem steht hier noch
  ein \LaTeX{}-Befehl drin.
  \end{TextFragment}
  \begin{Paragraph}[]{para1}{First Paragraph}{authors=Achim Mahnke,date=11.09.2002}
    \begin{Definition}[]{def1}{First Definition}{}
      Text innerhalb der Definition (muss zum textFragment werden).

      \begin{Program}[C]{}{First program}{}
        \begin{ProgramFragment}[]{}{}{}
          Programtext in C.
        \end{ProgramFragment}
        Dieser Text erscheint hinter dem ProgramFragment.
      \end{Program}
    \end{Definition}
  \end{Paragraph}
\end{Section}
\begin{Paragraph}[]{para2}{Second Paragraph}{authors=Achim Mahnke,date=11.09.2002}
\begin{Table}[LaTeX]{Table1.1.1}{Waterfall Model}{}
{\small
\begin{center}
\begin{tabular}{lcl}
               & Requirement Elicitation & \\ 
               & and Analysis            &\\
& $\uparrow$ $\downarrow$ &\\
Nat.~Lang.   & Non Formal Specification &\\
& $\uparrow$ $\downarrow$ & Validation \\

\Emphasis{Spec.~Lang.}  & Formal Specification & \\
& $\uparrow$ $\downarrow$ & ``Inv.~\& Verify'' or\\

Progr.~Lang. & Implementation & Transformation \\
& $\uparrow$ $\downarrow$ &\\

& Test & \\

& $\uparrow$ $\downarrow$ &\\

& Maintenance& \\

\end{tabular}
\end{center}
}
\end{Table}
\begin{List}{List1.1.1}{itemize}{}
\ListItem{}
{\Emphasis{ writing} formal specifications}
\ListItem{}
{\Emphasis{ proving} properties about formal specifications}
\ListItem{}
{\Emphasis{ constructing} a program by mathematical \\
            manipulation of a formal specification}
\ListItem{}
{\Emphasis{ verifying} a program by mathematical argument}
\end{List}
\end{Paragraph}
\end{Section}
\end{Package}
\end{document}

