%\documentclass[light-hb,landscape,slides]{mmisslides}
%\documentclass[light-hb,landscape,slides]{mmiss}
\documentclass[light-hb,slides]{mmiss}

% \usepackage{times}
\usepackage{amssymb}
\usepackage{amsmath}
% \usepackage{epsfig}
\usepackage{isolatin1}
\usepackage{german}
% \usepackage{supertabular}
% \usepackage{diagrams}
\usepackage{verbatim}
%\usepackage{pdf-effects}
% wird durch mmiss.cls geladen - CK

\input{../example-ontology}
\input{mmiss-prelude}

% \include{prelude}


\begin{document}

\begin{Package}[Title={Praktische Informatik 3}, Authors={Christoph
                L{\"u}th}, Date={WS 02/03}, LevelOfDetail=Lecture,
                Language=de, Label={Functional_Programming}]


\scatterslidebody{}
\setcounter{page}{30}

\begin{Section}[Title={Introduction to functions},
  ShortTitle={Functions}, Label={FuncIntro}]

\begin{Paragraph}[Label=DefFunctions,
Title={\Def[Definition of Functions]{FunctionDefinitionH}},
Formalism=Haskell]
How to declare and define a function:
\begin{Itemize}[Label=List1]
  \item \Def[Signature]{SignatureH} (optional):
    \begin{ProgramFragment}[Label=SigMax]
      max :: Int-> Int-> Int
    \end{ProgramFragment}
  \item \Def[Equation]{FunctionDeclarationH} (one or many):
    \Def[]{MaxF} \Def[]{GreaterThanF} 
    \Relate{calls}{MaxF}{GreaterThanF}
    \begin{ProgramFragment}[Label=EqMax]
      max x y = if x < y then y else x    
    \end{ProgramFragment}
  \begin{Itemize}[Label=List2]
    \item \Def{LhsH}: head with parameters
    \item \Def{RhsH}: body (usually more than one line)
    \item Typical pattern: \Ref{CaseDistinctionH} and \Ref{Recursion}.
  \end{Itemize}
\end{Itemize}
\end{Paragraph}

\end{Section}
\end{Package}

\end{document}

