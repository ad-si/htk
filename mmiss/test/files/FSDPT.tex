\documentclass[landscape, autoslides, light]{mmiss}

%\input{diagrams}
%\diagramstyle[size=2cm,heads=blacktriangle]
\MMlogo{TU_ohne}
\usepackage{latexsym}
%\usepackage{epsfig}
\input xy
\xyoption{all} \xyoption{2cell}
\usepackage{amsfonts}
\usepackage{amssymb}
\usepackage[all]{xy}
\usepackage{proof}
\usepackage{xspace}
\usepackage{casl}








\newcommand{\ns}{\normalsize}
\newcommand{\kl}{[ \hspace{-4pt} [ \hspace{4pt}}
\newcommand{\kr}{ \hspace{4pt} ] \hspace{-4pt} ] }
\newcommand{\eqd}{\stackrel{\mathrm def}{=}}
\newcommand{\ekl}{[\![}
\newcommand{\ekr}{]\!]}
\newcommand{\nat}{{I\!\!N}}
\newcommand{\CASL}{\textmd{\textsc{Casl}}\xspace }
\newcommand{\CoCASL}{\textmd{\textsc{CoCasl}}\xspace }
\newcommand{\vsp}{\pause\vspace{8mm}}
\maketoc

\begin{document}
\begin{Package}[Label={FSDPT}, Title={Formal Specification of Data and Process Types}, ShortTitle={FSDPT}, Authors={Horst Reichel}, Date={February 2003}, LevelOfDetail=Lecture, Language=en-GB]


\begin{Section}[Title={Literature}, Label={section1}]

%............................................

\begin{Paragraph}[Title={Data Types}, Label=Paragraph1]
\small
\begin{List}[ListType=itemize]
\ListItem J. Loeckx, H.--D. Ehrich, M. Wolf: Specification of
   Abstract Data Types, Wiley and Teubner, 1996
 \ListItem E. Astesiano, H.--J. Kreowski, B. Krieg--Br{\"u}ckner
       (Eds.): Algebraic Foundations of Systems Specification, IFIP
      State--of--the--art reports, Springer,1999
\ListItem   N. Shankar, S. Owre, J. M. Rushby, D. W.
Stringer--Calvert: PVS Prover Guide, Version 2.4, SRI
International, November 2001, (http://pvs.csl.sri.com)
 \ListItem The algebraic specification language \CASL,
 (http://www.brics.dk/Projects/CoFI/CASL.html)
%%%%%%%%%%%%%%%%%
%\MMstartApplication{knopf}{C:/JFLAP04/JFLAP.jar}
%%%%%%%%%%%%%%%%%
\end{List}

%............................................

\end{Paragraph}
\begin{Paragraph}[Title={Process Types}, Label=Paragraph2]
\begin{List}[ListType=itemize]
\ListItem B. Jacobs: Exercises in coalgebraic specification, In
  \textsc{R. Backhous,  R. Crole} and \textsc{J. Gibbons} (Eds.):
  Algebraic and Coalgebraic Methods in the Mathematics of
  Program Construction, vol. 2297 of Lecture Notes in
  Computer Scienc, pages 237--280, Springer, Berlin, 2002
 % \item H. Tews: The Coalgebraic Class Specification Language CCSL
 % - Syntax and Semantics - Technical Report ISSN 1430--211X,
 % TU Dresden Fakult{\"a}t Informatik, URL:
 % http://www.inf.tu-dresden.de,
 \ListItem Proceedings of the CMCS--Workshops, ENTCS Volumes 11, 19,
 33, 44, 65
 \ListItem T. Mossakowski, H. Reichel, M. Roggenbach, L.
 Schr\"oder: Algebraic-co-algebraic specification in \CoCASL, to
 appear in Proceedings of WADT 2002, Springer LNCS.
\end{List}
\ns

\end{Paragraph}
\end{Section}
\begin{Section}[Title={Introduction}, Label={section2}]

%............................................

\begin{Paragraph}[Title={Kinds of  Specifications}, Label=Paragraph3]
\pause

There are several kinds of specifications \pause

\begin{List}[ListType=itemize]
 \ListItem model oriented specifications \pause
 \ListItem axiomatic specifications \pause
 \ListItem executable specifications
\end{List}

%............................................

\end{Paragraph}
\begin{Paragraph}[Title={Model oriented Specifications}, Label=Paragraph4]

\small
In model oriented specifications a high level formal language is
used to describe a specific model for the intended data type.
Those high level formal languages may be specially designed
languages like \textbf{Z} (see J. Bowen: Formal Specification and
Documentation using Z, International Thomson Computer Press,
London, 1996) or it can be a \textbf{type theory}.

In this Lecture we will not present this style of specifications.

The specification language \hyperlink{BOBJ}{BOBJ} is an example
with executable specifications.
\ns

%............................................

\end{Paragraph}
\begin{Paragraph}[Title={Axiomatic Specifications}, Label=Paragraph5]

This lecture will focus on axiomatic specifications. By means of
suitable logics classes of models will represent abstract type,
such that each model represents a correct implementation of the
abstract type.\pause

This approach raises the question: \Emphasis{What kind of
mathematical structures can be used to represent correct
implementations of data and process types?}

%............................................

\end{Paragraph}
\begin{Paragraph}[Title={Basic Intuition}, Label=Paragraph6]
\pause

The foundations presented in the course are based on the
assumption that \pause
\begin{List}[ListType=itemize]
    \ListItem \textcolor{blue}{a \textbf{data type} can be represented
    by a set of \textbf{constructors} and their properties}, and \pause
    \ListItem \textcolor{blue} {a \textbf{process type} can be
    represented by a set of \textbf{observers (atomic experiments, destructors)}
    and their properties}.
\end{List}

\end{Paragraph}
\begin{Paragraph}[Title={Basic Intuition}, Label=Paragraph7]

To illustrate this rather abstract formulation, let us discuss
informally some examples.

%............................................

\end{Paragraph}
\begin{Paragraph}[Title={Basic Intuition}, Label=Paragraph8]


>From the Peano axiomatization of natural numbers we know that the
data type of natural numbers can be defined by two constructors:
\vspace{8mm} \small
\begin{verbatim}
   BEGIN Nat :  SPEC
       SORT        Nat ;
       CONSTRUCTOR  zero : -> Nat;
                    suc  : Nat ->Nat
   END Nat
\end{verbatim}
\ns

%............................................

\end{Paragraph}
\begin{Paragraph}[Title={Basic Intuition}, Label=Paragraph9]

satisfying the following (second--order) axioms:\vspace{8mm}

 \begin{List}[ListType=enumeration]
 \ListItem $\forall x_{Nat}. \neg (suc(x) = zero )$; \pause
 \ListItem  $\forall x_{Nat} , y_{Nat} . suc(x) = suc(y) \Rightarrow
    x = y $; \pause
 \ListItem  $\forall R_{Nat \to Bool} . ((R(zero) \wedge \forall x_{Nat} .
   (R(x) \Rightarrow R(suc(x)))) \Rightarrow \forall y_{Nat} .
   R(y))$
 \end{List}

%............................................

\end{Paragraph}
\begin{Paragraph}[Title={Basic Intuition}, Label=Paragraph10]

Similarly, the data type of finite lists can be defined by two
constructors: \vspace{8mm} \small
\begin{verbatim}
   BEGIN List[ X : TYPE] : SPEC
      SORT               List;
      CONSTRUCTOR  Nil : -> List;
                   Cons : X * List -> List
   END List
\end{verbatim}
\ns


%............................................

\end{Paragraph}
\begin{Paragraph}[Title={Basic Intuition}, Label=Paragraph11]

satisfying the following properties:\vspace{8mm}
\begin{List}[ListType=enumeration]
    \ListItem  $\forall x_{Elem}, l_{Lists}. \neg( cons(x,l) = nil) $ \pause
    \ListItem  $\forall x_{Elem}, y_{Elem}, l_{Lists}, r_{Lists}.
   (cons(x,l) = cons(y,r) \Rightarrow $ \\
      $(x = y \wedge l = r)) $ \pause
    \ListItem $\forall R_{Lists \to Bool}.(R(nil) \wedge \forall x_{Elem}, l_{Lists}.
   (R(l) \Rightarrow R(cons(x,l))) \Rightarrow \forall r_{Lists} .
   R(r))$
\end{List}

%............................................

\end{Paragraph}
\begin{Paragraph}[Title={Basic Intuition}, Label=Paragraph12]

We see that correct implementations of data types can be
represented by first--order structures of a many--sorted signature
(alphabet), whose properties can be expressed in first-- and
second--order logic.

%............................................

\end{Paragraph}
\begin{Paragraph}[Title={Basic Intuition - Initiality}, Label=Paragraph13]

The  axioms for natural number and for list can equivalently be
expressed in the following way:\pause \\ Let $\mathbb{N}$ denote
the \texttt{Nat}--structure of natural numbers with the usual
interpretation of the two constructors. Then the  axioms are
equivalent to the following \emph{\textbf{initiality axiom:}}
\begin{quote} For each $\Sigma_{Nat}$--structure $\mathbb{A}$
there exists exactly one homomorphism \\$ h : \mathbb{ N} \to
\mathbb{A}$ \end{quote}

%............................................

\end{Paragraph}
\begin{Paragraph}[Title={Basic Intuition - Initiality}, Label=Paragraph14]

Let $ \mathbb{L}(M)$ denote the \texttt{List(M)}--structure which
interprets \texttt{Nil} as the empty list and \texttt{cons} as
extending a given list by a given element out of the set $M$. Then
the axioms are equivalent to \begin{quote} For each
\texttt{List(M)}--structure $ \mathbb{A}(M)$ there exits exactly
one homomorphism $$ h : \mathbb{L}(M) \to \mathbb{A}(M)$$ which is
the identity on $M$.\end{quote}

%............................................

\end{Paragraph}
\begin{Paragraph}[Title={Basic Intuition - Initiality}, Label=Paragraph15]

The initiality axiom provides us with a powerful
\textcolor{blue}{\textbf{definition and proof principle for data
types:}}
 \pause
\begin{List}[ListType=itemize]
 \ListItem the \textbf{definition principle} is based on the
\textbf{existence} of the homomorphism  \pause \ListItem the
\textbf{proof principle} is based on the \textbf{uniqueness} of
the homomorphism.
\end{List}

%............................................

\end{Paragraph}
\begin{Paragraph}[Title={Basic Intuition - Finality}, Label=Paragraph16]

To discuss process types, let us start with the \emph{signature of
partial deterministic automata}:

\small
\begin{verbatim}
    BEGIN Automata[In , Out : Type] : SPEC
       SORT         State ;
       OPERATIONS   next : In * State -> State + 1 ;
                    out : State -> Out + 1
    END
 \end{verbatim}
 \ns

 where + denotes the disjoint union and $1 = \{ * \}$ the singleton
 set.

%............................................

\end{Paragraph}
\begin{Paragraph}[Title={Basic Intuition - Finality}, Label=Paragraph17]

For fixed interpretation of the input and output alphabets by sets
$A$ and $B$ an $(A, B)$--automaton is given by
$$ \mathbb{M} = (A, B, M, next_{\mathbb{M}}: A \times M \to M + 1,
out_{\mathbb{M}}: M \to B +1)$$ For a given state $m \in M$ we can
define the \textcolor{blue}{behaviour of $ \mathbb{M}$ at $m \in
M$} $$ beh_{m, \mathbb{M}} : A^* \to B + 1 $$ as follows:


Let $$ next^* : A^* \times M \to M +1 $$ be defined by
\small
$$\begin{array}{lll}next^*(\varepsilon, m)&  = &  m \; \mbox{ for
all } m \in M,\\ & & \\
 next^*(aw,m)&  = & \left\{ \begin{array}{ll}
 m' & \mbox{ if } next^*(w,m) = m_1 \mbox{ and } next(a,m_1) =
 m'\\  &
 \mbox{ for some } m_1 \in M \\
 \ast & \mbox{else} \end{array} \right .\end{array} $$
where $w \in A^*, a \in A$.
\ns
 Then the behaviour function is
given by:

$$\begin{array}{lll}
beh_{m,\mathbb{M}}(\varepsilon) & = & out_{\mathbb M}(m) \\
 &  &  \\
beh_{m,\mathbb{M}}(aw) & = & \left\{\begin{array}{ll} out_{\mathbb
M}(m') & \mbox{if } next^*(aw,m) = m' \\ \ast & else
\end{array} \right. \end{array}$$

One can easily check that for each behaviour function holds:

If $beh_{m,\mathbb{M}}(w) = *$ then $beh_{m,\mathbb{M}}(aw) = *$
for all $w \in A^*, a \in A$.

Now the behaviour functions itself can be seen as states of an
$(A, B)$--automaton
$$\mathbb{F} = (A, F, B; next_{\mathbb F}: A
\times F \to F + \mathbf{1}, out_{\mathbb F}: F \to B +
\mathbf{1})$$ with $$ F = \{ \beta: A^* \to B + \mathbf{1}\; | \;
\forall a,w .(\beta(w) = \ast) \Rightarrow (\beta(aw) = \ast)\}$$
and $$\begin{array}{lll} next_{\mathbb F}(a, \beta) & = & \lambda
w . \beta(aw)\\ out_{\mathbb F} (\beta) & = & \beta(\varepsilon)
\end{array}$$

It turns out that there is a unique homomorphism
$$h : \mathbb{M} \to \mathbb{F}$$  given
 by $h(m) = \beta_{\mathbb{M}, m}$, i.e.\begin{center}
 \textcolor{blue}{$\mathbb{F}$ is a final $(A, B)$--automaton.}
\end{center}
\pause In other words: \\ \pause \textcolor{red}{The final
$(A,B)$--automaton $\mathbb F$ represents the type of behaviours
of all $(A,B)$--automata.}

\end{Paragraph}
\begin{Paragraph}[Title={Conclusion}, Label=Paragraph18]
\pause
 \textcolor{red}{\textbf{
\begin{List}[ListType=itemize]
    \ListItem initial structures can be used to represent data types.
    \pause
    \ListItem final structure can be used to represent types of
    behaviours, i.e. process types.
\end{List}}}

%............................................

\end{Paragraph}
\begin{Paragraph}[Title={Notation}, Label=Paragraph19]

\small
\begin{List}[ListType=itemize]
\ListItem \textbf{Composition of mappings $(;)$}: If $f: A \to B, g :
B \to C$ then $f;g : A \to C$ with $(f;g)(x) = g(f(x))$ for each
$x \in A$.\pause

\ListItem \textbf{Composition of mappings $(\circ)$}: If $f: A \to B,
g : B \to C$ then $g \circ f : A \to C$ with $(g \circ f)(x) =
g(f(x))$ for each $x \in A$.\pause

\ListItem For any set $M$ there exists a unique mapping $!_M : M \to
\mathbf{1}$. \pause

\ListItem For any  sets $A, B$ there are \textbf{projections} $\pi_1 :
A \times B \to A, \; \pi_2 : A \times B \to B$ and for each pair
of mappings $f: M \to A, \; g : M \to B$ there is a unique mapping
$\langle f,g \rangle : M \to A \times B$ with $ \langle f,g
\rangle ;\pi_1 = f, \langle f,g \rangle ;\pi_2 = g$.\pause

\ListItem A consequence is the equality $\langle \pi_1, \pi_2 \rangle
= Id_{A \times B}$.

\ListItem  If $f: A \to B, g : M \to N$ are arbitrary mappings, then
$f \times g : A \times M \to B \times N$ denotes the mapping $
\langle \pi_1 ;f, \pi_2 ;g): A \times M \to B \times N$. \pause

\ListItem For any sets $A, B$ there are \textbf{injections} $\kappa_1
: A \to A + B, \kappa_2 : B \to A + B$ and for each pair of
mappings $f : A \to C, g : B \to C$ there is a unique mapping
$[f,g] : A + B \to C$ with $\kappa_1 ; [f,g] = f $ and $\kappa_2 ;
[f,g] = g $.\pause

\ListItem A consequence is the equality $[ \kappa_1, \kappa_2 ]= Id_{A
+ B}$.\pause

\ListItem If $f: A \to B, g : M \to N$ are arbitrary mappings, then $f
+ g : A + M \to B + N$ denotes the mapping $[ f ; \kappa_1 , g ;
\kappa_2 ] : A + M \to B + M$. \ns
\end{List}

%............................................

\end{Paragraph}
\begin{Paragraph}[Title={Specification Languages and Tools}, Label=Paragraph20]

There are several specification languages, supported by an
associated tool, which can be used to specify data and process
types: \pause

\begin{List}[ListType=itemize]
    \ListItem CCSL: Coalgebraic Class Specification Language, (TU
    Dresden, Technical Report: TUD--FI02--08--August 2002,
    (http://wwwtcs.inf.tu-dresden.de/~tews/ccsl/)\pause
    \ListItem \CASL (and its extension to \CoCASL): CoFI, The
     Common Framework Initiative for Algebraic Specification and
     Development
     (http://www.brics.dk/Projects/CoFI/Documents/CASL/)\pause
    \ListItem \hypertarget{BOBJ}{BOBJ}: Part of the TATAMI--Project
    (http://www.cs.ucsd.edu/groups/tatami/ and
     http://www.cs.ucsd.edu/groups/tatami/bobj/)
\end{List}


\end{Paragraph}
\end{Section}
\begin{Section}[Title={Initial Algebras as Data Types}, Label={section3}]

\begin{Section}[Title={Existence of initial models}, Label={section3_1}]
\begin{Paragraph}

%............................................

\end{Paragraph}
\begin{Paragraph}[Title={First Examples}, Label=Paragraph21]

The examples of natural numbers and list would be represented in
CCSL as follows:

\begin{verbatim}
  BEGIN nat : INITIAL SPEC
      SORT          Nat ;
      CONSTRUCTOR   zero : -> Nat ;
                    succ : Nat -> Nat
  END nat
\end{verbatim}
\end{Paragraph}
\begin{Paragraph}[Label=Paragraph22]
\begin{verbatim}
  BEGIN list[B : TYPE] : INITIAL SPEC
      SORT          List ;
      CONSTRUCTOR   nil : -> List ;
                    cons : B * List -> List
  END list
\end{verbatim}


 The corresponding specifications in
\hyperlink{nat-casl}{CASL} or in \hyperlink{nat-bobj}{BOBJ} differ
only on the syntactic level.

%............................................

\end{Paragraph}
\begin{Paragraph}[Label=Paragraph23]

The key word \texttt{INITIAL} within the specifications above
constrains the model classes to the initial ones. But, what about
the axioms in the Peano axiomatization given above? The
CCSL--specifications do not contain any axioms (explicitly). \\

%............................................

\end{Paragraph}
\begin{Paragraph}[Label=Paragraph24]
To discuss this problem we jump on a more abstract level,
syntactically independent from the specific specification
languages.

\pause

\small
\begin{Definition}[Title = {Many--sorted first--order signature}, Label = {signature}]
 A many--sorted first--order signature $$\Sigma =
(Sorts, Ops, Pred)$$ is given by a finite set $Sorts$ of sort
symbols and a finite set $Ops$ of typed operation symbols, each of
the form $\sigma : w \to s$ with $s \in Sorts$, $w \in Sorts^*$,
and a finite set of typed predicate symbols $\varrho : w$.

A $\Sigma$--structure ($\Sigma$--algebra) $$\mathbb{A} = ((A_s | s
\in Sorts), (\sigma_{\mathbb A} : A_w \to A_s \; | \; \sigma \in
Ops), (\varrho_{\mathbb A} \subseteq A_w \; | \;\varrho \in Pred
))$$ is given by a $Sorts$--indexed family of sets, and an
$Ops$--indexed family of functions $\sigma_{\mathbb A} : A_w \to
A_s $ where $A_{s_1\ldots s_n} = A_{s_1} \times \ldots A_{s_n}$
and $A_{\varepsilon} = \mathbf{1}$ and a $Pred$--indexed family of
sets $(\varrho_{\mathbb A} \subseteq A_w \; | \; \varrho \in
Pred)$.

A $\Sigma$--homomorphism $$h : \mathbb{A} \to \mathbb{B}$$ is
given by a $Sorts$--indexed family $h = (h_s : A_s \to B_s\; | \;
s \in Sorts)$, such that for each $\sigma : w \to s$, each
$\varrho : w$ and each $(a_1, \ldots , a_n) \in A_w$ the following
homomorphism conditions are satisfied: $$\begin{array}{ll}
 h_s(\sigma_{\mathbb A}(a_1, \ldots ,a_n)) = \sigma_{\mathbb
 B}(h_{s_1}(a_1), \ldots , h_{s_n}(a_n) \\
 (a_1, \ldots a_n) \in \varrho_{\mathbb A} \quad \Rightarrow \quad
 (h_{s_1}(a_1), \ldots , h_{s_n}(a_n)) \in \varrho_{\mathbb B}
 \end{array}$$ \end{Definition}


%............................................

\end{Paragraph}
\begin{Paragraph}[Label=Paragraph25]
 \small
 Recall, that for a given class $\mathcal{C}$ of
 $\Sigma$--structures a $\Sigma$--structure $\mathbb{I}$ is called
 \textbf{initial in $\mathcal{C}$}, if for each $\Sigma$--structure
 $\mathbb{M} \in \mathcal{C}$ there is exactly one
 $\Sigma$--homomorphism $h : \mathbb{I} \to \mathbb{M}$.


 By induction one can extend the typing of operation symbols to
typing of $\Sigma$--terms, such that for any term $t : s_1 \ldots
s_n \to s$ and any $\Sigma$--structure $\mathbb A$ one gets the
interpretation $$\ekl t \ekr_{\:{\mathbb A}} : A_w \to A_s$$ of
the term $t$ in $\mathbb A$.

%............................................

\end{Paragraph}
\begin{Paragraph}[Label=Paragraph26]
If the terms contain variables, i.e., if $ t \in T_s(\Sigma, X)$,
where $X = (X_s \; | \; s \in Sorts)$ is a set of typed
first--order variables, one needs in addition a valuation of the
variables, i.e., a mapping $\varphi = (\varphi_s : X_s \to A_s)$
to define inductively the value $$\ekl t \ekr_{\varphi} \in A_s$$
if $t \in T_s(\Sigma, X)$.

%............................................

\end{Paragraph}
 \begin{Paragraph}[Title={Minimal Herbrand Model}, Label=Paragraph27]

 The minimal Herbrand model $\mathbb{T}^{\Sigma}$ is given by the
ground terms ( variable free terms ), which means
$\mathbb{T}^{\Sigma}_s = T_s(\Sigma, \emptyset)$ then there is
exactly one $\Sigma$--homomorphism
$$ h^{\mathbb A} : \mathbb{T}_{\Sigma} \to {\mathbb A}$$ defined
by $ h^{\mathbb A}_s(t) = \ekl t \ekr_{\:{\mathbb A}}$. It is
worth to notice that in the minimal Herbrand model each relation
symbol $r \in Pred$ is interpreted by the empty set. \ns

%............................................

\end{Paragraph}
 \begin{Paragraph}[Title={Minimal Herbrand Model}, Label=Paragraph28]
\begin{Corollary}[Title = {Initial Algebras as Data Types}, Label = {Corollary1}]
For any first--order signature $\Sigma$ the minimal Herbrand model
$\mathbb{T}^{\Sigma}$ is an initial model in the class of all
$\Sigma$--structures. \end{Corollary}

\vspace{1.5cm}

This observation raises the question: \textcolor{red}{Does an
initial model exist for each class of models?}

%............................................

\end{Paragraph}
\begin{Paragraph}[Label=Paragraph29]
\small Let us consider the class of models defined by the
following specification:\\ \pause
\begin{verbatim}
   BEGIN spec-0 : LOOSE SPEC
     SORT           Char ;
     OPERATION   m : Char * Char -> Char ;
     ASSERTION
     ass  : FORALL(x, y, z : Char). m(x,m(y,z)) = m(m(x,y),z) ;
     idem : EXISTS(x : Char) . m(x,x) = x
   END spec-0
\end{verbatim}\pause

 The class of models is given by  non--empty sets together
with an associative binary operations such that there is at least
one idempotent element.\pause

It is not hard to see, that \textcolor{red}{there is no initial
model for that class}.

%............................................

\end{Paragraph}
\begin{Paragraph}[Label=Paragraph30]

If we modify the specification to
\begin{verbatim}
   BEGIN spec-1 : LOOSE SPEC
     SORT           Char ;
     OPERATION   m : Char * Char -> Char ;
                 c : -> Char ;
     ASSERTION
     ass  : FORALL(x, y, z : Char). m(x,m(y,z)) = m(m(x,y),z) ;
     idem :  m(c,c) = c
   END spec-1
\end{verbatim}
the resulting model class contains an initial model, given by an
one--element structure.

%............................................

\end{Paragraph}
\begin{Paragraph}[Label=Paragraph31]

 In general, for model classes specified by means of axioms
using existential quantifiers, disjunction of negation the
existence of an initial model is not guaranteed.\pause

To see this, let $AX$ be any set of first--order, closed formulae
over $\Sigma$, and let be
$$Mod(AX) = \{ \mathbb{A} \; | \; \mathbb{A}
\vdash F \mbox{ for all } F \in AX \; \}.$$
\
We define a family of relation $$\backsim^{AX} \; = \; (
\backsim_s^{AX} \; \subseteq \; T_s(\Sigma, \emptyset) \times
T_s(\Sigma, \emptyset) \; | \; s \in Sorts )$$ as follows:
$$ t_1 \backsim^{AX} t_2 \quad \mbox{iff} \quad AX \vdash t_1 = t_2 .
$$

%............................................

\end{Paragraph}
\begin{Paragraph}[Label=Paragraph32]
\ns
\begin{Theorem}[Title = {Initial Algebras as Data Types}, Label = {Theorem1}]
For each set of first--order axioms $AX$ the $Sorts$--indexed
family of relations $\backsim^{AX} $ is a congruence relation in
the minimal Herbrand model $\mathbb{T}^{\Sigma}$, i.e., each
$\backsim^{AX}_s $ is an equivalence relation, and for each
function symbol $\sigma : s_1 \ldots s_n \to s_0$ it holds
$$\sigma(t_1, \ldots t_n) \backsim^{AX}_{s_0} \sigma(t'_1, \ldots
, t'_n) \quad \mbox{if} \quad t_i \backsim^{AX}_{s_i} t'_i \;
\mbox{ for } i = 1, \ldots , n .$$\end{Theorem}

%............................................

\end{Paragraph}
\begin{Paragraph}[Label=Paragraph33]
\small
 \textbf{Proof:} The above definition of the relation
 $\backsim^{AX}$ can be expressed equivalently by
 $t_1 \backsim^{AX}_s t_2$ iff $\ekl t_1 \ekr_{\mathbf{A}} = \ekl t_2
 \ekr_{\mathbf{A}}$ for all $\mathbf{A} \in Mod(AX)$, if $t_1, t_2
 \in \mathbb{T}^{\;\Sigma}_s$. From this one gets immediately that
 each relation is an equivalence relation.

 In order to see that $\backsim^{AX}$ is a congruence relation one
 has to remind the inductive definition of $\ekl t
 \ekr_{\;\mathbb{A}}$ which is given by $$\ekl \sigma(t_1, \ldots , t_n)
 \ekr_{\;\mathbb{A}} = \sigma_{\mathbb{A}}(\ekl t_1 \ekr_{\;\mathbb{A}} ,
 \ldots ,\ekl t_n \ekr_{\;\mathbb{A}}).$$ With the assumption
$ t_i \backsim^{AX}_{s_i} t'_i \; \mbox{ for } i = 1, \ldots , n$
we get $$\ekl \sigma ( t_1, \ldots , t_n) \ekr_{\;\mathbb{A}} =
\sigma_{\mathbb{A}}(\;\ekl t_1 \ekr_{\;\mathbb{A}} ,
 \ldots , \ekl t_n \ekr_{\;\mathbb{A}}) = \sigma_{\mathbb{A}}
 (\;\ekl t'_1 \ekr_{\;\mathbb{A}} ,
 \ldots ,\ekl t'_n \ekr_{\;\mathbb{A}}) = \\ \ekl \sigma (t'_1, \ldots ,
 t'_n) \ekr_{\;\mathbb{A}},$$ which means $\sigma (t_1, \ldots t_n)
\backsim^{AX}_{s_0} \sigma(t'_1, \ldots , t'_n) $. \hfill qed.


%............................................

\end{Paragraph}
\begin{Paragraph}[Title={Canonical Structure}, Label=Paragraph34]
 The theorem above and the notation introduced gives us
the possibility to construct a new $\Sigma$--structure, which is
called \textcolor{blue}{the canonical $AX$--structure}, denoted by
\textcolor{blue}{$\mathbb{T}(\Sigma,AX)$}. The carrier sets are
defined by
$$T(\Sigma,AX)_s = \{ \overline{t}_{AX} \; | \; t \in T(\Sigma,
\emptyset) \},$$ the operations are defined by $$
\sigma_{\mathbb{T}(\Sigma,AX)}( \overline{t_1}_{AX}, \ldots ,
\overline{t_n}_{AX}) = \overline{ \sigma (t_1, \ldots ,
t_n)}_{AX}$$
 and the interpretation of the predicate symbols is given by
$$ \varrho_{\mathbb{T}(\Sigma,AX)} =  \{ (\overline{t_1}_{AX},
\ldots , \overline{t_m}_{AX} ) \; | \; AX \vdash \varrho (t_1,
\ldots , t_m) \; \}.$$

%............................................

\end{Paragraph}
\begin{Paragraph}[Label=Paragraph35]
\ns
 Now, the question is, whether $$ \mathbb{T}(\Sigma,AX) \in
Mod(AX)$$ holds?

Let us consider the specification \textsc{Spec-0}. In that case
the induced congruence relation would be the identity relation, so
that $ \mathbb{T}(\Sigma,AX)$ becomes the minimal Herbrand model
which does not satisfy the required axiom. Therefore, in general
the answer to the question above is NO.

%............................................

\end{Paragraph}
\begin{Paragraph}[Label=Paragraph36]

\small
The answer is YES, if the set of axioms satisfy some  conditions.
To specify these conditions, we have to introduce two concepts.
\begin{List}[ListType=enumeration]
  \ListItem A set of axioms $AX$ is called \emph{negation true}, if for
  each formula either $AX \vdash F$ or $AX \vdash \neg F$.
  \ListItem A set of axioms \emph{owns witnesses}, if for each formula
  $\exists x . F$ there exists a term with
  $AX \; \vdash \; \exists x . F \rightarrow F[x := t]$.
\end{List}
Specification \textsc{Spec-1} has a set of axioms  which owns
witnesses.


%............................................

\end{Paragraph}
\begin{Paragraph}[Label=Paragraph37]
\ns
\begin{Theorem}[Title = {Initial Algebras as Data Types}, Label = {Theorem2}] Let $AX$ be a consistent set
of axioms which is negation true and which owns witnesses, then
for each formula $F$ holds
$$\mathbb{T}(\Sigma,AX) \vDash F \quad \mbox{iff} \quad AX \vdash
F$$
\end{Theorem}\pause
\small

The proof of this theorem is strongly related to the completeness
of deduction calculi of first--order logic and will be omitted
here.

%............................................

\end{Paragraph}
\begin{Paragraph}[Label=Paragraph38]

In practical applications the two properties of
negation truth and of owning witnesses are hard to check. \\
\pause For so--called universal Horn--formulae one can prove in a
direct way that the canonical model of a set $AX$  of universal
Horn--formulae is a model of that set.

%............................................

\end{Paragraph}
\begin{Paragraph}[Title={Universal Horn--formulae}, Label=Paragraph39]
The set of \emph{universal Horn--formulae} is inductively defined
by the following set of derivation rules:
\begin{List}[ListType=enumeration]
  \ListItem  If $F_1, \ldots , F_n, F$ for $n \in \nat$ are atomic
  formulae then $\neg F_1 \vee \ldots \vee \neg F_n \vee F$ is an
  universal Horn--formula; \pause
  \ListItem If $F_1, \ldots , F_n$ for $n \in \nat$ are atomic
  formulae then $\neg F_1 \vee \ldots \vee \neg F_n$ is a
  universal Horn--formula;\pause
  \ListItem If $F, G$ are universal Horn--formulae, then the conjunction $F
  \wedge G$ is an universal Horn--formula too; \pause
  \ListItem If $F$ is an universal Horn--formula, then $\forall x . F$
  is a universal Horn--formula.
\end{List}

%............................................

\end{Paragraph}
\begin{Paragraph}[Title={Existence of Initial Models}, Label=Paragraph40]
\ns
\begin{Theorem}[Title = {Initial Algebras as Data Types}, Label = {Theorem3}] Let $AX$ be
a consistent set of universal Horn formulae over a signature
$\Sigma$. Then the canonical $AX$--structure $
\mathbb{T}(\Sigma,AX)$ is an initial model in the class of all
$\Sigma$--structures satisfying each axiom out of $AX$.
\end{Theorem}

%............................................

\end{Paragraph}
\begin{Paragraph}[Label=Paragraph41]

\small \small \textbf{Proof:}  We will prove that for each
universal Horn--formula $G$ holds
$$(*) \hspace{1cm} AX \vdash G \quad \mbox{implies} \quad
\mathbb{T}(\Sigma,AX) \vDash G.$$ The theorem is an immediate
consequence of (*).

First we prove that (*) and its inverse is true for atomic
formulae. If $G$ equals $\varrho (t_1, \ldots , t_m)$ by
definition $\mathbb{T}(\Sigma,AX) \vDash \varrho (t_1, \ldots ,
t_m)$ iff $(\overline{(t_1)}_{AX} , \ldots ,
\overline{(t_m)}_{AX}) \in \varrho_{\mathbb{T}(\Sigma,AX)}$ iff
$AX \vdash \varrho (t_1, \ldots , t_m)$. If $G$ equals $t = t'$
then by definition $\mathbb{T}(\Sigma,AX) \vDash t = t'$ iff
$\overline{t}_{AX} = \overline{t'}_{AX}$ iff $t \backsim^{AX} t'$
iff $AX \vdash t = t'$.

Now we assume that $G$ equals $\neg F_1 \vee \ldots \vee \neg F_n
\vee F$. The case $n = 0$ is proved by (*). We have to prove
$\mathbb{T}(\Sigma,AX) \vDash (F_1 \wedge \ldots F_n \to F)$. Let
be $\mathbb{T}(\Sigma,AX) \vDash (F_1 \wedge \ldots F_n )$ which
means $\mathbb{T}(\Sigma,AX) \vDash F_i$ for $i = 1, \ldots , n$.
Since each $F_i$ is atomic, by (*) this is equivalent to $AX
\vdash F_i, \; i = 1, \ldots , n$. By the precondition $AX \vdash
(F_1 \wedge \ldots F_n \to F)$ this implies $AX \vdash F$ which by
(*) is equivalent to $\mathbb{T}(\Sigma,AX) \vDash F$, so that
finally $\mathbb{T}(\Sigma,AX) \vDash (F_1 \wedge \ldots F_n \to
F)$ holds.

If $G$ equals $\neg F_1 \vee \ldots \neg F_n$ and $AX \vdash \neg
F_1 \vee \ldots \neg F_n$, then $AX \vdash \neg (F_1 \wedge \ldots
\wedge F_n)$. If $\mathbb{T}(\Sigma,AX)$ would not be a model of
$\neg F_1 \vee \ldots \neg F_n$ , then holds
$\mathbb{T}(\Sigma,AX) \vDash F_i, \; i = 1, \ldots , n.$ By $(*)$
this implies $AX \vdash (F_1 \wedge \ldots \wedge F_n)$, i.e., $AX
\vdash \neg F_1 \vee \ldots \neg F_n$ and $AX \vdash (F_1 \wedge
\ldots \wedge F_n)$. This contradicts to the assumption, that $AX$
is a consistent set of axioms.

The last two cases can be proved by induction on the syntactic
structure of $G$.

If $G = G_1 \wedge G_2$, where $G_1, G_2$ are universal
Horn--formulae, $(*)$ follows immediately from the induction
hypothesis.

Finally let be $G = \forall x . F$, $F$ a universal Horn--formula
and $AX \vdash \forall x . F$. Then $AX \vdash F[x := t]$ for each
ground term $t$. Since each $F[x := t]$ is again a universal
Horn--formula  the induction hypothesis implies
$\mathbb{T}(\Sigma,AX) \vDash F[x := t]$ for each ground term $t$,
which means $\mathbb{T}(\Sigma,AX) \vDash \forall x . F$. \hfill
qed.

\emph{This theorem implies, that we can use the same kind of
formulae to specify  data types by initiality constraints that are
used in logic programming, i.e. in PROLOG.}

\end{Paragraph}
\end{Section}
\begin{Section}[Title={Initial Specifications of Data Types}, Label={section3_2}]
\begin{Paragraph}

%............................................

\end{Paragraph}
\begin{Paragraph}[Title={Examples}, Label=Paragraph42]

\ns The first example is the specification of
\textcolor{blue}{natural numbers}. For this specification one has
to find the right constructors. Having the Peano axiomatization of
natural numbers in mind, it is not difficult to choose the
constant \textcolor{blue}{$zero$} and the unary successor function
\textcolor{blue}{$succ$} as constructors.

In case of \textcolor{green}{finite lists} over a parametric type
of list elements one takes the empty list \textcolor{green}{$nil$}
as a constant and \textcolor{green}{$cons$} as a binary
constructor.

The resulting \hyperlink{nat-casl}{\CASL specifications } are as
follows.

%............................................

\end{Paragraph}
\begin{Paragraph}[Title={Examples}, Label=Paragraph43]

Parametric data types will be discussed in more detail in the
section of Generic Specifications.

%............................................

\end{Paragraph}
\begin{Paragraph}[Title={Examples}, Label=Paragraph44]

Finding the right constructors is not always as simple as in the
previous examples. Let us consider \textcolor{blue}{finitely
branching trees with finite depth}.

The inductive structure of \textcolor{blue}{binary trees} is easy,
one takes a constant \textcolor{blue}{$zero$}, representing the
binary tree with exactly one node and with no subtrees, and a
binary constructor \textcolor{blue}{$2-cons$} which makes a new
binary tree out of two binary trees, by making the given trees to
the subtrees of the (new) root node.

The resulting \hyperlink{2-trees}{specification of binary trees}
is a straight forward generalization of the specification of
natural numbers.

%............................................

\end{Paragraph}
\begin{Paragraph}[Title={Examples}, Label=Paragraph45]

However, the inductive structure of binary trees does not help us
to find the inductive structure of finitely branching trees with
finite depth. In case of binary trees we have just one dimension,
the depth of a binary tree, which characterizes the inductive
structure.

In case of finitely branching trees with finite depth we have two
dimensions, the depth and the breadth. This means, we have to
generalize the constructor $2-cons$ to a constructor which accepts
an finite sequence of trees as argument, not just a pair as above.

This leads to the following \hyperlink{trees}{specification of
finitely branching trees with finite depth}.

%............................................

\end{Paragraph}
\begin{Paragraph}[Title={Examples}, Label=Paragraph46]
In the previous examples the specifications have not used any
axioms, i.e. any two constructor terms represent two different
data values of the corresponding data typ. This situation is not
always given. Let us consider the data type of
\textcolor{blue}{integers}.

A first possibility is the following
\hyperlink{integer}{specification } which uses the constant $zero$
and the two unary constructors $pred$ and $suc$.

%............................................

\end{Paragraph}
\begin{Paragraph}[Label=Paragraph47]

But, one could specify the data types of integers also by sticking
together two copies of natural numbers. This idea would lead to
the following \hyperlink{integer2}{specification of integers}.

This observation raises the question, in which sense  the two
specifications of integers specify the same data type. We will
come back to this question if we have discussed the functional
enrichment of data types.

\end{Paragraph}
\end{Section}
\begin{Section}[Title={Functional Enrichment of Data Types}, Label={section3_3}]
\begin{Paragraph}

%............................................

\end{Paragraph}
\begin{Paragraph}[Label=Paragraph48]
\small

The inductive definition of functions on natural numbers can be
extended to arbitrary initially specified data types.

Formally we take a specification of a data type and enrich this
specification by one or more new function symbols and new axioms
in such a way that \pause
\begin{List}[ListType=itemize]
    \ListItem each term over the extended set of function symbols can
      value preservingly be reduced by the added axioms to a term over the
    original set of function symbols (\textcolor{red}{no
    junk})\pause
    \ListItem the new axioms do not force terms over the original set
    of functions symbols to have equal values, if the terms have
    different values with respect to the original set of axioms
    (\textcolor{red}{no confusion}).
\end{List}

%............................................

\end{Paragraph}
\begin{Paragraph}[Label=Paragraph49]
\ns

 The following functional enrichment of natural numbers is an
example with junk:

\begin{SpecDefn}{Nat-with-junk}~= \item[\Free~\Group]
\begin{Items} \item[\Sort] \( Nat \)
\item[\Ops] \( 0zero : Nat;\)
\\          \( suc\_\_ : Nat \tfun Nat; \)
\\ \(  \_\_+\_\_ : Nat \times Nat \tfun Nat;\)
\item[\Axiom] \( \forall x : Nat \. x + zero = x ; \)~\EndGroup
\end{Items} \item[\End] \end{SpecDefn}

%............................................

\end{Paragraph}
\begin{Paragraph}[Label=Paragraph50]

Whereas the next specification illustrates confusion:

\begin{SpecDefn}{Nat-with-confusion}~= \item[\Free~\Group]
\begin{Items} \item[\Sort] \( Nat \)
\item[\Ops] \( zero : Nat;\)
\\          \( suc\_\_ : Nat \tfun Nat; \)
\\ \(  \_\_+\_\_ : Nat \times Nat \tfun Nat;\)
\item[\Axioms] \( \forall x : Nat \. x + zero = x ;\)
\\ \( \forall x,y : Nat \. x + suc~ y = suc( x+y ) ; \)
\\ \( \forall x: Nat \. x + x = zero; \) ~\EndGroup
\end{Items} \item[\End] \end{SpecDefn}

%............................................

\end{Paragraph}
\begin{Paragraph}[Label=Paragraph51]

A correct functional enrichment of natural numbers with addition
is given by

\begin{SpecDefn}{Nat-with-addition}~= \item[\Free~\Group]
\begin{Items} \item[\Sort] \( Nat \)
\item[\Ops] \( zero : Nat;\)
\\          \( suc\_\_ : Nat \tfun Nat; \)
\\ \(  \_\_+\_\_ : Nat \times Nat \tfun Nat;\)
\item[\Axioms] \( \forall x : Nat \. x + zero = x ;\)
\\ \( \forall x,y : Nat \. x + suc~ y = suc( x+y ) ; \)
 ~\EndGroup
\end{Items} \item[\End] \end{SpecDefn}

%............................................

\end{Paragraph}
\begin{Paragraph}[Label=Paragraph52]

An interesting exercise is the inductive definition of the depth
function on finitely branching trees with finite depth. The
definition of the function $depth : Tree \to Nat$ requires some
auxiliary functions like the maximum of a list of natural numbers,
since the depth of a tree is one plus the maximal depth of the
sons of the root node. The inductive definition of the maximum
function again requires an auxiliary function namely the
less--or--equal relation on natural numbers. At the end we obtain
a rather complex specification, which will be discusses in the
tutorial.

%............................................

\end{Paragraph}
\begin{Paragraph}[Title={Equivalence of Specifications}, Label=Paragraph53]
\small
Now we can come back to the two specifications of integers.

Generally one can say that \textcolor{blue}{two specifications
define the same data type if there is a specification which
extends the given specification conservatively.}

To make this precise we have to define the concept of conservative
extensions of specifications.

%............................................

\end{Paragraph}
\begin{Paragraph}[Title={Equivalence of Specifications}, Label=Paragraph54]

\begin{Definition}[Title = {Initial Algebras as Data Types}, Label = {Definition1}]
\begin{List}[ListType=enumeration]
    \ListItem An ordered pair $Spec = (\Sigma, AX)$, consisting of a
signature $\Sigma$ and a set $AX$ formulae is called a
\textbf{specification}. A $\Sigma$--structur $ \mathbb{A}$ is
called a $Spec$--model if it satisfies each axiom out of $AX$.
\pause
 \ListItem A signature $\Sigma_1 = ( Sorts_1, Ops_1, Pred_1)$ is
a \textbf{sub--signature} of a signature $\Sigma_2 = ( Sorts_2,
Ops_2, Pred_2)$ if the corresponding components are subsets
compatible with the typing of the operations and predicates. For
each $\Sigma_2$--structure $ \mathbb{A}$ the \textbf{restriction}
of $ \mathbb{A}$ to the sub--signature $\Sigma_1$ is denoted by $
\mathbb{A}|_{\Sigma_{1}}$
 \pause
  \ListItem $Spec_1 = (\Sigma_1, AX_1)$
is a \textbf{sub--specification} of $Spec_2 = (\Sigma_2, AX_2)$ if
$\sigma_1$ is a sub--signature of $\Sigma_2$ and $AX_2 \models F $
for each $F \in  AX_1$.
 \pause
 \ListItem  $Spec_2 = (\Sigma_2, AX_2)$ is a \textbf{conservative} extension
 of $Spec_1 = (\Sigma_1, AX_1)$ if $Spec_1$ is a
 sub--specification of $Spec_2$ and for each $Spec_2$--model
 $\mathbb{A}$ the restriction $\mathbb{A}|_{\Sigma_{1}}$ is a
 $Spec_1$--model.
\end{List}
\end{Definition}
\pause
 \vspace{2cm}
 The following specification  is a conservative extension of the
two previously given specification of integers:

%............................................

\end{Paragraph}
\begin{Paragraph}[Label=Paragraph55]
\footnotesize


\begin{SpecDefn}{Integer\_3}~= \item[\Free~\Group]
\begin{Items} \item[\Sort] \( Nat, Int; \)
\item[\Ops] \( zero : Nat; \)
\\ \( 0: Int;\)
\\ \( suc : Nat \tfun Nat; \)
\\ \( suc, pred : Int \tfun Int;\)
\\ \( pos, neg : Nat \tfun Int;\)
\item[\Axioms] \( \forall x : Int \. pred( suc (x)) = x ; \)
\\ \( \forall x : Int \. suc (pred (x)) = x ; \)
\\  \( pos(zero) = neg(zero) = 0 ;\)
\\ \( \forall x : Nat \. pred (neg (x)) = neg (suc(x)) ; \)
 \\ \( \forall x : Nat \. pred(pos (suc (x))) = pos(x) ; \)
\\ \( \forall x : Nat \. suc(neg (suc (x))) = neg (x) ; \)
\\ \( \forall x : Nat \. suc(pos (x)) = pos (suc(x)) ; \)
\\ \( \forall x : Nat \. neg ( sux (x)) = pred (neg(x)) ; \)
\\ \( \forall x : Nat \. pos( suc (x)) = suc ( pos(x)) ; \)
~\EndGroup
\end{Items} \item[\End] \end{SpecDefn}

%............................................

\end{Paragraph}
\begin{Paragraph}[Title={Equivalence of Specifications}, Label=Paragraph56]
\ns

 Intuitively the equivalence of two initial specifications of
one and the same data type means: \\ \pause \textcolor{blue}{the
constructors of one specification can inductively be defined by
the constructors of the other specification and vice versa.}

\end{Paragraph}
\end{Section}
\begin{Section}[Title={Generic Specifications}, Label={section3_4}]
\begin{Paragraph}

%............................................

\end{Paragraph}
\begin{Paragraph}[Label=Paragraph57]
\small
In many cases data types are parametric like lists, decorated
trees, sets, and multisets.\pause

Semantically a parametric data type can not be represented by an
isomorphism class. \pause

Let us discuss the situation for the generic data typ of
\emph{finite lists}

\(\Sigma_{Lists}\) = \begin{BasicSpec} \I\Sorts \( Elem, Lists;\)
\I\Ops \( nil : Lists; \)
\\ \( cons : Elem \* Lists \tfun Lists \)
\end{BasicSpec}

%............................................

\end{Paragraph}
\begin{Paragraph}[Label=Paragraph58]
The generic data type of finite list would be represented by the
 class of  those $\Sigma_{Lists}$--structures $\mathbb{A}$
such that up to isomorphisms $ A_{Lists} = (A_{Elem})^*,
nil_{\mathbb A} = []$, $ cons_{\mathbb A} (x, [x_1, \ldots , x_n])
= [x,x_1, \ldots,x ],$ i.e., only the interpretation of the sort
$Elem$ is not constraint. \pause

The constraint part satisfies again an initiality condition:\\
\pause A $\sigma_{Lists}$--structure $\mathbb L$ is an model of
the described class of models if for each
$\Sigma_{Lists}$--structure $\mathbb A$ with $L_{Elem} = A_{Elem}$
there exists exactly one homomorphism $\varphi : {\mathbb L} \to
{\mathbb A}$ with $\varphi_{Elem} = Id_{(L_{Elem})} (=
Id_{(A_{Elem})}$.

%............................................

\end{Paragraph}
\begin{Paragraph}[Label=Paragraph59]
\small
\begin{Definition}[Title = {Initial Algebras as Data Types}, Label = {Definition2}]
If $Spec_1$ is a sub--specification of $Spec_2$ then
\textcolor{red}{ $Spec_1$ \textbf{then free} $Spec_2$} is called
an \textcolor{red}{\textbf{initial constraint} in $Spec_2$}.

\pause A $Specc_2$--model $\mathbb{A}$ satisfies the initial
constraint, written $ Spec_1 \; \mathbf{then \; free} \; Spec_2 \;
\models \mathbb{A} $ if for each $Spec_2$--model $ \mathbb{B}$
with $ \mathbb{A}|_{Spec_1} = \mathbb{B}|_{Spec_1}$ there is
exactly one $\Sigma_2$--homomorphism $\varphi  : \mathbb{A} \to
\mathbb{B}$ which extends the identity , i.e. $\varphi_s =
Id_{A_s}$ for each $s \in Sorts(\Sigma_1)$.

\pause If one abstains from the uniqueness of the homomorphism
$\varphi^* : \mathbb{A} \to \mathbb{B}$ then one obtains a
\textcolor{red}{\textbf{generating constraint} in $Spec_2$},
written $ Spec_1 \; \mathbf{then \; generated} \; Spec_2 \;
\models \mathbb{A}. $
\end{Definition}

%............................................

\end{Paragraph}
\begin{Paragraph}[Label=Paragraph60]
The case of an initial constraint $Spec_1 \; \mathbf{then \; free}
\; Spec_2 $ with $Spec_1 ( \emptyset, \emptyset)$ corresponds to
the specification of the class of initial $Spec_2$--models,
discussed in the previous section.\\ \pause The example of list is
given by \pause  \footnotesize
\begin{SpecDefn}{List\_1}~= \I\Sort \( Elem;
\) \item[\Then] \item[\Free~\Group]
\begin{Items}
\I\Sort \( Lists; \) \I\Ops \( nil : Lists; \)
\\ \( cons : Elem \* Lists \tfun Lists \)
 ~\EndGroup \end{Items} \I \End
\end{SpecDefn} \small
In this example the specification $Spec_1$ consists just of the
sort name $Elem$, i.e. the sets of operation names, of predicate
names, and of axioms are empty.

%............................................

\end{Paragraph}
\begin{Paragraph}[Label=Paragraph61]

The previously given specification
\begin{SpecDefn}{Lists}[\Sort \hspace{1pt} \(Elem\)] ~=
 \item[\Free~\Group]
\begin{Items}
\I\Sort \(Lists;\) \I\Ops \(nil : Lists; \)
\\ \( cons : Elem \* Lists \tfun Lists \)
 ~\EndGroup \end{Items}  \item[\End]
\end{SpecDefn}
is another syntactic representation of the same data type. This
version make the parameter explicit and allows the substitution of
the formal parameters by actual ones.

%............................................

\end{Paragraph}
\begin{Paragraph}[Title={Ordered Lists}, Label=Paragraph62]

A parameter of a generic specification can be an arbitrary
specification. A typical situation where the parameter part of a
generic specification is a nontrivial specification is the case of
\textcolor{blue}{ordered lists}. \footnotesize
\begin{SpecDefn}{Linear\_Order} ~=
\I\Sort \(Elem \) \I\Pred \( \_\_ \leq \_\_ : Elem \* Elem;\)
\I\Vars \( x,y,z:  Elem \) \I\Axioms~
\\ \( x \leq x; \)
\\ \( ( x \leq y \wedge y \leq x ) \imp x = y; \)
\\ \( (x \leq y \wedge y \leq z ) \imp x \leq z; \)
\\ \( x \leq y \vee y \leq x ;\)
\I\End \end{SpecDefn}


%............................................

\end{Paragraph}
\begin{Paragraph}[Label=Paragraph63]
\small \hypertarget{orderdlists}{}
\begin{SpecDefn}{Ordered\_Lists}[\SpecName{Linear\_Order}] ~=
 \item[\Free~\Group]
\begin{Items}
\I\Sort \( Lists; \) \I\Ops \( nil : Lists; \)
\\ \( cons : Elem \* Lists \tfun Lists; \)
\I\Pred \( ordered : Lists; \)
 \I\Vars \(x,y : Elem, l: Lists; \)
 \I\Axioms \( ordered (nil); \)
 \\ \(ordered(cons(x,nil)); \)
 \\ \( (x \leq y \wedge x \not= y \wedge ordered(cons(y,l))) \imp
 ordered(cons(x, cons(y,l))); \)
 ~\EndGroup \end{Items}
\item[\Then] \item[\Sort ]\( Ordered = \{ l : Lists \.
ordered(l)\}; \)
  \item[\End] \end{SpecDefn}

%............................................

\end{Paragraph}
\begin{Paragraph}[Title={Predicative Subtyping}, Label=Paragraph64]

There are two new topics in this specification:\pause
\begin{List}[ListType=enumeration]
    \ListItem the use of a predicate within an initiality
    constraint;\pause
    \ListItem defining a subtype by a predicate.
\end{List}\pause

A predicate introduced within an initiality constraint is
interpreted minimally in a model of that specification.\pause

The second point is just a very useful feature of the \CASL
language which allows to name a subset defined by a predicate.

%............................................

\end{Paragraph}
\begin{Paragraph}[Title={Decorated Trees}, Label=Paragraph65]

 Generic Specifications with several parameters can also be used.
 In that case parameters can individually be instantiated. An
 example is \textcolor{blue}{ trees with different types of labels
 at leafs and internal nodes}.\pause

 \begin{SpecDefn}{Trees}[\Sort  \(Elem1 \)][\Sort  \(Elem2 \)] ~=
 \item[\Free~\Group]
\begin{Items}
\I\Sorts \(Trees < Treeseq;\) \I\Ops \( leaf : Elem1 \tfun
Trees;\) \\ \( h\_cons : Trees \* Treeseq \tfun Treeseq; \)
\\ \( v\_cons : Elem2 \* Treeseq \tfun Trees \)
  ~\EndGroup \end{Items}  \item[\End]
\end{SpecDefn}

%............................................

\end{Paragraph}
\begin{Paragraph}[Title={Sets}, Label=Paragraph66]

The next generic specification describes the \textcolor{blue}{set
of all finite subsets} of a given parameter set: \pause

\begin{SpecDefn}{Sets}[\Sort \hspace{1pt} \(Elem\)] ~=
 \item[\Free~\Group]
\begin{Items}
\I\Sort \(Sets;\) \I\Ops \(empty: Sets; \)
\\ \( add : Elem\* Sets \tfun Sets; \)
\I\Axioms \( \forall x : Elem; s : Sets \. add(x, add(x,s)) =
add(x,s); \)
\\ \( \forall x,y :Elem; s : Sets \. add(x, add(y,s)) = add(y,
add(x,s))\)
 ~\EndGroup \end{Items}  \item[\End]
\end{SpecDefn}

%..........................................

\end{Paragraph}
 \begin{Paragraph}[Title={functional ennrichment of sets}, Label=Paragraph67]

 What about the definition of the
 \textcolor{blue}{membership relation}?

 \begin{SpecDefn}{Membership} ~= \SpecName{Sets}
 \I\Then
 \I{\Free~\Type}~
  \( Boole ::= true \| false; \)
\I\Then \item[\Group] \begin{Items} \I\Ops \( \_\_ \in \_\_ : Elem
* Sets \tfun Bool; \) \I\Vars \( x, y : Elem; s : Sets ;\)
\I\Axioms~
\\ \( x \in empty = false ; \)
\\ \( x \in add(x,s) = true ; \)
\\ \( \neg x = y  \imp x \in add(y,s) = x \in s ;\)
~\EndGroup \end{Items}
\end{SpecDefn}

%............................................

\end{Paragraph}
 \begin{Paragraph}[Title={functional ennrichment of sets}, Label=Paragraph68]

 One should be aware, that the previous functional enrichment is
 not an inductive definition of the membership relation since it
 uses a negated equality as premise of an implication. For those
 kinds of axioms the existence of initial models is not
 guaranteed. An inductive definition of the membership relation
 will be discussed in the tutorial.


%............................................

\end{Paragraph}
\begin{Paragraph}[Title={Parameter instantiation}, Label=Paragraph69]

\begin{SpecDefn} {List\_of\_Interger} ~= \textsc{List} [\textsc{Integer}
 \Fit \( Elem \mapsto Int \)] \End \end{SpecDefn}

\begin{SpecDefn} {Sets\_of\_Interger} ~= \textsc{Sets} [\textsc{Integer}
 \Fit \( Elem \mapsto Int \)] \End \end{SpecDefn}

Generic specification with several parameters can be partially
instantiated:
\begin{SpecDefn}{Trees\_with\_Integer} ~=
\textsc{Trees}[\textsc{Integer} \Fit \(Elem1 \mapsto Int\)] \End
\end{SpecDefn}

Whereas the following specification instantiates both parameters
\begin{SpecDefn}{Trees\_with\_Integers\_and\_Naturals} ~=
\textsc{Trees}[\textsc{Integer} \Fit \(Elem1 \mapsto
Int\)][\textsc{Nat\_0} \Fit \(Elem2 \mapsto Nat\)] \End
\end{SpecDefn}

%...........................................................

\end{Paragraph}
\end{Section}
\begin{Section}[Title={Partial Constructors}, Label={section3_5}]
\begin{Paragraph}

\ns

 In some applications it may happen that some functions are not
totally defined. To see a simple example for partial functions,
consider the construction of the set of finite, connected paths
over a directed graph. It makes only sense to append an edge to a
path if the range node of the edge equals the domain node of the
path. Another example is given by \textcolor{blue}{powerlists},
introduced by Jayadev Misra as a data type for parallel recursion.

%........................................................

\end{Paragraph}
\begin{Paragraph}[Title={basic notions}, Label=Paragraph70]

A partial function will be represented by $f : A \pfun B$ and the
domain is denoted by $def \; f $ which is formally a unary
predicate on $A$.

If one extends the notion of a $\Sigma$--structure  $ \mathbb{A}$
in such a way that $\sigma_{ \mathbb{A}}: A_w \pfun A_s$ may be a
partial function, then one obtains the notion of \textbf{partial
$\Sigma$--structures}.



For \textbf{partial $\Sigma$--structures} the notion of a
homomorphism has to be generalized.

\end{Paragraph}
\begin{Paragraph}[Label=Paragraph71]

A \textbf{homomorphism between partial $\Sigma$--structures}
$\varphi : \mathbb{A}  \to \mathbb{B}$ is given by a family
$\varphi = (\varphi_s : A_s \to B_s \;|\; s \in Sorts_{\Sigma})$
of mappings, such that  for each $\sigma : w \to s$, each $\varrho
: w$ and each $(a_1, \ldots , a_n) \in A_w$ the following
homomorphism conditions are satisfied: \small $$\begin{array}{l}
 \mbox{if
} a_1, \ldots , a_n) \in dom(\sigma_{\mathbb A}) \mbox{ then }
(h_{s_1}(a_1), \ldots , h_{s_n}(a_n)) \in dom(\sigma_{\mathbb B})
\mbox{ and}\\
 ( h_s(\sigma_{\mathbb A}(a_1, \ldots ,a_n)) =
\sigma_{\mathbb
 B}(h_{s_1}(a_1), \ldots , h_{s_n}(a_n)), \\
  (a_1, \ldots a_n) \in \varrho_{\mathbb A} \quad \Rightarrow \quad
 (h_{s_1}(a_1), \ldots , h_{s_n}(a_n)) \in \varrho_{\mathbb B}
 \end{array}$$

\end{Paragraph}
 \begin{Paragraph}[Label=Paragraph72]
\ns
 By this definition homomorphisms between partial
$\Sigma$--structures have to preserve definedness, but do not
necessarily  reflect it. Therefore, an initial, partial
$\Sigma$--structure has minimal domains for its partial
operations. If there are no axioms which imply the applicability
of partial operations to some arguments, the domains of the
partial operations in initial $\Sigma$--structures are empty. But,
in that case a partial constructor would generate nothing and
would be useless.

\end{Paragraph}
\begin{Paragraph}[Label=Paragraph73]

In order to state applicability of partial operations by axioms,
we associate with each partial operation $\sigma$ a predicate $def
\sigma$.

Another way to state the applicability of partial operations is
the use of term equations. First we have to define the validity of
term equations for partial $\Sigma$--structures. There are two
versions of validity, \textbf{strong equality} and
\textbf{existential equality}.

%...............................................................

\end{Paragraph}
\begin{Paragraph}[Title={strong and existential equalities}, Label=Paragraph74]
\small
\begin{Definition}[Title = {Initial Algebras as Data Types}, Label = {Definition3}]
Let be $\Sigma$ a signature which may contain partial operations,
$X$  a finite set of typed variables, and $t_1, t_2 \in T(\Sigma,
X)_s $ for some sort $s \in Sorts_{\Sigma}$. Then $X . t_1 = t_2$
denotes a \emph{strong equality} and $X . t_1 \eeq t_2$ denotes an
\emph{existential equality}.


In order to define the notion of a solution of a term equation in
the presence of partial operations, we define formally whether
\textbf{a term $t \in T(\Sigma, X)_s$ defines a value (denoted
$\ekl t \ekr_{ \mathbb{A}, \varphi}$) in a partial
$\Sigma$--structure $ \mathbb{A}$} for a given valuation $\varphi
: X \to A$:
\begin{List}[ListType=enumeration]
  \ListItem if $t= x \in X_s, s \in Sorts_{\Sigma}$ then  $\ekl t \ekr_{
  \mathbb{A}, \varphi}$ is defined and its value is given by $$\ekl t
  \ekr_{\mathbb{A}, \varphi} = \varphi(x).$$ \pause
  \ListItem if $t = \sigma (t_1, \ldots , t_n)$ then $\ekl \sigma (t_1, \ldots ,
  t_n) \ekr_{ \mathbb{A}, \varphi}$ is defined if $\ekl t_i \ekr_{\mathbb{A},
  \varphi}$ is defined for $i = 1, \ldots , n$ and $(\ekl t_1
  \ekr_{\mathbb{A}, \varphi}, \ldots, \ekl t_n \ekr_{\mathbb{A},
  \varphi}) \in dom(\sigma_{ \mathbb{A}})$. \\ In that case the value
  is given by $$\ekl \sigma (t_1, \ldots ,
  t_n) \ekr_{ \mathbb{A}, \varphi} = \sigma_{ \mathbb{A}}(\ekl t_1
  \ekr_{\mathbb{A}, \varphi}, \ldots, \ekl t_n \ekr_{\mathbb{A},
  \varphi}).$$
\end{List}

For any partial $\Sigma$--structure $ \mathbb{A}$ \textbf{a
valuation $\varphi : X \to A$ is a solution of the strong equality
$X . t_1 = t_2$ in $ \mathbb{A}$}, if either, both $\ekl t_1
\ekr_{\mathbb{A},\varphi} $ and $\ekl t_2
\ekr_{\mathbb{A},\varphi} $ are not defined, or both are defined
and evaluate to the same value.

For any partial $\Sigma$--structure $ \mathbb{A}$ \textbf{a
valuation $\varphi : X \to A$ is a solution of the existential
equality $X . t_1 \eeq t_2$ in $ \mathbb{A}$}, if  both $\ekl t_1
\ekr_{\mathbb{A},\varphi} $ and $\ekl t_2
\ekr_{\mathbb{A},\varphi} $  are defined and evaluate to the same
value.\\
 Finally $ \mathbb{A} \models X . t_1 = t_2$, if each valuation
 $\varphi X \to A$ is a solution of $X . t_1 = t_2$, and
$ \mathbb{A} \models X . t_1 \eeq t_2$, if each valuation
 $\varphi X \to A$ is a solution of $X . t_1 \eeq_2$.

 If $\varrho \in Pred_{\Sigma}$ then $\varrho(t_1,
 \ldots , t_m)_{ \mathbb{A}, \varphi} = true$ if $\ekl t_i  \ekr_{\mathbb{A},
 \varphi}$ is defined for each $i = 1, \ldots , m$ and if $(\ekl t_1
 \ekr_{\mathbb{A}, \varphi}, \ldots , \ekl t_m
 \ekr_{\mathbb{A}, \varphi}) \in \varrho_{ \mathbb{A}}$.
\end{Definition}


\newpage{}
\ns

 This definition implies, that for $t, t' \in T(\Sigma, X)$ and
each valuation $\varphi : X \to A$ it holds that $\ekl t'
\ekr_{\mathbb{A}, \varphi})$ is defined, if $t'$ is a subterm of
$t$ and $\ekl t \ekr_{\mathbb{A}, \varphi})$ itself is defined. A
predicate assigns the truth value $false$ if it is applied to (at
least one) undefined argument.


It is worth to notice, that existential equality alone would be
sufficient. But since strong equality has some popularity too,
CASL offers both of them. For instance the following logical
equivalence is always true: $$ def (\sigma (x_1, \ldots , x_n))
\quad  \Leftrightarrow \quad \sigma (x_1, \ldots , x_n) \eeq
\sigma (x_1, \ldots , x_n).$$

\newpage{}

Let's illustrate the use of partial operations in specifications.


\begin{SpecDefn}{Free\_Sets\_with\_P\_choose} ~=
\SpecName{Sets}[\Sort~\(Elem\)] \item[\Then]\item[\Free~\Group]
\begin{Items}
\item[\Pred] \( \_\_ is\_in \_\_ : Elem \* Sets; \) \item[\Axioms]
\( \forall x : Elem \. x~ is\_in~ empty = false;\)
\\ \(\forall x,y : Elem, s : Sets \. x~ is\_in~ s \Rightarrow x~
is\_in~ add(y,s);\)
\\ \( \forall x : Elem, s : Sets \. x~ is\_in~ add(x,s);\)
 ~\EndGroup \end{Items}
 \item[\Then]
\item[\Op] \(choose : Sets \pfun Elem;\) \item[\End]
\end{SpecDefn}

\newpage{}
\small
In this example $choose(s)~ is\_in ~s$ equals to $false$, since
$choose(s)$ is a term representing an undefined value, for short
an undefined term.

The equation $$ add( choose(s),s) = add(choose(s), add(y,s))$$ is
true, since both sides are undefined terms, whereas the
existential equation $$add( choose(s),s) \eeq add(choose(s),
add(y,s))$$ does not hold.

Asserting $\forall s : Sets \; . \;choose(s)\; is\_in \; s$ as an
axiom implies that $choose(s)$ is defined, for any $s$. The same
effect is caused if $\forall s : Sets \; .\; add( choose(s),s) =
s$ would be added as an axiom, since the right hand side of the
strong equality always denotes a defined value.

\newpage{}
\ns

 Let us now specify the motivating example from the beginning
of this section.


\begin{SpecDefn}{Directed\_Graph} ~=
\I\Sorts \(Nodes, Edges;\) \I\Ops \( from : Edges \tfun Nodes ;\)
\\ \( to : Edge \tfun Nodes; \) \I\End \end{SpecDefn}

\newpage{}
\small
\begin{SpecDefn}{Path\_over\_Graphs}[\SpecName{Directed\_Graph}] ~=
\item[\Then] \item[\Free~\Group]
\begin{Items}
\I\Sort \( Path;\) \I\Ops \( nil : Nodes \tfun Path;\)
\\ \(app : Edges \* Path \pfun Path;\)
\\ \( from, to : Path \tfun Nodes;\)
\I\Axiom \(\[\forall x : Edges, p : Path \. app(x,p) \eeq
 append(x,p) \Implies
 \\\M to(app(x,p)) = to(p) \wedge
 from(app(x,p)) = from(x) \wedge
 \\\M to(x) = from(p);\]\)
 \\ \( \forall x :Edge, p : Path \. to(x) = from(p) \Rightarrow
 app(x,p) \eeq app(x,p);\)
  ~\EndGroup \end{Items} \item[\End] \end{SpecDefn}


\newpage{}
\ns

Good exercises for the tutorial are the specifications of
functional enrichments  of  concatenating  finite paths as a new
partial operation and the specification of the length function.


\end{Paragraph}
\begin{Paragraph}[Label=Paragraph77]

The next example is the specification of powerlists: \small
\begin{SpecDefn}{PowerLists}[\Sort~\(Elem\)] ~=
\SpecName{Nat-with-addition} \item[\Then] \item[\Free~\Group]
\begin{Items}
\I\Sort \( Powerlist ; \) \I\Ops \( \{\_\} : Elem \tfun
Powerlist;\)
 \\ \( \_\_|\_\_ : Powerlist \* Powerlist \pfun Powerlist; \)
 \\ \( length : Powerlist \tfun Nat; \)
 \I\Vars \( x : Elem; l, r : Powerlist; \)
 \I\Axioms~
 \\ \( length( \{x\}) = suc(Zero);\)
 \\ \( def (l | r) \Implies length( l | r ) = length(l ) +
 length(r); \)
 \\ \( length(l) = length(r) \Implies def (l | r) ;\)
 ~\EndGroup \end{Items} \item[\End] \end{SpecDefn}

\end{Paragraph}
\begin{Paragraph}[Title={$WHILE$--loop}, Label=Paragraph78]
\ns

 As next example we study the specification of the
$WHILE$--loop. This example is of interest since without partial
operations this important construction could not be specified. On
the other hand, this example can be used to prove, that each
computable function on natural numbers can be specified as a
functional enrichment of the data type \textsc{Nat}.



\end{Paragraph}
\begin{Paragraph}[Label=Paragraph79]
\small
The first version uses strong equality:

\begin{SpecDefn}{While\_with\_strong\_equality} ~=
\SpecName{Boole} \I \Then
 \I\Sort \(M; \)
 \I\Op \( p : M \tfun Boole ;\)
 \I\Ops \( f,g : M \pfun M;\) \I\Then
 \I{\Free~\Group} \begin{Items}
 \I\Op \( while : M \pfun M; \)
 \I\Var \( x : M; \)
 \I\Axioms \(  p(x) = false \imp while(x ) = f(x); \)
 \\ \(p(x) = true \imp while(x) = while(g(x));\) \EndGroup
 \end{Items}
 \I\End
 \end{SpecDefn}

\end{Paragraph}
\begin{Paragraph}[Label=Paragraph80]
The second version uses existential equality:

\begin{SpecDefn}{While\_with\_existential\_equality} ~= \textsc{Boole}
\I \Then
 \I\Sort \(M; \)
 \I\Op \( p : M \tfun Boole ;\)
 \I\Ops \( f,g : M \pfun M;\) \I \Then
 \I{\Free~\Group} \begin{Items}
 \I\Op \( while : M \pfun M; \)
 \I\Var \( x : M; \)
 \I\Axioms \(  p(x) = false \wedge f(x) \eeq f(x)
 \imp while(x ) \eeq f(x); \)
 \\ \(p(x) = true \wedge while(g(x)) \eeq while(g(x))
 \imp \\\M while(x) \eeq while(g(x));\) \EndGroup
 \end{Items}
 \I\End
 \end{SpecDefn}

\end{Paragraph}
\end{Section}
\begin{Section}[Title={Data Refinement}, Label={section3_6}]
\begin{Paragraph}
\small
 Axiomatic specifications of data types are aimed at minimizing
the conceptual distance between an application area and the first
formal model. As a consequence, the conceptual distance between
the first formal model and the implementation level can be rather
big. In order to bridge the conceptual distance between the data
types used in the specification and the data types available at
the implementation level, it may be useful to introduce
intermediate levels of abstraction and to design an implementation
of the abstract specification by \textbf{stepwise refinements}.


\end{Paragraph}
\begin{Paragraph}[Label=Paragraph81]

Implementations of abstract specifications by means of stepwise
refinements leads to  the concept of
\textcolor{blue}{implementations of an abstract data type by
another one}, short \textcolor{blue}{\textbf{abstract
implementations}}.\pause \vspace{8mm}

We illustrate this concept, by implementing the generic data type
of sets by the data type of ordered list.\pause \vspace{8mm}

Since this generic implementation requires not just elements but
elements with a linear order, we modify the generic data type of
set accordingly:

\end{Paragraph}
\begin{Paragraph}[Label=Paragraph82]
\begin{SpecDefn}{Sets2}[\SpecName{Linear\_Order}] ~=
\item[\Then] \item[\Free~\Group]
\begin{Items}
\I\Sort \(Sets; \) \I\Ops \( empty : Sets; \)
\\ \( add : Elem \* Sets \tfun Sets; \)
\I\Vars \( x,y: Elem, s: Sets\)
\\ \( add(x, add(x,s)) = add(x,s);\)
\\ \( add(x, add(y,s)) = add(y, add(x,s));\)
 ~\EndGroup \end{Items} \item[\End] \end{SpecDefn}

\end{Paragraph}
 \begin{Paragraph}[Label=Paragraph83]

 Next let us recall the generic specification of
 \hyperlink{orderdlists}{ordered lists}.\pause \vspace{8mm}

 We can use the sort $Orderd$ in that specification to implement
 the sort $Sets$ in \textsc{Sets2}. \pause \vspace{8mm}

  But, which operation in the
 specification of ordered list can be used to implement $add :
 Elem \times Sets \to Sets$? \pause \vspace{8mm}

 Since there is no appropriate
 operation, we have to define such an operation by a functional
 enrichment of \SpecName{Ordered\_Lists}.

\end{Paragraph}
 \begin{Paragraph}[Label=Paragraph84]
 \small
\begin{SpecDefn}{Orderd\_Lists\_2}[\SpecName{Linear\_Order}] ~=
\SpecName{Ordered\_Lists} \item[\Then] \Group \begin{Items}\I\Ops
\( add\_ol : Elem \* Ordered \tfun Ordered; \) \I\Vars \(x,y :
Elem, l:
Lists\) \I\Axioms \( add\_ol(x,nil) = cons(x,l);\) \\
 \(\[ (x \leq y \wedge x \not= y \wedge
ordered(cons(y,l))) \imp \\\M
 add\_ol (x, cons(y,l)) = cons(x, cons(y,l)); \]\)
 \\ \( (x = y \wedge ordered(cons(y,l))) \imp
 \\\M add\_ol(x, cons(y,l)) = cons(y,l); \)
 \\ \( \[ (y \leq x \wedge x \not= y \wedge ordered(cons(y,l)))
  \imp \\\M add\_ol(x, cons(y,l)) = cons(y, add\_ol(x,l)); \]\)
~\EndGroup \end{Items} \item[\End]
\end{SpecDefn}

\end{Paragraph}
 \begin{Paragraph}[Label=Paragraph85]

 The intended generic implementation can now be specified by a
so--called \textcolor{blue}{\textbf{view}}:\pause

\begin{SpecDefn}[\View]{Sets\_as\_Ordered\_lists} :
\SpecName{Sets2} \To \hspace{1mm}\SpecName{Ordered\_Lists\_2} ~= \\
\( Sets \mapsto Ordered, empty \mapsto nil, add \mapsto add\_ol\)
\End \end{SpecDefn}\pause

Since the generic parameter in both specifications is not
implicated in the implementation the view does not say anything
about the sort and predicate of the generic parameter.

\end{Paragraph}
 \begin{Paragraph}[Label=Paragraph86]

 However, the view above raises some \textcolor{blue}{proof obligations}:
 \pause \\ one has to
prove that the axioms of the domain specifications are mapped by
the view to logical consequences of the axioms in the range
specification, \pause \\ i.e. one has to show that the universally
quantified equations
$$\begin{array}{l}
 add\_ol(x, add\_ol(x,l)) = add\_ol(x,l) \\
 add\_ol(x, add\_ol(y,l)) = add\_ol(y, add\_ol(x,l)) \end{array}$$
are logical consequences of the axioms in
\textsc{Ordered\_Lists}.\pause \\ For the first equation this is
easy to see. The proof for the second equations takes some more
work.


\end{Paragraph}
\end{Section}
\end{Section}
\begin{Section}[Title={Final Coalgebras as Process Types}, Label={section4}]

In the introduction we have seen, that finality constraints can be
used to specify types of behavior which may also be called process
types. The first step is the formal definition of finality
constraints.

\begin{Section}[Title={Final Constraints}, Label={section4_1}]
\begin{Paragraph}

Dual to initiality constraints we want to restrict model classes
in such a way that for some parts of the specification the models
are final.\pause

\begin{Definition}[Title = {Final Coalgebras as Process Types}, Label = {Definition4}]
Provided $Spec_1 = (\Sigma_1, AX_1) \sqsubseteq Spec_2 =
(\Sigma_2, AX_2),$ then
$$Spec_1 \; \mathbf{then \;cofree} \; Spec_2 $$ is called an
\textbf{final constraint} in $Spec_2$.
\end{Definition}

\end{Paragraph}
\begin{Paragraph}[Title={Final Constraints}, Label=Paragraph87]

A $Spec_2$--model $ \mathbb{A}$ satisfies the final constraint,
written $$ \mathbb{A} \; \models \; Spec_1 \; \mathbf{then \;
cofree} \; Spec_2
$$ if for each $Spec_2$--model $ \mathbb{B}$ with
$\mathbb{B}|_{Spec_1} = \mathbb{A}|_{Spec_1}$ there is exactly one
$\Sigma_2$--homomorphism $\varphi : \mathbb{A} \to \mathbb{B}$
which extends the identity, i.e. $\varphi_s = Id_{A_s}$ for each
$s \in Sorts(\Sigma_1)$. \\ \pause If the existence condition is
given up and only the uniqueness is required, one gets a
\textbf{cogenerated constraint}, written
$$Spec_1 \; \mathbf{then \;cogenerated} \; Spec_2 $$


\end{Paragraph}
\begin{Paragraph}[Title={Final Constraints}, Label=Paragraph88]

If $Spec_1$ is the empty specification we will write just
$\mathbf{cofree} \; Spec_2$ respectively $\mathbf{cogenerated} \;
Spec_2$.

Let us first check some examples ( that turn out to be of no
interest):
\begin{SpecDefn}{Final-1} ~=
\item[\Cofree~\Group] \begin{Items} \item[\Sort] \( Elem \)
~\EndGroup
 \end{Items} \End \end{SpecDefn}\pause

The final constraint in this specification constrains the
interpretation of the sort $Elem$ to  singleton sets.

\end{Paragraph}
\begin{Paragraph}[Title={Final Constraints}, Label=Paragraph89]

What about the following specification:\pause

\begin{SpecDefn}{Final-2}~= \item[\Cofree~\Group]
\begin{Items} \item[\Sort] \( Nat \)
\item[\Ops] \( 0 : Nat;\)
\\          \( suc\_\_ : Nat \tfun Nat \)~\EndGroup
\end{Items} \item[\End] \end{SpecDefn}\pause

This specification specifies also the class of singleton sets,
additionally equipped with the unique interpretations of $0$ and
$suc$.

\end{Paragraph}
\begin{Paragraph}[Title={Final Constraints}, Label=Paragraph90]


 The collapsing to a singleton set ist caused by
the fact that there is no observation specified which has to
result in at least two different results, i.e. the result of any
experiment, if there is any,  is always the same. This means that
any two elements behaves in the same way an will therefore be
identified in the final model.

Final and cogenerated constraints restrict the model classes to so
called \textcolor{blue}{most abstract models} which identify any
two elements that are \textcolor{blue}{indistinguishable}.

\end{Paragraph}
\begin{Paragraph}[Label=Paragraph91]

What about the weaker requirement of a cogenerated constraint? It
turns out, that in both cases the specified model classes
coincide. We will come back to the different of cofree and
cogenerated constraints later.

\end{Paragraph}
\begin{Paragraph}[Label=Paragraph92]

Let us consider the next example:\pause
%\begin{Example}
\begin{SpecDefn}{Final-3}~= \SpecName{Boole}\\ \Then
\item[\Cofree~\Group]
\begin{Items} \item[\Sort] \( Elem \)
\item[\Op] \( observe : Elem \tfun Boole \)~\EndGroup
\end{Items} \item[\End] \end{SpecDefn}
%\end{Example}
\pause
In this example the states can only be observed, but there is no
possibility to change the state, i.e., there is no dynamics.
Therefore the interpretation $ \mathbb{F}$ with $
\mathbb{F}_{Elem} = \{ true, false \}$ and $observe_{ \mathbb{F}}
= Id_{Boole}$ forms an interpretation which satisfies the
specified final constraint.

\end{Paragraph}
\end{Section}
\begin{Section}[Title={Examples of Process Types}, Label={section4_2}]
\begin{Paragraph}

The previous examples have not really been useful. In this section
we will present some more useful specifications.\pause

Let us start with some very well known concepts, namely
\textcolor{blue}{Mealy and Moore automata}:\pause



\begin{SpecDefn}{Moore\_Automata\_1}~=
 \I\Sort \( In, Out, States; \)
 \I\Ops \( next : In\* States \tfun States;\) \\ \(observe : States \tfun Out; \)
 \item[\End]
\end{SpecDefn}



\end{Paragraph}
\begin{Paragraph}[Title={Moore--Automata}, Label=Paragraph93]

\begin{SpecDefn}{Moore\_Automata\_2}~=
\I\Sort \( In, Out; \) \item[\Then] \item[\Cogenerated~\Group]
\begin{Items}
\I\Sort \( States; \) \I\Ops \( next : In \* States \tfun
States;\) \\ \(observe : States \tfun Out; \)
 ~\EndGroup \end{Items}  \item[\End]
\end{SpecDefn}

\end{Paragraph}
\begin{Paragraph}[Title={Moore--Automata}, Label=Paragraph94]

\begin{SpecDefn}{Moore\_Automata\_3}~=
\I\Sort \( In, Out; \) \item[\Then] \item[\Cofree~\Group]
\begin{Items}
\I\Sort \( States; \) \I\Ops \( next : In \* States \tfun
States;\) \\ \(observe : States \tfun Out; \)
 ~\EndGroup \end{Items}  \item[\End]
\end{SpecDefn}

\end{Paragraph}
\begin{Paragraph}[Title={Moore--Automata}, Label=Paragraph95]

Which classes of models are specified by these three
specifications?\pause

\vspace{8mm}

The first specification uses loose semantics and specifies the
class of all Moore--Automata (with arbitrary input and output
alphabets). \pause

\vspace{8mm}

 The second specifications constrains the model class to the most
abstract, i.e. state reduced, Moore--Automata. \pause

\vspace{8mm}

 The third specification restricts  the model class
additionally to those Moore--Automata that are final in the
classes of Moore--Automata with equal input alphabets and equal
output alphabets.

\end{Paragraph}
\begin{Paragraph}[Title={final Moore--automaton}, Label=Paragraph96]

\ns

 What we can say on final Moore--automata? \pause \vspace{8mm}

First, it is easy to see that any two final Moore--automata with
equal input and output alphabet are isomorphic. \pause
\vspace{8mm}

But, does a final Moore--automaton exist and how it would looks
like? We know that in a final coalgebra any two elements can be
distinguished by an experiment. What kind of experiments can be
performed in a given state on an Moore--automaton?

\end{Paragraph}
\begin{Paragraph}[Title={final Moore--automaton}, Label=Paragraph97]
\small What kind of experiments can be performed with an arbitrary
state?\pause \vspace{8mm}
 First, a state denoted by $s$ may be observed by
$observe_{\mathbb{F}}(s)$ which results in a value in
$\mathbb{F}_{Out} = B$. Do we have to identify any two states
$s_1, s_2$ if $observe_{\mathbb{F}}(s_1) =
observe_{\mathbb{F}}(s_2)$? \\ \pause \vspace{8mm}
Now we can perform further experiments like \\
\pause $observe_{\mathbb{F}}(next_{\mathbb{F}}(x_1,s))$ \\ \pause
$observe_{\mathbb{F}}(next_{\mathbb{F}}(x_2,next_{\mathbb{F}}(x_1,s)))$ \\
\pause and so on.

\end{Paragraph}
\begin{Paragraph}[Label=Paragraph98]

 Let us define for any word $w \in
A^*$ $next_{\mathbb{F}}(w,s)$ inductively by
$$\begin{array}{lll}
next_{\mathbb{F}}(\varepsilon,s) &  = & s \\
next_{\mathbb{F}}(vx,s) & = &
next_{\mathbb{F}}(x,next_{\mathbb{F}}(v,s))\\ \end{array}$$\pause
Notice that the word $w \in A^*$ is processed from left to right.
By induction on $w \in A^*$ one can prove
 $$next_{\mathbb{F}}(w, next_{\mathbb{F}}(x ,s)) =
  next_{\mathbb{F}}(xw ,s)).$$\pause

 With this notation we can say that two
states behave differently iff there exists an input word $w \in
A^*$ with $$ observe_{\mathbb{F}}(next_{\mathbb{F}}(w,s_1)) \not=
observe_{\mathbb{F}}(next_{\mathbb{F}}(w,s_2)).$$

\end{Paragraph}
\begin{Paragraph}[Label=Paragraph99]

Or in other words, they behave equally, if the generate the same
input--output--mapping, i.e., $$\lambda\;  w \in A^* \; . \;
observe_{\mathbb{F}}(next_{\mathbb{F}}(w,s_1)) = \lambda\;  w \in
A^* \; . \;
observe_{\mathbb{F}}(next_{\mathbb{F}}(w,s_2)).$$\pause

The next important observation is that the mappings $\{f : A^* \to
B \}$ can be made to an interpretation which satisfies the final
constraint. \pause

We set $$\begin{array}{lll} \mathbb{F}_{States} & = & \{f : A^*
\to B \} \\ observe_{\mathbb{F}}(\varphi) &= &
\varphi(\varepsilon) \\ next_{\mathbb{F}}(x,f) & = & \lambda \; w
\in A^* \; . \; f(xw)\\ \end{array}$$

\end{Paragraph}
\begin{Paragraph}[Label=Paragraph100]

To prove that $ \mathbb{F}$ satisfies the final constraint let
$\mathbb{P}$ be any interpretation with $P_{In} = A,$ and $P_{Out}
= B$. \pause We have to prove that there is exactly on
homomorphism $\varphi^* : \mathbb{P} \to \mathbb{F}$ with
$\varphi^*_{In} = Id_A$ and $\varphi^*_{Out} = Id_B$. \\ \pause We
define $\varphi^*_{States} : P_{States} \to F_{States}$ by
$$\varphi^*_{States}(p) = \lambda \; w \in A^* . \;
observe_{\mathbb{P}}(next_{\mathbb{P}}(w,p)).$$ \pause It remains
to show, that this definition really defines a homomorphism.\pause
$$\begin{array}{lll}
\varphi^*_{States}(next_{\mathbb{P}}(x,p)) & = & \lambda \; w \in
A^* . \; observe_{\mathbb{P}}(next_{\mathbb{P}}(w,
next_{\mathbb{P}}(x,p)) \\
 & = & \lambda \; w \in A^* . \;
 observe_{\mathbb{P}}(next_{\mathbb{P}}(xw,p)) \\
 & = & next_{\mathbb{F}}(x, \lambda \; w \in A^* . \;
 observe_{\mathbb{P}}(next_{\mathbb{P}}(w,p))) \\
  & = & next_{\mathbb{F}}(x, \varphi^*_{States}(p))\\
  \end{array}$$
  proves the compatibility with the $next$--operation, and
$$\begin{array}{lll}
observe_{\mathbb{F}}(\varphi^*_{States}(p)) & = &
observe_{\mathbb{F}}(\lambda \; w \in A^* . \;
observe_{\mathbb{P}}(next_{\mathbb{P}}(w,p))) \\
 & = & \lambda \; w \in A^* . \;
observe_{\mathbb{P}}(next_{\mathbb{P}}( \varepsilon,p)))\\
 & = & \lambda \; w \in A^* . \; observe_{\mathbb{P}}(p) \\
 & = & observe_{\mathbb{P}}(p) \\  \end{array}$$
 proves the compatibility with the $observe$--operation.\pause

 It remains to show that this is the only homomorphism. This is
 left to the tutorial.

\end{Paragraph}
 \begin{Paragraph}[Title={Mealy--automata}, Label=Paragraph101]

 Another type of processes is given by Mealy--automata

\begin{SpecDefn}{Mealy\_Automata}~=
\I\Sort \( In, Out; \) \item[\Then] \item[\Cofree~\Group]
\begin{Items}
\I\Sort \( States; \) \I\Ops \( next : In \* States \tfun
States;\) \\ \(observe :In \*  States \tfun Out; \)
 ~\EndGroup \end{Items}  \item[\End]
\end{SpecDefn}\pause

In that process type the output mapping depends on the state and
an additional input value.

\end{Paragraph}
 \begin{Paragraph}[Title={Mealy--automata}, Label=Paragraph102]

 What about a process $\mathbb{F}$ which
satisfies this final constraint?\pause \vspace{8mm}

It is not to difficult to see, that one has to take now
$F_{States} = {States} = \{ f : A^+ \to B \}$ and
$observe_{\mathbb{F}}(x,f) = f(x)$, where on the right hand side
$x$ represents the string of length one, consisting of the input
value $x \in A$. The state transition function $next$ can be
defined as in the previous example.

\end{Paragraph}
\begin{Paragraph}[Title={Streams}, Label=Paragraph103]

A famous process type is given by so--called \textbf{streams}:

\vspace{12mm}

\begin{SpecDefn}{Streams}[\Sort \(Elem\)]~=
\item[\Then] \item[\Cofree~\Group]
\begin{Items}
\I\Sort \( Streams; \) \I\Ops \( head : Streams  \tfun Elem;\)
\\ \(tail : Streams \tfun Streams; \)
 ~\EndGroup \end{Items}  \item[\End]
\end{SpecDefn}

\end{Paragraph}
\begin{Paragraph}[Title={Streams}, Label=Paragraph104]

The model $\mathbb{F}$ with $F_{Elem} = B$ which satisfies this
specification is given by $$ \begin{array}{lll} F_{Streams} & = &
\{ s : \mathbb{N} \to B \} \\ head_{\mathbb{F}}(s) & = & s(0) \\
tail_{\mathbb{F}}(s)& = & \lambda \; x \in \mathbb{N} \;. \;
s(x+1) \\ \end{array}$$

\end{Paragraph}
\begin{Paragraph}[Title={Function Type}, Label=Paragraph105]

If one wants to work with the function type, this can also be
defined as a process type. The intuition behind this specification
is to consider functions as one--step computations.

\begin{SpecDefn}{Function\_Type}~=
\I\Sort \( In, Out; \) \item[\Then] \item[\Cofree~\Group]
\begin{Items}
\I\Sort \( Fun; \) \I\Ops \( apply : In \* Fun \tfun Out;\)
 ~\EndGroup \end{Items}  \item[\End]
\end{SpecDefn}

\end{Paragraph}
\begin{Paragraph}[Title={Function Type}, Label=Paragraph106]

Lambda--abstraction, known from $\lambda$--calculus turns out as
the universal property of the final model, i.e., let be given
$$ f : In \times M \to Out$$
then $ \lambda m \in M \centerdot f : M \to (In \to Out)$
coincides with the $Fun$--component of the unique homomorphism
$$!_{\mathbb{M}} : \mathbb{M} \to \mathbb{F}$$
where $\mathbb{F}_{Fun} = Out^{In}$.

\end{Paragraph}
\begin{Paragraph}[Title={Function Type}, Label=Paragraph107]

On the other side, if there is given a mapping $$\varphi : In \to
Out ( \in Out^{In})$$ \pause

this mapping can also be seen as $$\varphi : In \times \{ * \} \to
Out$$ \pause

such that the unique homomorphism to the final model maps $*$ to
$\varphi$.

\end{Paragraph}
\begin{Paragraph}[Title={Observable Termination}, Label=Paragraph108]
\ns

 By the next specification we want to specify a process type
which represents terminating and non--terminating  streams which
we call \textbf{sequences}. To represent the the observable
termination of a process we need something like runtime message.

\end{Paragraph}
\begin{Paragraph}[Title={Observable Termination}, Label=Paragraph109]
\small
\begin{SpecDefn}{Sequences}~=
\I\Sort \( Elem; \) \item[\Then] \item[\Cofree~\Group]
\begin{Items}
\I\Sort \( Seq; \) \item[\Then] \item[\Free~\Group]
\begin{Items}
\I\Sort \( Result; \) \I\Ops \( end : Result; \)
 \\ \( emb : Elem \* Seq \tfun Result;\)
  ~\EndGroup \end{Items}
  \I\Ops \( step : Seq \tfun Result;\)
 ~\EndGroup \end{Items}  \item[\End]
\end{SpecDefn}

\end{Paragraph}
\begin{Paragraph}[Title={Observable Termination}, Label=Paragraph110]
\ns
 This specification would look much simpler, if in addition to
the cartesian product $Typ_1 \times Typ_2$ the \textbf{disjoint
union} $Typ_1 + Typ_2$ together with the canonical injections
$inj_1 : Type_1 \to  Typ_1 + Typ_2$ and $inj_2 : Type_2 \to  Typ_1
+ Typ_2$ is on hand.

 \vspace{8mm}

 In that case the specification
would look like:

\end{Paragraph}
\begin{Paragraph}[Title={Observable Termination}, Label=Paragraph111]
\small
\begin{SpecDefn}{Sequences}~=
\I\Sort \( Elem; \) \item[\Then] \item[\Cofree~\Group]
\begin{Items}
\I\Sort \( Seq; \) \I\Ops \( step : Seq \tfun (Elem \* Seq)
+\mathbf{1};\)
 ~\EndGroup \end{Items}  \item[\End]
\end{SpecDefn}
where $\mathbf{1}$ denotes the constant type of the empty
cartesian product, representing the singleton set $\{*\}$.

In order to make the following specifications more readable,
\textbf{we assume for the following, that the disjoint union + is
available}.

\end{Paragraph}
\begin{Paragraph}[Title={Observable Termination}, Label=Paragraph112]
\ns
With this assumption partial Moore--Automata would be
specified as follow:
\begin{SpecDefn}{Partial\_Moore\_Automata\_3}~=
\I\Sort \( In, Out; \) \item[\Then] \item[\Cofree~\Group]
\begin{Items}
\I\Sort \( States; \) \I\Ops \( next : In \* States \tfun States +
1;\) \\ \(observe : States \tfun Out; \)
 ~\EndGroup \end{Items}  \item[\End]
\end{SpecDefn}


\end{Paragraph}
\begin{Paragraph}[Title={Nondeterminsm}, Label=Paragraph113]

Finally we will specify processes with finitary
nondeterminism.\vsp

For a given set of action the dynamics of the process associates
with each action and each state a finite, possibly empty set of
successor states.

\end{Paragraph}
\begin{Paragraph}[Label=Paragraph114]
\small
\begin{SpecDefn}{Nondeterministic\_Automata}~=
\I\Sort \( Actions; \) \item[\Then] \item[\Cofree~\Group]
\begin{Items}
\I\Sorts \( States;\) \item[\Then] \item[\Free~\Group]
\begin{Items}
\I\Sort \( StateSets; \) \I\Ops \( empty : StateSets; \)
 \\ \( add : States \* StateSets \tfun StateSets;\)
 \I\Axioms \( \forall x : States, s : StateSets \. add(x,
 add(x,s)) = s;\)
 \\ \(\[\forall x,y : States, s : StateSets \. \\ add(x,
 add(y,s)) =  add(y, add(x,s)); ~\EndGroup \]\)
   \end{Items}
  \I \Op \( step : States \tfun StateSets;\)
 ~\EndGroup \end{Items}  \item[\End]
\end{SpecDefn}

\end{Paragraph}
\begin{Paragraph}[Title={Nondeterminism}, Label=Paragraph115]

In that case it is not so easy to prove the existence of a most
abstract model.\vsp

If we use the weaker cogenerated constraint, the resulting class
of models is easier to describe.

\end{Paragraph}
\begin{Paragraph}[Label=Paragraph116]
\small
\begin{SpecDefn}{Nondeterministic\_Automata\_2}~=
\I\Sort \( Actions; \) \item[\Then] \item[\Cogenerated~\Group]
\begin{Items}
\I\Sorts \( States;\) \item[\Then] \item[\Free~\Group]
\begin{Items}
\I\Sort \( StateSets; \) \I\Ops \( empty : StateSets; \)
 \\ \( add : States \* StateSets \tfun StateSets;\)
 \I\Axioms \( \forall x : States, s : StateSets \. add(x,
 add(x,s)) = s;\)
 \\ \(\[\forall x,y : States, s : StateSets \. \\ add(x,
 add(y,s)) =  add(y, add(x,s)); ~\EndGroup \]\)
   \end{Items}
  \I \Op \( step : States \tfun StateSets;\)
 ~\EndGroup \end{Items}  \item[\End]
\end{SpecDefn}

\end{Paragraph}
\begin{Paragraph}[Title={Bisimulation}, Label=Paragraph117]
\small
This specification defines the class of so--called \Emphasis{most
abstract} non--deterministic automata, where an automaton is
called most abstract if any two states are indistinguishable. \vsp

In case of deterministic automata two states are indistinguishable
if they generate the same input--output function (see the
construction of a final deterministic automaton). \pause  For
non--deterministic automata this notion is based on the concept of
\Emphasis{bisimulation relations}.\vsp

Let $\mathbb{P}, \mathbb{Q}$ be any two non--deterministic
automata with equal sets of actions.

\end{Paragraph}
\begin{Paragraph}[Title={Bisimulation}, Label=Paragraph118]

\begin{Definition}[Label=DefBisimulation,Title=Bisimulation]
A binary relation $ \mathcal{S} \subseteq \mathbb{P}_{States}
\times \mathbb{Q}_{States}$ is called a \Emphasis{(strong)
simulation} if whenever $(p,q) \in \mathcal{S}$ implies that for
all $a \in \mathbb{P}_{Actions} (= \mathbb{Q}_{Actions}), q' \in
\mathbb{Q}_{States}$ with $q' \in step_{\mathbb{Q}}(q)$ there
exists a $p'\in \mathbb{P}_{States}$ with $p' \in
step_{\mathbb{P}}(p)$ and $(p',q') \in \mathcal{S}$.\\ \pause
$\mathcal{S}$ is called a \Emphasis{(strong) bisimulation} if both
$\mathcal{S}$ and the inverse relation $\mathcal{S}^{-1}$ are
strong simulations.\\ \pause Two states $p, q$ are called
\Emphasis{bisimular} (or indistinguishable), if there exists a
bisimulation $ \mathcal{S}$ with $(p,q) \in \mathcal{S}$.
\end{Definition}

\end{Paragraph}
\begin{Paragraph}[Title={Nondeterminism}, Label=Paragraph119]

The \textcolor{blue}{final non--deterministic automaton} can now
be understood as the greatest most abstract non--deterministic
automaton, containing each most abstract non--deterministic
automaton as a sub--automaton.

\end{Paragraph}
\end{Section}
\begin{Section}[Title={Coinduction}, Label={section4_3}]
\begin{Paragraph}

\small
In case of process type one wants to define functions that built
new behaviors out of given behaviors.\pause \vspace{8mm}

The standard way to define such functions is
\textcolor{red}{\textbf{coinduction}}. \pause \vspace{8mm}

The process type of streams is simple enough to present  fist
simple coinductive function definitions.\pause \vspace{8mm}

The \textcolor{blue}{\textbf{general methodology of a coinductive
definition}} is the following: \\ \pause \emph{Define over the
domain of the wanted function a process in such a way, that the
unique homomorphism from the domain process to the final model
represents the wanted function.}

\end{Paragraph}
\begin{Paragraph}[Title={Coinduction}, Label=Paragraph120]
\ns

 Let us consider the  function of  \textcolor{blue}{merging
two streams}. Streams are merged together like the two sides of a
zipper.

\end{Paragraph}
\begin{Paragraph}[Title={Coinduction}, Label=Paragraph121]

\small
\begin{SpecDefn}{Streams\_2}~=
\I\Sort \( Elem; \) \item[\Then] \item[\Cofree~\Group]
\begin{Items}
\I\Sort \( Streams; \) \I\Ops \( head : Streams  \tfun Elem;\)
\\ \(tail : Streams \tfun Streams; \)
 ~\EndGroup\end{Items}
\item[\Then] \item[\Cofree~\Group]
\begin{Items}
\I\Op \( merge : Streams \* Streams \tfun Streams; \)
 \I\Axiom \( \forall s_1, s_2 : Streams \. head(merge(s_1, s_2) =
 head(s_1) ; \) \\
 \(\forall s_1, s_2 : Streams \. tail(merge(s_1, s_2)) =
 merge(s_2, tail(s_1));\)
 ~\EndGroup \end{Items} \item[\End]
\end{SpecDefn}


\end{Paragraph}
\begin{Paragraph}[Title={Coinduction}, Label=Paragraph122]

According to that methodology we defined two mappings
$$\begin{array}{l} head_m : Streams \times Streams \to Elem, \\ tail_m :
Streams \times Streams \to Streams \times Streams\\ \end{array}$$
by $head_m(s_1, s_2) = head(s_1)$ and $tail_m(s_1, s_2) = (s_2,
tail(s_1))$. Since $merge$ becomes a homomorphism we obtain:
$(head(s_1) = )head_m(s_1, s_2) = head(merge(s_1, s_2))$ and
$merge(tail_m(s_1, s_2)) = tail(merge(s_1,s_2))$ which means by
definition of $tail_m$ that $merge(s_2, tail(s_1)) =
tail(merge(s_1, s_2).$ Thus, the specification above defines
$merge$ without explicitly using the two operations $head_m,
tail_m$.

\end{Paragraph}
\begin{Paragraph}[Title={Coinduction}, Label=Paragraph123]



\begin{SpecDefn}{Sequencial\_Composition\_of\_Sequences}~=
\SpecName{Sequences} \item[\Then] \item[\Cofree~\Group]
\begin{Items}\I\Op \( app : Seq \* Seq \tfun Seq; \)
\I\Axioms \(\[\forall s_1, s_2 : Seq \. step(s_1) = inj_2(*)
\wedge step(s_2) = inj_2(*) \\\M \imp step(app(s_1,s_2)) =
inj_2(*);\]\)
\\ \(\[\forall s, s_1, s_2 : Seq \. step(s_1) = inj_1(x,s) \\\M \imp
step(app(s_1,s_2)) = inj_1(x,(app(s,s_2)));\]\)
\\ \(\[\forall s_1, s_2 : Seq \. step(s_1) = inj_2(*) \wedge
step(s_2) = inj_1(x,s) \\\M \imp step(app(s_1,s_2)) = inj_1(x,s);
~\EndGroup \]\)
\end{Items}  \item[\End]
\end{SpecDefn}

\end{Paragraph}
\begin{Paragraph}[Title={Coinductive Proofs}, Label=Paragraph124]

This example can be used to show that final constraints do not
only provide a definition principle, as demonstrated above, but
also a \textcolor{red}{\textbf{proof principle, based on the
uniqueness of the homomorphism}}.\pause \vspace{8mm}

Intuitively one would expect that the sequence $app(s_1, s_2)$
behaves like $s_1$ if $s_1$ is a non--terminating sequence. To
prove this property by \textbf{coinduction} one has to transform
this property into the identity of homomorphisms to the final
model.\pause \vspace{8mm}

How we can characterize the class of non--terminating sequences,
i.e., streams, as a subclass of sequences?

\end{Paragraph}
\begin{Paragraph}[Title={Coinductive Proofs}, Label=Paragraph125]

First we have to be aware that in most cases subtypes of processes
can not be defined by means of a defining predicate.
\pause\vspace{8mm}

In the example of the process types of streams and sequences one
can see streams as a subtype of sequences. Since the halting
problem is undecidable we can not specify the characteristic
function of that subtype.

\end{Paragraph}
\begin{Paragraph}[Label=Paragraph126]

There are also formal reasons why we are not able to specify the
characteristic function of the subtype of non--terminating
sequences. The domain type of the function we want to specify is
defined be a final constraint and the range type by an initial
constraint. An inductive definition would require a domain type
defined by an initial constraint, and a coinductive definition
requires a range type defined by a final constraint.\vsp

But we
can construct by coinduction an embedding of the class of streams
into the class of sequences. In the following coinductive
specification $embed : Streams \to Seq$ will denote the embedding
of streams into sequences:

\end{Paragraph}
\begin{Paragraph}[Label=Paragraph127]

\begin{SpecDefn}{Embedding\_of\_Streams\_into\_Sequences}~=
\I\Sort \( Elem; \) \item[\Then] \item[\Cofree~\Group]
\begin{Items}
\I\Sorts \( Streams, Seq;\) \I\Ops \( head : Streams  \tfun
Elem;\)
\\ \(tail : Streams \tfun Streams; \)
\\ \(step : Seq \tfun (Elem \* Seq) + \mathbf{1};\)
 ~\EndGroup \end{Items}
\item[\Then] \item[\Cofree~\Group]
\begin{Items}
\I\Op \( \textcolor{red}{embed : Streams \tfun Seq}; \)
  \I\Axiom
\(\forall s : Streams \. step(embed(s)) =
inj_1(head(s),embed(tail(s)));\)
  ~\EndGroup \end{Items} \item[\End]
\end{SpecDefn}

\end{Paragraph}
\begin{Paragraph}[Title={Coinductive Proofs}, Label=Paragraph128]

To prove the expected property we define the following two
mappings
$$\begin{array}{lll}
f : Streams \times Seq \to Seq & \mbox{with} & f(s_1,s_2) =
embed(s_1) \\
h : Streams \times Seq \to Seq & \mbox{with} & h(s_1,s_2) =
app(embed(s_1),s_2) \\ \end{array}$$ and show that they are
equal.\pause

Since the range type of both mappings is a type defined by a final
constraint, we can prove the equality by coinduction, which means,
we make $Streams \times Seq$ to a process of the process type
\textsc{Sequences} and prove that both mappings are homomorphisms.
Since there is exactly one homomorphism, this proves the equality.

To follow this approach, we define $$step^* : Streams \times Seq
\to (Elem \times (Streams \times Seq)) + \mathbf{1}$$ by
$$step^*(s_1,s_2) = inj_1(head(s_1),(tail(s_1),s_2)).$$
It remains to prove the commutativity of the following diagrams:
$$ \xymatrix{\ar @{} [drrrr] | {(1)}
Str \times Seq \ar[d]_{step^*} \ar[rrrr]^f & & & & Seq \ar[d]^{step} \\
(El \times (Str \times Seq)) + \mathbf{1} \ar[rrrr]_{(Id_{El}
\times f) + Id_{\mathbf{1}}} & & & & (El\times Seq) + \mathbf{1}
}$$
$$\xymatrix{\ar @{} [drrrr] | {(2)}
Str \times Seq \ar[d]_{step^*} \ar[rrrr]^h & & & &  Seq \ar[d]^{step} \\
(El \times (Str \times Seq)) + \mathbf{1} \ar[rrrr]_{(Id_{El}
\times h) + Id_{\mathbf{1}}} & & & & (El\times Seq) + \mathbf{1}
}$$ The commutativity of diagram $(1)$ follows from
$$\begin{array}{lll}
step(f(s_1,s_2)) & = & step(embed(s_1)) \\
 & =  & inj_1(head(s_1), embed(tail(s_1))) \\
 & = & [(Id_{El} \times f) + Id_{\mathbf{1}}]
 (inj_1(head(s_1),(tail(s_1),s_2))) \\
 & = &  [(Id_{El} \times f) + Id_{\mathbf{1}}](step^*(s_1,s_2))
 \\ \end{array}$$
  and the commutativity of diagram $(2)$ is given by:
$$\begin{array}{lll}
step(h(s_1,s_2)) & = & step(app(embed(s_1),s_2)) \\
 & = & inj_1(head(s_1), app(embed(tail(s_1)),s_2)) \\
  & = & [(Id_{El} \times h) +
  Id_{\mathbf{1}}](inj_1(head(s_1),(tail(s_1),s_2)))\\
   & = & [(Id_{El} \times h) + Id_{\mathbf{1}}](step^*(s_1,s_2))
 \\ \end{array}.$$

\end{Paragraph}
\end{Section}
 \begin{Section}[Title={Behavioural Subtypes}, Label={section4_4}]
\begin{Paragraph}

Above we argued, that in general interesting subclasses of process
types can not be specified by boolean--valued functions.\vsp

In the following we will demonstrate that predicates defined by
initial and final constraints can be used to specify behavioural
subtypes.\vsp

The first example specifies those streams that eventually produce
(contain or output) an element that satisfies a given property.

\end{Paragraph}
\begin{Paragraph}[Title={sometimes}, Label=Paragraph129]

\begin{SpecDefn}{SomeTimes}~= \SpecName{Streams}
 \item[\Then]
\I\Pred \( P : Elem; \) \item[\Free~\Group]
\begin{Items}
\I\Pred \( ST : Streams;\) \I\Vars \( s : Streams; \) \I\Axioms \(
P(head(s)) \Implies ST(s);\)
\\ \(ST(tail(s)) \Implies ST(s);\)
  ~\EndGroup \end{Items} \item[\End]
\end{SpecDefn}

\end{Paragraph}
\begin{Paragraph}[Title={sometimes}, Label=Paragraph130]

\begin{SpecDefn}{SomeTimes}~= \SpecName{Streams}
 \item[\Then]
\I\Pred \( P : Elem; \) \item[\Free~\Group]
\begin{Items}
\I\Pred \( ST : Streams;\) \I\Vars \( s : Streams; \) \I\Axioms \(
P(head(s)) \Implies ST(s);\)
\\ \(ST(tail(s)) \Implies ST(s);\)
  ~\EndGroup \end{Items} \item[\End]
\end{SpecDefn}

Notice,  this \Emphasis{liveness property} is expressed by an
initial constraint. This is related to the fact, that liveness
properties are represented as least fixed points in
$\mu$--calculus.

\end{Paragraph}
\begin{Paragraph}[Title={allways}, Label=Paragraph131]

If we want to specify the class of all those streams that contain
only elements which satisfy a given property, then we have to
define the corresponding predicate by a final constraint:

\begin{SpecDefn}{AllWays}~= \SpecName{Streams}
 \item[\Then]
\I\Pred \( P : Elem; \) \item[\Cofree~\Group]
\begin{Items}
\I\Pred \( AW : Streams;\) \I\Vars \( s : Streams; \) \I\Axioms \(
AW(s) \Implies P(head(s));\)
\\ \(AW(s) \Implies AW(tail(s));\)
  ~\EndGroup \end{Items} \item[\End]
\end{SpecDefn}

\end{Paragraph}
\begin{Paragraph}[Title={allways}, Label=Paragraph132]

\ns Notice, this \Emphasis{safety property} is expressed by a
final constraint. This is related to the fact, that safety
properties are represented as greatest fixed points in
$\mu$--calculus.

\end{Paragraph}
\begin{Paragraph}[Title={allways sometimes}, Label=Paragraph133]

\ns The next example deals with the specification of a
\Emphasis{fairness property} which states that a stream contains
infinitely often elements satisfying  a given property.

\end{Paragraph}
\begin{Paragraph}[Label=Paragraph134]
\small
\begin{SpecDefn}{AllWays\_SomeTimes}~= \SpecName{Streams} \I \Then
\I\Pred \( P : Elem; \) \item[\Cofree~\Group] \item[\Then]
\begin{Items}
 \item[\Free~\Group]
\begin{Items}
\I\Pred \( ST : Streams;\) \I\Vars \( s : Streams; \) \I\Axioms \(
P(head(s)) \Implies ST(s);\)
\\ \(ST(tail(s)) \Implies ST(s);\)
  ~\EndGroup \end{Items}
\I\Pred \( AwSt : Streams;\) \I\Vars \( s : Streams; \) \I\Axioms
\( AwSt(s) \Implies ST(s);\)
\\ \(AwSt(s) \Implies AwSt(tail(s));\)
  ~\EndGroup \end{Items} \item[\End]
\end{SpecDefn}

\end{Paragraph}
\begin{Paragraph}[Title={allways sometimes}, Label=Paragraph135]

\ns The specification \SpecName{AllWays\_SomeTimes} demonstrates
that the combination of initial and final constraints is much more
expressive then initial or final constraints alone.

\end{Paragraph}
\begin{Paragraph}[Title={Component Connectors}, Label=Paragraph136]
\small The next examples present a notion of J. Rutten, who uses
coinductively defined predicates to represent a channel--based
coordination model. The coordinators are called
\textcolor{blue}{connectors}. The input ends and output ends of
connectors are so--called \textcolor{blue}{timed data streams}. A
timed data stream is a pair of a stream (infinite sequence) of
abstract data items and an infinite sequence of natural numbers,
representing the moments in time at which their corresponding data
items are being input or output.\pause

See: F. Arbab, J.J.M.M. Rutten: A coinductive calculus of
component connectors, CWI Report SEN--R0216, September 30, 2002


\end{Paragraph}
\begin{Paragraph}[Title={Timed Data Streams}, Label=Paragraph137]
\small
\begin{SpecDefn}{Timed\_Data\_Streams}[\Sort Data] ~= \textsc{Nat}
\Then \item[\Cofree~\Group]
\begin{Items}
\I\Sorts \( Ds, Ts; \) \I\Ops \( hd : Ds \tfun Data \)
\\ \( ht : Ts \tfun Nat; \)
\\ \( \_\_': Ds \tfun Ds; \)
\\ \( \_\_': Ts \tfun Ts;\)
\I\Pred \( Tds : Ds \* Ts;\) \I\Vars \( D : Ds; T : Ts\) \I\Axiom
\( Tds(D,T) \Implies ht(T) < ht(T') \wedge Tds(D', T');\)
~\EndGroup
\end{Items}
\I\Sort \(\textcolor{blue}{TDS[Data] = \{ (D,T): Ds \* Ts \.
Tds(D,T) \}} \);
 \item[\End] \end{SpecDefn}

\end{Paragraph}
 \begin{Paragraph}[Title={Comparing Time Streams}, Label=Paragraph138]

In the following one needs a relation $T1 < T2$ between time
streams  $T1, T2 : Ts$ which states that in each moment the time
stamp of $T1$ denotes an earlier point in time as the time stamp
of $T2$.\pause

\begin{SpecDefn}{Before\_and\_Equal} ~= \SpecName{Timed\_Data\_Streams}
\Then \item[\Cofree~\Group]
\begin{Items}
\I\Pred \( \textcolor{blue}{\_\_ < \_\_ , \_\_ = \_ \_ : Ts \*
Ts}; \)
\\ \(\textcolor{blue}{ \_\_ = \_ \_ , : Ds \* Ds}; \)
\I\Axiom \( \forall T1, T2 : Ts \. T1 < T2 \Implies ht(T1) <
ht(T2) \wedge T1' < T2'; \)
 \\ \( \forall T1, T2 : Ts \. T1 = T2
\Implies ht(T1) = ht(T2) \wedge T1' = T2'; \)
\\ \( \forall D1, TD : Ds \. D1 = D2
\Implies ht(D1) = ht(TD2) \wedge D1' = D2'; \) ~\EndGroup
\end{Items}  \item[\End] \end{SpecDefn}


\end{Paragraph}
\begin{Paragraph}[Title={Timed Merging of TDS}, Label=Paragraph139]
\small\vspace{-5mm}
\begin{SpecDefn}{Merge} ~= \SpecName{Timed\_Data\_Streams}
\item[\Then] \item[\Cofree~\Group]
\begin{Items}
 \I\Pred \( \textcolor{blue}{Me : TDS \* TDS \* TDS };\)
 \I\Vars \( (D1,T1), (D2,DT2), (D3,T3) : TDS ; \)
 \I\Axioms \( Me((D1,T1),(D2,T2),(D3,T3)) \Implies \neg(ht(T1)=
 ht(T2)) \wedge  ; \)
 \\ \(Me((D1,T1),(D2,T2),(D3,T3)) \wedge ht(T1) < ht(T2)) \Implies
 \\ \M hd(D1) = hd(D3) \wedge ht(T1) = ht(T3) \wedge
 Me((D1',T1'),(D2,T2),(D3',T3')); \)
\\ \(Me((D1,T1),(D2,T2),(D3,T3)) \wedge ht(T2) < ht(T1)) \Implies
 \\ \M hd(D2) = hd(D3) \wedge ht(T2) = ht(T3) \wedge
 Me((D1,T1),(D2',T2'),(D3',T3')); \)
  ~\EndGroup \end{Items} \item[\End] \end{SpecDefn}

\end{Paragraph}
\begin{Paragraph}[Title={Timed Merging of TDS}, Label=Paragraph140]

The connector $Me$ merges the two data Streams $D1, D2$ into the
data stream $D3$ on a 'first come first served' basis. It works
only if the data in $(D1,T1), (D2,T2)$ do not appear at the same
moment. The data element that is handled first, say $hd(D1)$ at
time $ht(T1) < ht(T2)$ is the first element to be put into $D3$:
$hd(D1) = hd(D3)$, at exactly the same moment: $ht(T1) = ht(T3)$.
After that, the connector handles the remainder of the streams in
the same manner again: $Me((D1',T1'),(D2,T2),(D3',T3'))$.\vsp

Is this merge connector fair, i.e., does it input infinitely many
data elements from both data streams $D1, D2$?

\end{Paragraph}
\begin{Paragraph}[Title={Component Connectors}, Label=Paragraph141]

By means of the product of relations the composition  of
connectors can be formalized in a natural way.\vsp

We will illustrate this by an example taken from Rutten. It
composes an unbounded fifo lossy buffer with a driver that
corrects the lossiness of the buffer.

\end{Paragraph}
\begin{Paragraph}[Title={Lossy Buffer}, Label=Paragraph142]

\begin{SpecDefn}{Lossy\_Buffer} ~= \SpecName{Timed\_Data\_Streams}
\item[\Then] \item[\Cofree~\Group]
\begin{Items}
 \I\Pred \( \textcolor{blue}{L : TDS[Data] \* TDS[Data] \* TDS[Boole] };\)
 \I\Vars \( (D1,T1), (D2,T2) : TDS[Data], (D3,T3) : TDS[Boole] ; \)
 \I\Axioms \( L((D1,T1), (D2,DT2), (D3,T3)) \Implies
 \\ \M (T1 < T3 \wedge T1 < T2) \wedge
  \\ \M (( hd(D3) = true \wedge hd(D1) = hd(D2) \wedge
  L((D1',T1'), (D2',T2'), (D3', T3')))
  \\ \M \vee (hd(D3) = false \wedge L((D1',T1'), (D2,T2), (D3',
  T3')))); \)
  ~\EndGroup \end{Items} \item[\End] \end{SpecDefn}

\end{Paragraph}
\begin{Paragraph}[Title={Lossy Buffer}, Label=Paragraph143]

The lossy buffer inputs data items at $(D1, T1)$. For every data
there are two possible scenarios:\pause
\begin{List}[ListType=enumeration]
    \ListItem the data item is stored successfully and is output at
    some later moment at $D2$ together (not necessarily
    simultaneously) with a success signal $true$ along $D3$, after
    which the connector proceeds as before with the remainder of
    all streams involved. \pause
    \ListItem storage of the data fails, no data item is output along
    $D2$ and $false$ signaling the failure is output along $D3$,
    and the connector proceeds as before, now with $(D1', T1'),
    (D3', T3')$ but with $(D2, T2)$ unchanged.
\end{List}

\end{Paragraph}
\begin{Paragraph}[Title={Driver}, Label=Paragraph144]

\begin{SpecDefn}{Driver} ~= \SpecName{Timed\_Data\_Streams}
\item[\Then] \item[\Cofree~\Group]
\begin{Items}
 \I\Pred \( \textcolor{blue}{D : TDS[Data] \* TDS[Boole] \* TDS[Data]  };\)
 \I\Vars \( (D1,T1), (D3,T3) : TDS[Data], (D2,T2) : TDS[Boole] ; \)
 \I\Axioms \( L((D1,T1), (D2,DT2), (D3,T3)) \Implies
 \\ \M ( T1 = T3 \wedge T1 < T2 < T1' \wedge hd(D3) = hd(D1))
 \wedge
 \\ \M ((hd(D2) = true \wedge D((D1',T1'), (D2', T2'), (D3',
 T3'))) \vee
 \\ \M (hd(D2) = false \wedge D((D1 ,T1'), (D2', T2'), (D3',
 T3')))); \)
 ~\EndGroup \end{Items} \item[\End] \end{SpecDefn}

\end{Paragraph}
\begin{Paragraph}[Title={Driver}, Label=Paragraph145]

The driver inputs data items at $D1$ and outputs them at $D3$.
Before proceeding with the next data item, it checks it input at
$D2$. \\ \pause If $hd(D2) = true$ then the last data item that
has been output is considered to have been handled correctly (by
the lossy buffer), and the driver proceeds as before with the
remainder of all streams involved. \\ \pause If $hd(D2) = false$,
however, something has gone wrong (the buffer has lost the data
item), and the driver sends the data item again. This is modelled
here by $D((D1 ,T1'), (D2', T2'), (D3',T3'))$, in which all
streams have progressed one step but for $D1$, which remains
unchanged, i.e. $hd(D1)$ is again the next data item that the
driver will output, but with changed time stream $T1'$.

\end{Paragraph}
\begin{Paragraph}[Title={Composing Connectors}, Label=Paragraph146]

\begin{SpecDefn}{Composition} ~= \SpecName{Driver} \And~
\SpecName{Lossy\_Buffer} \I\Then \I\Pred \( \textcolor{blue}{D
\circ L : DTS[Data] \* TDS[Data]};\) \I\Vars \( (D1, T1), (D2, T2)
: TDS[Data];\) \I\Axioms \(D \circ L ((D1, T1), (D2, T2)) \Equiv
\\ \M \Exists (D3, T3) : DTS[Data] \. \Exists (D,T) : DTS[Boole]
\. \\ \M D((D1,T1), (D,T), (D3, T3)) \wedge L((D3, T3), (D,L),
(D2,T2)); \)
\\ \(\textcolor{red}{D \circ L ((D1, T1), (D2, T2))
\Implies D1 = D2 \wedge T1 <
T2 }; \) \item[\End] \end{SpecDefn}

\end{Paragraph}
\begin{Paragraph}[Title={Composing Connectors}, Label=Paragraph147]

The driver takes $(D1, T1)$ and $(D,T)$ as input streams and $(D3,
T3)$ as output stream, whereas the lossy buffer takes $(D3, T3)$
as input and both $(D,T), (D2, T2)$ as output streams. \vsp

This means, that according to the composition $D \circ L$ the
timed streams $(D3, T3)$ and $(D,T)$ are used only for internal
communication. \vsp

The axiom \(\textcolor{red}{D \circ L ((D1, T1), (D2, T2))
\Implies D1 = D2 \wedge T1 < T2 } \) states that \Emphasis{$D
\circ L$ behaves like an unbounded fifo buffer}. The validity of
the axiom can be proved by coinduction. (The proof will be
discussed in the tutorial.)

\end{Paragraph}
\end{Section}
\end{Section}
\begin{Section}[Title={Summary}, Label={section5}]

\begin{Paragraph}[Title={Data Types}, Label=Paragraph148]
\ns
%\textcolor{blue}{\textbf{Data Types}}
\begin{List}[ListType=itemize]
    \ListItem generic data types can be specified by means of initial
    constraints.\pause
    \ListItem for the definition of a data type one should use a
    minimal set of constructors.\pause
    \item initial constraints offer \pause \begin{List}[ListType=itemize]
        \ListItem \Emphasis{definition principle}, based on the existence of a
        homomorphism from the initial model to an arbitrary
        model.\pause
        \ListItem \Emphasis{proof principle}, based on the uniqueness
        of the homomorphism.
    \end{List}
\end{List}\end{Paragraph}
\begin{Paragraph}[Title={Data Types}, Label=Paragraph149]
\begin{List}[ListType=itemize]
    \ListItem Due to the minimality of the set of constructors, data
    types may be laborious to use. However, the value in use can
    be increased by \Emphasis{functional enrichments} of data
    types.\pause
    \ListItem functional enrichments of data types be not allowed to
    generate \Emphasis{confusion} or \Emphasis{junk}.\pause
    \ListItem for each data type one should define the equality
    function as one of the functional enrichments.\pause
    \ListItem functional enrichments are powerful enough to specify
    all computable functions.

\end{List}\end{Paragraph}
\begin{Paragraph}[Title={Data Types}, Label=Paragraph150]
\begin{List}[ListType=itemize]

    \ListItem relations defined by initial constraints are interpreted
    minimally.\pause
    \ListItem partial constructors of data types have minimal domains
    of definitions.
\end{List}

\end{Paragraph}
\begin{Paragraph}[Title={Process Types}, Label=Paragraph151]
\begin{List}[ListType=itemize]
    \ListItem generic process types can be specified by means of final
    constraints.\pause
    \ListItem for the definition of a process type one should use a
    minimal set of observations (destructors, experiments). \pause
    \item \begin{List}[ListType=itemize]
        \ListItem final  constraints offer a
       \Emphasis{definition principle}, based on the existence of a
        homomorphism to the final  model from an arbitrary
        model.\pause
        \ListItem \Emphasis{proof principle}, based on the uniqueness
        of the homomorphism.
    \end{List}
\end{List}\end{Paragraph}
\begin{Paragraph}[Title={Process Types}, Label=Paragraph152]
\begin{List}[ListType=itemize]
    \ListItem based on the chosen set of destructors one can define
    process composing operations a functional enrichments of
    process types.\pause
    \ListItem predicates defined by final constraints are interpreted
    maximally.\pause
    \ListItem the notion of sub--model of a process types corresponds
    to a safety property for the process type.\pause
    \ListItem cogenerated constraints can be used in cases where no
    final model exists, but most abstract models exist.
\end{List}

\end{Paragraph}
\begin{Paragraph}[Title={Nesting of Process and Data Types}, Label=Paragraph153]

\begin{List}[ListType=itemize]
    \ListItem Nesting of final constraints in initial constraints
    leads to new data types that can not be specified with final
    constraints only.\pause
    \ListItem Nesting of initial constraints in final constraints is
    the general case in order to define the observable environment
    of a process type.\pause
    \ListItem Predicates on process types defined by initial
    constraints represent \Emphasis{liveness properties}.

\end{List}\end{Paragraph}
\begin{Paragraph}[Title={Nesting of Process and Data Types}, Label=Paragraph154]
\begin{List}[ListType=itemize]

    \ListItem Predicates on process types defined by final
    constraints represent \Emphasis{safety properties}. \pause
    \ListItem Predicates on process types defined by nesting of final
    and initial constraints represent \Emphasis{fairness properties}.
\end{List}
%%%%%%%%%%%%%%%%%%%%%%%%%%%%%%%%%%%%%%%%%%%%%%%%%%%%%%%%%%%%%%%%%%%%%%

%\end{document}

\end{Paragraph}
\end{Section}
%\begin{Section}[Title={CCSL specifications}, Label={section6}]

%\end{Section}
\begin{Section}[Title={\CASL specifications}, Label={section7}]

\begin{Paragraph}[Label=Paragraph155]
\hypertarget{nat-casl}{\CASL specifications of natural numbers and
lists:}\pause
\begin{SpecDefn}{Natural}~= \item[\Free~ \Type] \(Nat ::= zero \;| \;
suc(Nat)\) \End \end{SpecDefn}\pause \vfill
\begin{SpecDefn}{Lists}[\Sort \hspace{1pt} \(Elem\)] ~=
 \item[\Free~\Group]
\begin{Items}
\I\Sort \(Lists;\) \I\Ops \(nil : Lists; \)
\\ \( cons : Elem \* Lists \tfun Lists \)
 ~\EndGroup \end{Items}  \item[\End]
\end{SpecDefn}

\end{Paragraph}
\begin{Paragraph}[Label=Paragraph156]
\hypertarget{2-trees}{\CASL specifications of binary trees:}\pause
\begin{SpecDefn}{Bintree}~= \item[\Free~ \Type] \(2\_Trees ::= zero \;| \;
2\_cons(2\_Trees, 2\_Trees)\) \End \end{SpecDefn}

\end{Paragraph}
\begin{Paragraph}[Label=Paragraph157]
\hypertarget{trees}{\CASL specification of finitely branching
trees with finite depth}

\begin{SpecDefn}{Trees}~= \item[\Free~\Group]
\begin{Items} \item[\Sort] \(Tree, TreeSeq;\)
\item[\Ops] \( nil : TreeSeq;\)
\\ \( h\_cons : Tree \times TreeSeq \tfun TreeSeq;\)
\\ \( v\_cons : TreeSeq \tfun Tree \)  ~\EndGroup
\end{Items} \item[\End] \end{SpecDefn}

\end{Paragraph}
\begin{Paragraph}[Label=Paragraph158]
\hypertarget{integer}{\CASL specification of integers}

\begin{SpecDefn}{Integer}~= \item[\Free~\Group]
\begin{Items} \item[\Sort] \( Int; \)
\item[\Ops] \( 0: Int;\)
\\ \( suc, pred : Int \tfun Int;\)
\item[\Axioms] \( \forall x : Int \. pred( suc (x)) = x ; \)
\\ \( \forall x : Int \. suc (pred (x)) = x ; \)
~\EndGroup
\end{Items} \item[\End] \end{SpecDefn}

\end{Paragraph}
\begin{Paragraph}[Label=Paragraph159]

\hypertarget{integer2}{another \CASL specification of integers}

\begin{SpecDefn}{Integer\_2}~= \\ \textsc{Natural}
\item[\Then] \item[\Free~\Group]
\begin{Items} \item[\Sort] \( Int; \)
\item[\Ops] \( pos : Nat \tfun Int; \)
\\ \( neg : Nat \tfun Int\)
\item[\Axiom] \( pos(0) = neg(0)\) ~\EndGroup
\end{Items} \item[\End] \end{SpecDefn}


\end{Paragraph}
\end{Section}
\begin{Section}[Title={BOBJ specifications}, Label={section8}]
\begin{Paragraph}[Label=Paragraph160]
\hypertarget{nat-bobj}{BOBJ specifications of natural numbers and
lists:}\pause
\small
\begin{verbatim}
  dth NAT is
     sort Nat .
     op   zero : -> Nat .
     op   suc : Nat -> Nat .
  end
\end{verbatim}\pause

\begin{verbatim}
  dth LIST[B :: TRIV] is
     sort List .
     op   nil : -> List .
     op   cons : Elt List -> List .
  end
\end{verbatim}

\end{Paragraph}
\begin{Paragraph}[Label=Paragraph161]

where \texttt{TRIV} is given by
\begin{verbatim}
 th TRIV is
    sort Elt .
 end
 \end{verbatim}
 \ns
\end{Paragraph}
\end{Section}
\end{Package}
\end{document}



\end{Package}
\end{document}
