\documentclass[landscape, slides, light]{mmiss}

\usepackage{latexsym}
\usepackage{amssymb}
\usepackage{casl}

\newcommand{\gram}[1]{{\texttt{#1}}}
\newcommand{\power}{\hat{\ }}
\newcommand{\annoimplies}{\%{\bf implies}\ }
\newcommand{\annodef}{\%{\bf def}\ }
\newcommand{\annocons}{\%{\bf conservatively}\ }
\newcommand{\annoprec}{\%{\bf prec}\ }
\newcommand{\annorassoc}{\%{\bf right\ assoc}\ }
\newcommand{\annolassoc}{\%{\bf left\ assoc}\ }
\newcommand{\annobrackets}{{\bf brackets}\ }
\newcommand{\Nat}{\mbox{$\mathbb N$}} % this requires amssymb
\def\ShortAuthor{C.~L\"uth, M.~Roggenbach}
\def\ShortTitle{Algebraic Specification}
\def\Title{Formal Program Development with Algebraic Specifications}


\begin{document}
\begin{Package}[Label=AlgSpec, 
Title={Formale Programmentwicklung mit algebraischen Spezifikationen},
Date=12.09.2002,
Authors={Christoph L�th, Markus Roggenbach},
ShortAuthor={C.~L�th, M.~Roggenbach},
ShortTitle={Algebraische Spezifikation},
Language=de,LevelOfDetail=Lecture,InteractionLevel=Hyper]

\begin{Section}
[Label=IntroAlgSpec,
 Title={Signaturen, Terme und Algebren}]

\begin{Paragraph}[Label=Signatures, Title=Signaturen]
Signaturen erlauben die Definition von Operationen.
\vspace{1cm}

\begin{Definition}[Label=DefSignature, Title=Signaturen]
Eine \Emphasis{Signatur} $\Sigma=(S, \Omega)$ sei gegeben durch
\begin{List}[Label=SignatureComponents, ListType=itemize]
    \ListItem eine Menge $S$ von \Emphasis{Sorten}, und
    \ListItem eine Familie $\{\Omega_{w,s}\}_{w\in S^{*},s\in S}$ von Operationen.
  \end{List}
\end{Definition}

Eine Operation $\omega\in\Omega_{w,s}$ mit $w=s_1\ldots s_n$ hat
Argumentsorten $s_1,\ldots,s_n$ und die Zielsorte $s$, auch
geschrieben als $\omega:s_1\ldots s_n\rightarrow s$.
\end{Paragraph}
\end{Section}
\end{Package}
\end{document}














