\documentclass[landscape, slides, light]{mmiss2}
\usepackage{latexsym}
\usepackage{casl}
\begin{document}
\begin{Package}[Notation=MMISS_Latex2, Label=pack1, Title=Algebraic Specification, Date=6.09.2002,Version=4,PreviousVersion=3,Authors={Markus Roggenbach, Michael Drouineaud},PriorAuthors={Markus Roggenbach},Comment=Test,Language=de,LevelOfDetail=LectureNotes,InteractionLevel=Hyper]

%\maketitle

\begin{Section}[Label=Chapter1, Title=Introduction, Date=6.09.2002, Version=4, PreviousVersion=3, Authors={Markus Roggenbach, Michael Drouineaud}, PriorAuthors={Markus Roggenbach}]
\begin{Abstract}[Label=a1, Title=Abstract]
  Dies ist der Abstract.
\end{Abstract}
\begin{Introduction}[Label=intro1, Title=Introduction]
 \begin{TextFragment}[Notation=Latex, Label=tf1]
  Dies ist die Einf\"uhrung.
 \end{TextFragment}
\end{Introduction}
\begin{Section}[Label={Section1.1}, Title={Formal Methods in Software Design}]
  Text, direkt am Beginn der Section1.1 
  ``Use of mathematics in software development''\\
\begin{TextFragment}
 Dies ist ein TextFragment hinter der Einf\"uhrung mit
 \Emphasis{hervorgehobenem} Text. Ausserdem steht hier noch
 ein \LaTeX{}-Befehl drin.
\end{TextFragment}
\Link[]{para2}{}
\ForwardLink[]{para2}{}
\Reference{para2}{}
\ForwardReference{para2}{}
\begin{Paragraph}[Label=para1, Title=First Paragraph, Authors=Achim Mahnke, Date=11.09.2002]
\begin{Definition}[Label=def1, Title=First Definition]
Text innerhalb der Definition (muss zum textFragment werden).
\begin{Program}[Notation=C, Title=First program]
\begin{ProgramFragment}
Programtext in C.
\end{ProgramFragment}
Dieser Text erscheint hinter dem ProgramFragment.
\end{Program}
\end{Definition}
\end{Paragraph}
\end{Section}
\begin{Paragraph}[Label=para2, Title=Second Paragraph, Authors=Achim Mahnke,Date=11.09.2002]
\begin{Table}[Notation=LaTeX, Label= Table1.1.1, Title=Waterfall Model]
{\small
\begin{center}
\begin{tabular}{lcl}
               & Requirement Elicitation & \\ 
               & and Analysis            &\\
& $\uparrow$ $\downarrow$ &\\
Nat.~Lang.   & Non Formal Specification &\\
& $\uparrow$ $\downarrow$ & Validation \\

\Emphasis{Spec.~Lang.}  & Formal Specification & \\
& $\uparrow$ $\downarrow$ & ``Inv.~\& Verify'' or\\

Progr.~Lang. & Implementation & Transformation \\
& $\uparrow$ $\downarrow$ &\\

& Test & \\

& $\uparrow$ $\downarrow$ &\\

& Maintenance& \\

\end{tabular}
\end{center}
}
\end{Table}
\begin{List}[Label=List1.1.1, ListType=itemize]
\ListItem
{\Emphasis{ writing} formal specifications}
\ListItem
{\Emphasis{ proving} properties about formal specifications}
\ListItem
{\Emphasis{ constructing} a program by mathematical \\
            manipulation of a formal specification}
\ListItem
{\Emphasis{ verifying} a program by mathematical argument}
\end{List}
\end{Paragraph}
\IncludeGroup{para1}{}
\end{Section}
\end{Package}
\end{document}

