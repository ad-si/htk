\documentclass[landscape, slides, light]{mmiss2}

\usepackage{latexsym}
\usepackage{amssymb}
\usepackage{casl}

\newcommand{\gram}[1]{{\texttt{#1}}}
\newcommand{\power}{\hat{\ }}
\newcommand{\annoimplies}{\%{\bf implies}\ }
\newcommand{\annodef}{\%{\bf def}\ }
\newcommand{\annocons}{\%{\bf conservatively}\ }
\newcommand{\annoprec}{\%{\bf prec}\ }
\newcommand{\annorassoc}{\%{\bf right\ assoc}\ }
\newcommand{\annolassoc}{\%{\bf left\ assoc}\ }
\newcommand{\annobrackets}{{\bf brackets}\ }
\newcommand{\Nat}{\mbox{$\mathbb N$}} % this requires amssymb


\begin{document}
\begin{Package}[MMISS_Latex2]{Root}
{Formal Program Development with Algebraic Specifications}
{Date=12.09.2002,
Authors={Christoph L\"uth, Markus Roggenbach},
PriorAuthors={Markus Roggenbach, Michael Drouineaud},
ShortAuthor={C.~L\"uth, M.~Roggenbach},
ShortTitle={Algebraic Specification},
Language=en-GB,LevelOfDetail=LectureNotes,InteractionLevel=Hyper}

\begin{Section}{IntroAlgSpec}{Signatures, Terms and Algebras}{}

\begin{Abstract}{Abstract}{Overview}{}
This section will introduce the basic mathematics of algebraic
specifications, namely:
\begin{List}{}{itemize}{}
  \ListItem{} \Emphasis{signatures} and \Emphasis{terms} give us the syntax,
  \ListItem{} and \Emphasis{algebras} give us the semantics.
\end{List}  
\end{Abstract}

\begin{Introduction}{Intro}{Why Formal Specification?}{}
\begin{List}{Reasons}{itemize}{}
\ListItem{}
formal specifications are \Emphasis{precise}  \\
(non formal and sometimes even semi formal \\
specifications are open to re-interpretation)\pause
\ListItem{}
syntactical and semantical \Emphasis{correctness} \\ 
independent of tools 
\pause
\ListItem{}
\Emphasis{mathematical methods} \\
(consistency, completeness)
\end{List}
\end{Introduction}

\begin{Paragraph}{Signatures}{Signatures}{}
Signatures allow us to define operations.

\begin{Definition}{DefSignatures}{Signatures}{}
A \Emphasis{Signature} $\Sigma=(S, \Omega)$ is given by 
\begin{List}{SignatureComponents}{itemize}{}
    \ListItem{} a set $S$ of \Emphasis{sorts}, and 
    \ListItem{} a family $\{\Omega_{w,s}\}_{w\in S^{*},s\in S}$ of operations. 
  \end{List}
\end{Definition}

An operation $\omega\in\Omega_{w,s}$ with $w=s_1\ldots s_n$ has
argument sorts $s_1,\ldots,s_n$ and target sort $s$, also written
$\omega:s_1\ldots s_n\rightarrow s$.
\end{Paragraph}

\begin{Paragraph}{Algebras}{Algebras}{}
Algebras are models of signatures.

\begin{Definition}{DefAlgebra}{$\Sigma$-Algebra}{}
  An \Emphasis{Algebra} $A= (S_A, \Omega_A)$ for a signature
  $\Sigma=(S, \Omega)$ ($\Sigma$-Algebra) is given by 
  \begin{List}{AlgebraComponents}{itemize}{}
    \ListItem{}
    for each sort $s\in S$, a \Emphasis{carrier set} 
    $A_s\in S_A$;
    \ListItem{}
    for each operation $\omega:s_1\ldots s_n\rightarrow s$, an
    operation $\omega_A:A_{s1}\ldots A_{sn}\rightarrow A_s$.
  \end{List}
\end{Definition}
\end{Paragraph}


\begin{Paragraph}{Terms}{Terms over a Set}{}
Let $\Sigma=(S, \Omega)$ be a signature. 

An \Emphasis{$S$-sorted set of variables} is given by $X=
\{X_s\}_{s\in S}$.

Intuitively: terms are the smallest set containing the variables which
is closed under application of operations.                                
\end{Paragraph}

\begin{Paragraph}{Example}{A Simple Example}{}

An Example (in CASL): the natural numbers.

\begin{ProgramFragment}[CASL_Spec]{CASL_Nats}{Natural Numbers}{}
\begin{SpecDefn}{Nat} =
\I{}
\begin{Items}
\I\Free\Types \
\(\[
Nat & \ ::= & \ 0 \, |  \, \Sort \ Pos; \\
Pos & \ ::= & \ suc(pre:Nat)
\]\)
\end{Items}
\I\End
\end{SpecDefn}
\end{ProgramFragment}

Note how structural induction on $\Nat$ corresponds to the usual
natural induction.
\end{Paragraph}

\begin{Summary}{Summary}{Summing Up}{}
In this section, we have encountered the following concepts:
\begin{List}{Concepts}{itemize}{}
\ListItem{}
\emph{Signatures} allow to declare sorts and operations;
\ListItem{}
\emph{Terms} are build over signatures;
\ListItem{}
\emph{Algebras} give a semantics to signatures.
\ListItem{}
Further, we have seen how to specify natural numbers in the algebraic
specification language CASL.
\end{List}
\end{Summary}

\end{Section}


\end{Package}
\end{document}














