\documentclass[landscape, slides, light]{mmiss}

\usepackage{latexsym}
\usepackage{amssymb}
\usepackage{mmiss-casl}

\newcommand{\gram}[1]{{\texttt{#1}}}
\newcommand{\power}{\hat{\ }}
\newcommand{\annoimplies}{\%{\bf implies}\ }
\newcommand{\annodef}{\%{\bf def}\ }
\newcommand{\annocons}{\%{\bf conservatively}\ }
\newcommand{\annoprec}{\%{\bf prec}\ }
\newcommand{\annorassoc}{\%{\bf right\ assoc}\ }
\newcommand{\annolassoc}{\%{\bf left\ assoc}\ }
\newcommand{\annobrackets}{{\bf brackets}\ }
\newcommand{\Nat}{\mbox{$\mathbb N$}} % this requires amssymb
\def\ShortAuthors{C.~L{\"u}th, M.~Roggenbach}
\def\ShortTitle{Algebraic Specification}
\def\Title{Formal Program Development with Algebraic Specifications}


\begin{document}
\begin{Package}[Label=AlgSpec,
  Title={Formal Program Development with Algebraic Specifications},
  Date=12.09.2002,
  Authors={Christoph L{\"u}th, Markus Roggenbach},
  PriorAuthors={Markus Roggenbach, Michael Drouineaud},
  ShortAuthors={C.~L{\"u}th, M.~Roggenbach},
  ShortTitle={Algebraic Specification},
  Language=en-GB,LevelOfDetail=Lecture,InteractionLevel=Hyper]

\begin{Section}[Label=IntroAlgSpec, Title={Signatures, Terms and Algebras}]

\begin{Abstract}[Label=Abstract, Title=Overview]
This section will introduce the basic mathematics of algebraic
specifications, namely:
\begin{Enumerate}
  \item \Emphasis{signatures} and \Emphasis{terms} give us the syntax,
  \item and \Emphasis{algebras} give us the semantics.
\end{Enumerate}  
\end{Abstract}

\begin{Introduction}[Label=Intro, Title={Why Formal Specification?}]
\begin{Itemize}
\item
formal specifications are \Emphasis{precise}  \\
(non formal and sometimes even semi formal \\
specifications are open to re-interpretation)\pause
\item
syntactical and semantical \Emphasis{correctness} \\ 
independent of tools 
\pause
\item{}
\Emphasis{mathematical methods} \\
(consistency, completeness)
\end{Itemize}
\end{Introduction}


\begin{Paragraph}[Label=Signatures, Title=Signatures]
Signatures allow us to define operations.
\vspace{1cm}

\begin{Definition}[Label=DefSignature, Title=Signature]
A \Emphasis{Signature} $\Sigma=(S, \Omega)$ is given by 
\begin{Itemize}
    \item a set $S$ of \Emphasis{sorts}, and 
    \item a family $\{\Omega_{w,s}\}_{w\in S^{*},s\in S}$ of operations. 
  \end{Itemize}
What about \Reference[algebras]{DefAlgebra}?
\end{Definition}

An operation $\omega\in\Omega_{w,s}$ with $w=s_1\ldots s_n$ has
argument sorts $s_1,\ldots,s_n$ and target sort $s$, also written
$\omega:s_1\ldots s_n\to s$.
\end{Paragraph}

\begin{Paragraph}[Label=Algebra, Title=Algebra]

Algebras are models of \Reference[signatures]{DefSignature}.
\vspace{1cm}
\begin{Definition}[Label=DefAlgebra, Title={$\Sigma$-Algebra}]
  An \Emphasis{Algebra} $A= (S_A, \Omega_A)$ for a
  \Reference[signatures]{DefSignature} 
  $\Sigma=(S, \Omega)$ ($\Sigma$-Algebra) is given by 
  \begin{Itemize}
    \item
    for each sort $s\in S$, a \Emphasis{carrier set} 
    $A_s\in S_A$;
    \item
    for each operation $\omega:s_1\ldots s_n\to s$, an
    operation $\omega_A:A_{s1}\ldots A_{sn}\to A_s$.
  \end{Itemize}
\end{Definition}
\end{Paragraph}


\begin{Paragraph}[Label=Terms, Title={Terms over a Set}]
Let $\Sigma=(S, \Omega)$ be a \Reference[signature]{DefSignature}.

An \Emphasis{$S$-sorted set of variables} is given by $X=
\{X_s\}_{s\in S}$.

Intuitively: terms are the smallest set containing the variables which
is closed under application of operations.
\vspace{1cm}

\begin{Lemma}[Label=TermAlgebra, Title={The Term Algebra}]
Terms form an \Reference[algebra]{DefAlgebra}.  
\end{Lemma}

\end{Paragraph}

\begin{Paragraph}[Label=Example, Title=A Simple Example]

An Example (in CASL): the natural numbers.

%% \IncludeProgramFragment{CASL_Nats}{}
\begin{ProgramFragment}[Formalism=CASL_Spec, Label=CASL_Nats, Title={Natural Numbers}]
\begin{SpecDefn}{Nat} =
\I{}
\begin{Items}
\I\Free\Types \
\(\[
Nat & \ ::= & \ 0 \, |  \, \Sort \ Pos; \\
Pos & \ ::= & \ suc(pre:Nat)
\]\)
\end{Items}
\I\End
\end{SpecDefn}
\end{ProgramFragment}

Note how $(\Nat, 0, +)$ form an \Reference[algebra]{DefAlgebra}.
\end{Paragraph}

\begin{Summary}[Label=Summary, Title={Summing Up}]
In this section, we have encountered the following concepts:
\begin{Itemize}
\item
\emph{Signatures} allow to declare sorts and operations;
\item
\emph{Terms} are build over signatures;
\item
\emph{Algebras} give a semantics to signatures.
\item
Further, we have seen how to specify natural numbers in the algebraic
specification language CASL.
\end{Itemize}
\end{Summary}
\end{Section}
\end{Package}
\end{document}














